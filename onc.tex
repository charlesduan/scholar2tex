\documentclass[twoside,11pt]{article}
\usepackage{casemacs}

\citation{84 Fed. Reg. 7424 (March 4, 2019)}
\caption{DEF}
\shortcaption{}
\docket{JKL}
\court{MNO}

\begin{document}
\thispagestyle{empty}
\begin{center}\bfseries Proposed Rule\\84 Fed. Reg. 7424\\March 4, 2019
\end{center}

        
           \pagenum{7424}\ifhmode\expandafter\xspace\fi 
          \begin{center}
\bfseries
DEPARTMENT OF HEALTH AND HUMAN SERVICES\end{center}


          \begin{center}
\bfseries
Office of the Secretary\end{center}


          \begin{center}
\bfseries
45 CFR Parts 170 and 171\end{center}


          \begin{center}
\bfseries
RIN 0955-AA01\end{center}


          \begin{center}
\bfseries
21st Century Cures Act: Interoperability, Information Blocking, and the ONC Health IT Certification Program\end{center}


          
            \paragraph{AGENCY:}

            Office of the National Coordinator for Health Information Technology (ONC), Department of Health and Human Services (HHS).


          
          
            \paragraph{ACTION:}

            Proposed rule.


          
          
            \paragraph{SUMMARY:}

            This proposed rule would implement certain provisions of the 21st Century Cures Act, including conditions and maintenance of certification requirements for health information technology (health IT) developers under the ONC Health IT Certification Program (Program), the voluntary certification of health IT for use by pediatric health care providers, and reasonable and necessary activities that do not constitute information blocking. The implementation of these provisions would advance interoperability and support the access, exchange, and use of electronic health information. The proposed rule would also modify the 2015 Edition health IT certification criteria and Program in additional ways to advance interoperability, enhance health IT certification, and reduce burden and costs.


          
          
            \paragraph{DATES:}

            To be assured consideration, written or electronic comments must be received at one of the addresses provided below, no later than 5 p.m. on May 3, 2019.


          
          
            \paragraph{ADDRESSES:}

            You may submit comments, identified by RIN 0955-AA01, by any of the following methods (please do not submit duplicate comments). Because of staff and resource limitations, we cannot accept comments by facsimile (FAX) transmission.


            • \emph{Federal eRulemaking Portal:} Follow the instructions for submitting comments. Attachments should be in Microsoft Word, Microsoft Excel, or Adobe PDF; however, we prefer Microsoft Word. \emph{\url{http://www.regulations.gov.}}
            


            • \emph{Regular, Express, or Overnight Mail:} Department of Health and Human Services, Office of the National Coordinator for Health Information Technology, Attention: 21st Century Cures Act: Interoperability, Information Blocking, and the ONC Health IT Certification Program Proposed Rule, Mary E. Switzer Building, Mail Stop: 7033A, 330 C Street SW, Washington, DC 20201. Please submit one original and two copies.


            • \emph{Hand Delivery or Courier:} Office of the National Coordinator for Health Information Technology, Attention: 21st Century Cures Act: Interoperability, Information Blocking, and the ONC Health IT Certification Program Proposed Rule, Mary E. Switzer Building, Mail Stop: 7033A, 330 C Street SW, Washington, DC 20201. Please submit one original and two copies. (Because access to the interior of the Mary E. Switzer Building is not readily available to persons without federal government identification, commenters are encouraged to leave their comments in the mail drop slots located in the main lobby of the building.)


            
              \emph{Enhancing the Public Comment Experience:} To facilitate public comment on this proposed rule, a copy will be made available in Microsoft Word format on ONC's website (\emph{\url{http://www.healthit.gov}}). We believe this version will make it easier for commenters to access and copy portions of the proposed rule for use in their individual comments. Additionally, a separate document (“public comment template”) will also be made available on ONC's website (\emph{\url{http://www.healthit.gov}}) for the public to use in providing comments on the proposed rule. This document is meant to provide the public with a simple and organized way to submit comments on proposals and respond to specific questions posed in the preamble of the proposed rule. While use of this document is entirely voluntary, we encourage commenters to consider using the document in lieu of unstructured comments, or to use it as an addendum to narrative cover pages. We believe that use of the document may facilitate our review and understanding of the comments received. The public comment template will be available shortly after the proposed rule publishes in the \textbf{Federal Register}. This short delay will permit the appropriate citation in the public comment template to pages of the published version of the proposed rule.


            
              \emph{Inspection of Public Comments:} All comments received before the close of the comment period will be available for public inspection, including any personally identifiable or confidential business information that is included in a comment. Please do not include anything in your comment submission that you do not wish to share with the general public. Such information includes, but is not limited to: A person's social security number; date of birth; driver's license number; state identification number or foreign country equivalent; passport number; financial account number; credit or debit card number; any personal health information; or any business information that could be considered proprietary. We will post all comments that are received before the close of the comment period at \emph{\url{http://www.regulations.gov.}}
            


            
              \emph{Docket:} For access to the docket to read background documents or comments received, go to \emph{\url{http://www.regulations.gov}} or the Department of Health and Human Services, Office of the National Coordinator for Health Information Technology, Mary E. Switzer Building, Mail Stop: 7033A, 330 C Street SW, Washington, DC 20201 (call ahead to the contact listed below to arrange for inspection).


          
          
            \paragraph{FOR FURTHER INFORMATION CONTACT:}

            Michael Lipinski, Office of Policy, Office of the National Coordinator for Health Information Technology, 202-690-7151.


          
        
        
          \paragraph{SUPPLEMENTARY INFORMATION:}

          \section{Table of Contents}

          
            \indentpar{0}{I}{Executive Summary}


            \indentpar{1}{A}{Purpose of Regulatory Action}


            \indentpar{1}{B}{Summary of Major Provisions and Clarifications}


            \indentpar{2}{1}{Deregulatory Actions for Previous Rulemakings}


            \indentpar{2}{2}{Updates to the 2015 Edition Certification Criteria}


            \indentpar{3}{a}{Adoption of the United States Core Data for Interoperability as a Standard}


            \indentpar{3}{b}{Electronic Prescribing}


            \indentpar{3}{c}{Clinical Quality Measures—Report}


            \indentpar{3}{d}{Electronic Health Information Export}


            \indentpar{3}{e}{Application Programming Interfaces}


            \indentpar{3}{f}{Privacy and Security Transparency Attestations}


            \indentpar{3}{g}{Data Segmentation for Privacy and Consent Management}


            \indentpar{2}{3}{Modifications to the ONC Health IT Certification Program}


            \indentpar{2}{4}{Health IT for the Care Continuum}


            \indentpar{2}{5}{Conditions and Maintenance of Certification}


            \indentpar{2}{6}{Information Blocking}


            \indentpar{1}{C}{Costs and Benefits}


            \indentpar{0}{II}{Background}


            \indentpar{1}{A}{Statutory Basis}


            \indentpar{2}{1}{Standards, Implementation Specifications, and Certification Criteria}


            \indentpar{2}{2}{Health IT Certification Program(s)}


            \indentpar{1}{B}{Regulatory History}


            \indentpar{2}{1}{Standards, Implementation Specifications, and Certification Criteria Rules}


            \indentpar{2}{2}{ONC Health IT Certification Program Rules}


            \indentpar{0}{III}{Deregulatory Actions for Previous Rulemakings}


            \indentpar{1}{A}{Background}


            \indentpar{2}{1}{History of Burden Reduction and Regulatory Flexibility}


            \indentpar{2}{2}{Executive Orders 13771 and 13777 \pagenum{7425}\ifhmode\expandafter\xspace\fi 
            }


            \indentpar{1}{B}{Proposed Deregulatory Actions}


            \indentpar{2}{1}{Removal of Randomized Surveillance Requirements}


            \indentpar{2}{2}{Removal of the 2014 Edition From the Code of Federal Regulations}


            \indentpar{2}{3}{Removal of the ONC-Approved Accreditor From the Program}


            \indentpar{2}{4}{Removal of Certain 2015 Edition Certification Criteria and Standards}


            \indentpar{3}{a}{2015 Edition Base EHR Definition Certification Criteria}


            \indentpar{3}{b}{Drug-Formulary and Preferred Drug Lists}


            \indentpar{3}{c}{Patient-Specific Education Resources}


            \indentpar{3}{d}{Common Clinical Data Set Summary Record—Create; and Common Clinical Data Set Summary Record—Receive}


            \indentpar{3}{e}{Secure Messaging}


            \indentpar{2}{5}{Removal of Certain ONC Health IT Certification Program Requirements}


            \indentpar{3}{a}{Limitations Disclosures}


            \indentpar{3}{b}{Transparency and Mandatory Disclosures Requirements}


            \indentpar{2}{6}{Recognition of Food and Drug Administration Processes}


            \indentpar{3}{a}{FDA Software Pre-Certification Pilot Program}


            \indentpar{3}{b}{Development of Similar Independent Program Processes—Request for Information}


            \indentpar{0}{IV}{Updates to the 2015 Edition Certification Criteria}


            \indentpar{1}{A}{Standards and Implementation Specifications}


            \indentpar{2}{1}{National Technology Transfer and Advancement Act}


            \indentpar{2}{2}{Compliance with Adopted Standards and Implementation Specifications}


            \indentpar{2}{3}{“Reasonably Available” to Interested Parties}


            \indentpar{1}{B}{Revised and New 2015 Edition Criteria}


            \indentpar{2}{1}{The United States Core Data for Interoperability Standard (USCDI)}


            \indentpar{3}{a}{USCDI 2015 Edition Certification Criteria}


            \indentpar{3}{b}{USCDI Standard—Data Classes Included}


            \indentpar{3}{c}{USCDI Standard—Relationship to Content Exchange Standards and Implementation Specifications}


            \indentpar{3}{d}{Clinical Notes C-CDA Implementation Specification}


            \indentpar{2}{2}{Electronic Prescribing Criterion}


            \indentpar{2}{3}{Clinical Quality Measures—Report Criterion}


            \indentpar{2}{4}{Electronic Health Information Export Criterion}


            \indentpar{3}{a}{Patient Access}


            \indentpar{3}{b}{Transitions Between Health IT Systems}


            \indentpar{3}{c}{Scope of EHI}


            \indentpar{3}{d}{Export Format}


            \indentpar{3}{e}{Initial Step to Persistent Access to All of a Patient's EHI}


            \indentpar{3}{f}{Timeframes}


            \indentpar{3}{g}{Replaces the 2015 Edition “Data Export” Criterion in the 2015 Edition Base EHR Definition}


            \indentpar{2}{5}{Standardized API for Patient and Population Services Criterion}


            \indentpar{2}{6}{Privacy and Security Transparency Attestations Criteria}


            \indentpar{3}{a}{Background}


            \indentpar{3}{b}{Encrypt Authentication Credentials}


            \indentpar{3}{c}{Multi-factor Authentication}


            \indentpar{2}{7}{Data Segmentation for Privacy and Consent Management Criteria}


            \indentpar{3}{a}{Implementation With the Consolidated CDA Release 2.1}


            \indentpar{3}{b}{Implementation With FHIR Standard}


            \indentpar{1}{C}{Unchanged 2015 Edition Criteria—Promoting Interoperability Programs Reference Alignment}


            \indentpar{0}{V}{Modifications to the ONC Health IT Certification Program}


            \indentpar{1}{A}{Corrections}


            \indentpar{2}{1}{Auditable Events and Tamper Resistance}


            \indentpar{2}{2}{Amendments}


            \indentpar{2}{3}{View, Download, and Transmit to 3rd Party}


            \indentpar{2}{4}{Integrating Revised and New Certification Criteria Into the 2015 Edition Privacy and Security Certification Framework}


            \indentpar{1}{B}{Principles of Proper Conduct for ONC-ACBs}


            \indentpar{2}{1}{Records Retention}


            \indentpar{2}{2}{Conformance Methods for Certification Criteria}


            \indentpar{2}{3}{ONC-ACBs to Accept Test Results From Any ONC-ATL in Good Standing}


            \indentpar{2}{4}{Mandatory Disclosures and Certifications}


            \indentpar{1}{C}{Principles of Proper Conduct for ONC-ATLs—Records Retention}


            \indentpar{0}{VI}{Health IT for the Care Continuum}


            \indentpar{1}{A}{Health IT for Pediatric Setting}


            \indentpar{2}{1}{Background and Stakeholder Convening}


            \indentpar{2}{2}{Recommendations for the Voluntary Certification of Health IT for Use in Pediatric Care}


            \indentpar{3}{a}{2015 Edition Certification Criteria}


            \indentpar{3}{b}{New or Revised Certification Criteria in This Proposed Rule}


            \indentpar{1}{B}{Health IT and Opioid Use Disorder Prevention and Treatment—Request for Information}


            \indentpar{2}{1}{2015 Edition Certification Criteria}


            \indentpar{2}{2}{Revised or New 2015 Edition Certification Criteria in This Proposed Rule}


            \indentpar{2}{3}{Emerging Standards and Innovations}


            \indentpar{2}{4}{Additional Comment Areas}


            \indentpar{0}{VII}{Conditions and Maintenance of Certification}


            \indentpar{1}{A}{Implementation}


            \indentpar{1}{B}{Provisions}


            \indentpar{2}{1}{Information Blocking}


            \indentpar{2}{2}{Assurances}


            \indentpar{3}{a}{Full Compliance and Unrestricted Implementation of Certification Criteria Capabilities}


            \indentpar{3}{b}{Certification to the “Electronic Health Information Export” Criterion}


            \indentpar{3}{c}{Records and Information Retention}


            \indentpar{3}{d}{Trusted Exchange Framework and the Common Agreement—Request for Information}


            \indentpar{2}{3}{Communications}


            \indentpar{3}{a}{Background and Purpose}


            \indentpar{3}{b}{Condition of Certification Requirements}


            \indentpar{3}{c}{Maintenance of Certification Requirements}


            \indentpar{2}{4}{Application Programming Interfaces}


            \indentpar{3}{a}{Statutory Interpretation and API Policy Principles}


            \indentpar{3}{b}{Key Terms}


            \indentpar{3}{c}{Proposed API Standards, Implementation Specifications, and Certification Criterion}


            \indentpar{3}{d}{Condition of Certification Requirements}


            \indentpar{3}{e}{Maintenance of Certification Requirements}


            \indentpar{3}{f}{2015 Edition Base EHR Definition}


            \indentpar{2}{5}{Real World Testing}


            \indentpar{2}{6}{Attestations}


            \indentpar{2}{7}{EHR Reporting Criteria Submission}


            \indentpar{1}{C}{Compliance}


            \indentpar{1}{D}{Enforcement}


            \indentpar{2}{1}{ONC Direct Review of the Conditions and Maintenance of Certification Requirements}


            \indentpar{2}{2}{Review and Enforcement Only by ONC}


            \indentpar{2}{3}{Review Processes}


            \indentpar{3}{a}{Initiating Review and Health IT Developer Notice}


            \indentpar{3}{b}{Relationship with ONC-ACBs and ONC-ATLs}


            \indentpar{3}{c}{Records Access}


            \indentpar{3}{d}{Corrective Action}


            \indentpar{3}{e}{Certification Ban and Termination}


            \indentpar{3}{f}{Appeal}


            \indentpar{3}{g}{Suspension}


            \indentpar{3}{h}{Proposed Termination}


            \indentpar{2}{4}{Public Listing of Certification Ban and Terminations}


            \indentpar{2}{5}{Effect on Existing Program Requirements and Processes}


            \indentpar{2}{6}{Concurrent Enforcement by the Office of Inspector General}


            \indentpar{2}{7}{Applicability of Conditions and Maintenance of Certification Requirements for Self-Developers}


            \indentpar{0}{VIII}{Information Blocking}


            \indentpar{1}{A}{Statutory Basis}


            \indentpar{1}{B}{Legislative Background and Policy Considerations}


            \indentpar{2}{1}{Purpose of the Information Blocking Provision}


            \indentpar{2}{2}{Policy Considerations and Approach to the Information Blocking Provisions}


            \indentpar{1}{C}{Relevant Statutory Terms and Provisions}


            \indentpar{2}{1}{“Required by Law”}


            \indentpar{2}{2}{Health Care Providers, Health IT Developers, Exchanges, and Networks}


            \indentpar{3}{a}{Health Care Providers}


            \indentpar{3}{b}{Health IT Developers of Certified Health IT}


            \indentpar{3}{c}{Networks and Exchanges}


            \indentpar{2}{3}{Electronic Health Information}


            \indentpar{2}{4}{Interests Promoted by the Information Blocking Provision}


            \indentpar{3}{a}{Access, Exchange, and Use of EHI}


            \indentpar{3}{b}{Interoperability Elements}


            \indentpar{2}{5}{Practices That May Implicate the Information Blocking Provision}


            \indentpar{3}{a}{Prevention, Material Discouragement, and Other Interference}


            \indentpar{3}{b}{Likelihood of Interference}


            \indentpar{3}{c}{Examples of Practices Likely To Interfere With Access, Exchange, or Use of EHI}


            \indentpar{2}{6}{Applicability of Exceptions}


            \indentpar{3}{a}{Reasonable and Necessary Activities}


            \indentpar{3}{b}{Treatment of Different Types of Actors}


            \indentpar{3}{c}{Establishing That Activities and Practices Meet the Conditions of an Exception}


            \indentpar{1}{D}{Proposed Exceptions to the Information Blocking Provision}


            \indentpar{2}{1}{Preventing Harm}


            \indentpar{2}{2}{Promoting the Privacy of EHI}


            \indentpar{2}{3}{Promoting the Security of EHI}


            \indentpar{2}{4}{Recovering Costs Reasonably Incurred}


            \indentpar{2}{5}{Responding to Requests That Are Infeasible}


            \indentpar{2}{6}{Licensing of Interoperability Elements on Reasonable and Non-discriminatory Terms}


            \indentpar{2}{7}{Maintaining and Improving Health IT Performance}



            \indentpar{1}{E}{Additional Exceptions—Request for Information \pagenum{7426}\ifhmode\expandafter\xspace\fi 
            }


            \indentpar{2}{1}{Exception for Complying With Common Agreement for Trusted Exchange}


            \indentpar{2}{2}{New Exceptions}


            \indentpar{1}{F}{Complaint Process}


            \indentpar{1}{G}{Disincentives for Health Care Providers—Request for Information}


            \indentpar{0}{IX}{Registries Request for Information}


            \indentpar{0}{X}{Patient Matching Request for Information}


            \indentpar{0}{XI}{Incorporation by Reference}


            \indentpar{0}{XII}{Response to Comments}


            \indentpar{0}{XIII}{Collection of Information Requirements}


            \indentpar{1}{A}{ONC-ACBs}


            \indentpar{1}{B}{Health IT Developers}


            \indentpar{0}{XIV}{Regulatory Impact Analysis}


            \indentpar{1}{A}{Statement of Need}


            \indentpar{1}{B}{Alternatives Considered}


            \indentpar{1}{C}{Overall Impact}


            \indentpar{2}{1}{Executive Orders 12866 and 13563—Regulatory Planning and Review Analysis}


            \indentpar{2}{2}{Executive Order 13771—Reducing Regulation and Controlling Regulatory Costs}


            \indentpar{3}{a}{Costs and Benefits}


            \indentpar{3}{b}{Accounting Statement and Table}


            \indentpar{2}{3}{Regulatory Flexibility Act}


            \indentpar{2}{4}{Executive Order 13132—Federalism}


            \indentpar{2}{5}{Unfunded Mandates Reform Act of 1995}


            \indentpar{2}{6}{Executive Order 13771 Reducing Regulation and Controlling Regulatory Costs}


            Regulation Text


          
          \section{I. Executive Summary}

          \subsection{A. Purpose of Regulatory Action}

          ONC is responsible for the implementation of key provisions in Title IV of the 21st Century Cures Act (Cures Act) that are designed to advance interoperability; support the access, exchange, and use of electronic health information; and address occurrences of information blocking. This proposed rule would implement certain provisions of the Cures Act, including Conditions and Maintenance of Certification requirements for health information technology (health IT) developers, the voluntary certification of health IT for use by pediatric health providers, and reasonable and necessary activities that do not constitute information blocking. In addition, the proposed rule would implement parts of section 4006(a) of the Cures Act to support patient access to their electronic health information (EHI), such as making a patient's EHI more electronically accessible through the adoption of standards and certification criteria and the implementation of information blocking policies that support patient electronic access to their health information at no cost. Additionally, the proposed rule would modify the 2015 Edition health IT certification criteria and ONC Health IT Certification Program (Program) in other ways to advance interoperability, enhance health IT certification, and reduce burden and costs.


          In addition to fulfilling the Cures Act's requirements, the proposed rule would contribute to fulfilling Executive Order (E.O.) 13813. The President issued E.O. 13813 on October 12, 2017, to promote health care choice and competition across the United States. Section 1(c) of the E.O., in relevant part, states that government rules affecting the United States health care system should re-inject competition into the health care markets by lowering barriers to entry and preventing abuses of market power. Section 1(c) also states that government rules should improve access to and the quality of information that Americans need to make informed health care decisions. For example, as mentioned above, the proposed rule focuses on establishing Application Programming Interfaces (APIs) for several interoperability purposes, including patient access to their health information without special effort. The API approach also supports health care providers having the sole authority and autonomy to unilaterally permit connections to their health IT through certified API technology the health care providers have acquired. In addition, the proposed rule provides ONC's interpretation of the information blocking definition as established in the Cures Act and the application of the information blocking provision by identifying reasonable and necessary activities that would not constitute information blocking. Many of these activities focus on improving patient and health care provider access to electronic health information and promoting competition.


          \subsection{B. Summary of Major Provisions and Clarifications}

          \subsubsection{1. Deregulatory Actions for Previous Rulemakings}

          Since the inception of the Program, we have aimed to implement and administer the Program in the least burdensome manner that supports our policy goals. Throughout the years, we have worked to improve the Program with a focus on ways to reduce burden, offer flexibility to both developers and providers, and support innovation. This approach has been consistent with the principles of Executive Order 13563 on Improving Regulation and Regulatory Review (February 2, 2011), which instructs agencies to “determine whether any [agency] regulations should be modified, streamlined, expanded, or repealed so as to make the agency's regulatory program more effective or less burdensome in achieving the regulatory objectives.” To that end, we have historically, where feasible and appropriate, taken measures to reduce burden within the Program and make the Program more effective, flexible, and streamlined.


          ONC has reviewed and evaluated existing regulations to identify ways to administratively reduce burden and implement deregulatory actions through guidance. In this proposed rule, we also propose potential new deregulatory actions that will reduce burden for health IT developers, providers, and other stakeholders. We propose six deregulatory actions in section III.B: (1) Removal of a threshold requirement related to randomized surveillance which allows ONC-Authorized Certification Bodies (ONC-ACBs) more flexibility to identify the right approach for surveillance actions, (2) removal of the 2014 Edition from the Code of Federal Regulations (CFR), (3) removal of the ONC-Approved Accreditor (ONC-AA) from the Program, (4) removal of certain 2015 Edition certification criteria, (5) removal of certain Program requirements, and (6) recognition of relevant Food and Drug Administration certification processes with a request for comment on the potential development of new processes for the Program.


          \subsubsection{2. Updates to the 2015 Edition Certification Criteria}

          This rule proposes to update the 2015 Edition by not only proposing criteria for removal, but by proposing to revise and add new certification criteria that would establish the capabilities and related standards and implementation specifications for the certification of health IT.


          \subsubsection{a. Adoption of the United States Core Data for Interoperability (USCDI) as a Standard}


          As part of ONC's continued efforts to assure the availability of a minimum baseline of data classes that could be commonly available for interoperable exchange, we adopted the 2015 Edition “Common Clinical Data Set” (CCDS) definition and used the CCDS shorthand in several certification criteria. However, the CCDS definition also began to be colloquially used for many different purposes. As the CCDS definition's relevance grew outside of its regulatory context, it became a symbolic and practical limit to the industry's collective interests to go beyond the CCDS data for access, exchange, and use. In addition, as we move further towards value-based care, the need for the inclusion of additional data classes that go beyond clinical data is necessary. In order to advance interoperability, we propose to remove the CCDS definition and its references  \pagenum{7427}\ifhmode\expandafter\xspace\fi from the 2015 Edition and replace it with the “United States Core Data for Interoperability.” We propose to adopt the USCDI as a standard, naming USCDI Version 1 (USCDI v1) in \textsection{} 170.213 and incorporating it by reference in \textsection{} 170.299. The USCDI standard, if adopted, would establish a set of data classes and constituent data elements that would be required to be exchanged in support of interoperability nationwide. To achieve the goals set forth in the Cures Act, ONC intends to establish and follow a predictable, transparent, and collaborative process to expand the USCDI, including providing stakeholders with the opportunity to comment on the USCDI's expansion. Once the USCDI is adopted in regulation naming USCDI v1, health IT developers would be allowed to take advantage of a flexibility under the Maintenance of Certification real world testing requirements, which we refer to as the “Standards Version Advancement Process” (described in section VII.B.5 of this proposed rule). The Standards Version Advancement Process would permit health IT developers to voluntarily implement and use a new version of an adopted standard, such as the USCDI, so long as the newer version was approved by the National Coordinator through the Standards Version Advancement Process for use in certification.


          \subsubsection{b. Electronic Prescribing}

          We propose to update the electronic prescribing (e-Rx) SCRIPT standard in 45 CFR 170.205(b) to NCPDP SCRIPT 2017071, which would result in a new e-Rx standard eventually becoming the baseline for certification. We also propose to adopt a new certification criterion in \textsection{} 170.315(b)(11) for e-Rx to reflect these updated proposals. ONC and CMS have historically maintained complementary policies of maintaining aligned e-Rx and medical history (MH) standards to ensure that the current standard for certification to the electronic prescribing criterion permits use of the current Part D e-Rx and MH standards. This proposal is made to ensure such alignment as CMS recently finalized its Part D standards to NCPDP SCRIPT 2017071 for e-RX and MH, effective January 1, 2020 (83 FR 16440). In addition to continuing to reference the current transactions included in \textsection{} 170.315(b)(3), in keeping with CMS' final rule, we also propose to require all of the NCPDP SCRIPT 2017071 standard transactions CMS adopted at 42 CFR 423.160(b)(2)(iv).


          \subsubsection{c. Clinical Quality Measures—Report}

          We propose to remove the HL7 Quality Reporting Document Architecture (QRDA) standard requirements from the 2015 Edition “CQMs—report” criterion in \textsection{} 170.315(c)(3) and, in their place, require Health IT Modules to support the CMS QRDA Implementation Guide (IGs).\textsuperscript{1}

             This would reduce the burden for health IT developers by only having to support one form of the QRDA standard rather than two forms (\emph{i.e.,} the HL7 and CMS forms).


          \insfootnote{\textsuperscript{1} \emph{\url{https://ecqi.healthit.gov/qrda-quality-reporting-document-architecture.}}}


          \subsubsection{d. Electronic Health Information Export}

          We propose a new 2015 Edition certification criterion for “electronic health information (EHI) export” in \textsection{} 170.315(b)(10), which would replace the 2015 Edition “data export” certification criterion (\textsection{} 170.315(b)(6)) and become part of the 2015 Edition Base EHR definition. The proposed criterion supports situations in which we believe that all EHI produced and electronically managed by a developer's health IT should be made readily available for export as a standard capability of certified health IT. Specifically, this criterion would: (1) Enable the export of EHI for a single patient upon a valid request from that patient or a user on the patient's behalf, and (2) support the export of EHI when a health care provider chooses to transition or migrate information to another health IT system. This criterion would also require that the export include the data format, made publicly available, to facilitate the receiving health IT system's interpretation and use of the EHI to the extent reasonably practicable using the developer's existing technology.


          This criterion provides developers with the ability to create innovative export capabilities according to their systems and data practices. We do not propose that the export must be executed according to any particular standard, but propose to require that the export must be accompanied by the data format, including its structure and syntax, to facilitate interpretation of the EHI therein. Overall, this new criterion is intended to provide patients and health IT users, including providers, a means to efficiently export the entire electronic health record for a single patient or all patients in a computable, electronic format.


          \subsubsection{e. Application Programming Interfaces (APIs)}

          We propose to adopt a new API criterion in \textsection{} 170.315(g)(10), which would replace the “application access—data category request” certification criterion (\textsection{} 170.315(g)(8)) and become part of the 2015 Edition Base EHR definition. This new “standardized API for patient and population services” certification criterion would require the use of Health Level 7 (HL7®) Fast Healthcare Interoperability Resources (FHIR®) standards \textsuperscript{2}
             and several implementation specifications. The new criterion would focus on supporting two types of API-enabled services: (1) Services for which a single patient's data is the focus and (2) services for which multiple patients' data are the focus.


          \insfootnote{\textsuperscript{2} \emph{\url{https://www.hl7.org/fhir/overview.html.}}}


          \subsubsection{f. Privacy and Security Transparency Attestations}

          We propose to adopt two new privacy and security transparency attestation certification criteria, which would identify whether certified health IT supports encrypting authentication credentials and/or multi-factor authentication. In order to be issued a certification, we propose to require that a Health IT Module developer attest to whether the Health IT Module encrypts authentication credentials and whether the Health IT Module supports multi-factor authentication. These criteria are not expected to place additional burden on health IT developers since they do not require net new development or implementation to take place in order to be met. However, certification to these proposed criteria would provide increased transparency and potentially motivate health IT developers to encrypt authentication credentials and support multi- factor authentication, which could help prevent exposure to unauthorized persons/entities.


          \subsubsection{g. Data Segmentation for Privacy and Consent Management}


          In the 2015 Edition, we adopted two “data segmentation for privacy” (DS4P) certification criteria, one for creating a summary record according to the DS4P standard and one for receiving a summary record according to the DS4P standard. Certification to the 2015 Edition DS4P criteria focus on data segmentation only at the document level. As noted in the 2015 Edition final rule (80 FR 62646)—and to our knowledge still an accurate assessment—certification to these criteria is currently not required to meet the Certified EHR Technology definition  \pagenum{7428}\ifhmode\expandafter\xspace\fi (CEHRT) or required by any other HHS program. Since the 2015 Edition final rule, the health care industry has engaged in additional field testing and implementation of the DS4P standard. In addition, stakeholders shared with ONC—through public forums, listening sessions, and correspondence—that focusing certification on segmentation to only the document level does not permit providers the flexibility to address more granular segmentation needs. Therefore, we propose to remove the current 2015 Edition DS4P criteria. We propose to replace these two criteria with three new 2015 Edition “DS4P” certification criteria (two for C-CDA and one for a FHIR-based API) that would support a more granular approach to privacy tagging data consent management for health information exchange supported by either the C-CDA- or FHIR-based exchange standards.


          \subsubsection{3. Modifications to the ONC Health IT Certification Program}

          We propose to make corrections to the 2015 Edition privacy and security certification framework (80 FR 62705) and relevant regulatory provisions. These corrections have already been incorporated in the relevant Certification Companion Guides (CCGs).


          We propose new and revised principles of proper conduct (PoPC) for ONC-Authorized Certification Bodies (ONC-ACBs). We propose to clarify that the records retention provision includes the “life of the edition” as well as after the retirement of an edition related to the certification of Complete EHRs and Health IT Modules. We also propose to revise the PoPC in \textsection{} 170.523(h) to clarify the basis for certification, including to permit a certification decision to be based on an evaluation conducted by the ONC-ACB for Health IT Modules' compliance with certification criteria by use of conformity methods approved by the National Coordinator for Health Information Technology (National Coordinator). We also propose to update \textsection{} 170.523(h) to require ONC-ACBs to accept test results from any ONC-ATL that is in good standing under the Program and is compliant with its ISO 17025 accreditation requirements. We believe these proposed new and revised PoPCs would provide necessary clarifications for ONC-ACBs and would promote stability among the ONC-ACBs. We also propose to update \textsection{} 170.523(k) to broaden the requirements beyond just the Medicare and Medicaid Electronic Health Record (EHR) Incentive Programs (now renamed the Promoting Interoperability Programs) and provide other necessary clarifications.


          We propose to revise a PoPC for ONC-ATLs. We propose to clarify that the records retention provision includes the “life of the edition” as well as after the retirement of an edition related to the certification of Complete EHRs and Health IT Modules.


          \subsubsection{4. Health IT for the Care Continuum}

          Section 4001(b) of the Cures Act includes two provisions related to supporting health IT across the care continuum. The first instructs the National Coordinator to encourage, keep or recognize through existing authorities, the voluntary certification of health IT for use in medical specialties and sites of service where more technological advancement or integration is needed. The second outlines a provision related to the voluntary certification of health IT for use by pediatric health providers to support the health care of children. These provisions align closely with ONC's core purpose to promote interoperability to support care coordination, patient engagement, and health care quality improvement initiatives. Advancing health IT that promotes and supports patient care when and where it is needed continues to be a primary goal of the Program. This means health IT should support patient populations, specialized care, transitions of care, and practice settings across the care continuum.


          ONC has explored how we might work with the health IT industry and with specialty organizations to collaboratively develop and promote health IT that supports medical specialties and sites of service. Over time, ONC has taken steps to make the Program modular, more open and accessible to different types of health IT, and able to advance functionality that is generally applicable to a variety of care and practice settings. Specific to the provisions in the Cures Act to support providers of health care for children, we considered a wide range of factors. These include: The evolution of health IT across the care continuum, the costs and benefits associated with health IT, the potential regulatory burden and compliance timelines, and the need to help advance health IT that benefits multiple medical specialties and sites of service involved in the care of children. In consideration of these factors, and to advance implementation of Sections 4001(b) of the Cures Act specific to pediatric care, we held a listening session where stakeholders could share their clinical knowledge and technical expertise in pediatric care and pediatric sites of service. Through the information learned at this listening session and our analysis of the health IT landscape for pediatric settings, we have identified existing 2015 Edition criteria, as well as new and revised 2015 Edition criteria proposed in this rule, that we believe could benefit providers of pediatric care and pediatric settings. In this proposed rule, we seek comment on our analysis and the correlated certification criteria that we believe would support the health care of children.


          We also recognize the significance of the opioid epidemic confronting our nation and the importance of helping to support the health IT needs of health care providers committed to preventing inappropriate access to prescription opioids and to providing safe, appropriate treatment. We believe health IT offers promising strategies to help assist medical specialties and sites of services impacted by the opioid epidemic. Therefore, we request public comment on how our existing Program requirements and the proposals in this rulemaking may support use cases related to Opioid Use Disorder (OUD) prevention and treatment and if there are additional areas that ONC should consider for effective implementation of health IT to help address OUD prevention and treatment.


          \subsubsection{5. Conditions and Maintenance of Certification}


          We propose to establish certain Conditions and Maintenance of Certification requirements for health IT developers based on the conditions and maintenance of certification requirements outlined in section 4002 of the Cures Act. We propose an approach whereby the Conditions and Maintenance of Certification express both initial requirements for health IT developers and their certified Health IT Module(s) as well as ongoing requirements that must be met by both health IT developers and their certified Health IT Module(s) under the Program. In this regard, we propose to implement the Cures Act Conditions of Certification with further specificity as it applies to the Program and propose to implement any accompanying Maintenance of Certification requirements as standalone requirements to ensure that not only are the Conditions of Certification met, but that they are continually being met through the Maintenance of Certification requirements. For ease of reference and to distinguish from other conditions, we propose to capitalize “Conditions of Certification” and “Maintenance of Certification” when referring to Conditions and Maintenance  \pagenum{7429}\ifhmode\expandafter\xspace\fi of Certification requirements established under the Cures Act.


          \subsubsection{Information Blocking}

          The Cures Act requires that a health IT developer, as a Condition and Maintenance of Certification under the Program, not take any action that constitutes information blocking as defined in section 3022(a) of the Public Health Service Act (PHSA). We propose to establish this information blocking Condition of Certification in \textsection{} 170.401. The Condition of Certification would prohibit any health IT developer under the Program from taking any action that constitutes information blocking as defined by section 3022(a) of the PHSA and proposed in \textsection{} 171.103.


          \subsubsection{Assurances}

          Section 3001(c)(5)(D)(ii) of the Cures Act requires that a health IT developer, as a Condition of Certification under the Program, provide assurances to the Secretary that, unless for legitimate purposes specified by the Secretary, the developer will not take any action that constitutes information blocking as defined in section 3022(a) of the PHSA, or any other action that may inhibit the appropriate exchange, access, and use of EHI. We propose to implement this provision through several Conditions of Certification and accompanying Maintenance requirements, which are set forth in proposed \textsection{} 170.402. We also propose to establish more specific Conditions and Maintenance of Certification requirements to provide assurances that a health IT developer does not take any other action that may inhibit the appropriate exchange, access, and use of EHI. These proposed requirements serve to provide further clarity under the Program as to how health IT developers can provide such broad assurances with more specific actions.


          \subsubsection{Communications}

          As a Condition and Maintenance of Certification under the Program, the Cures Act requires that health IT developers do not prohibit or restrict communications about certain aspects of the performance of health IT and the developers' related business practices. We propose that developers will be permitted to impose certain kinds of limited prohibitions and restrictions that we believe strike a reasonable balance between the need to promote open communication about health IT and related developer business practices and the need to protect the legitimate interests of health IT developers and other entities. However, certain narrowly-defined types of communications—such as communications required by law, made to a government agency, or made to a defined category of safety organization—would receive “unqualified protection,” meaning that developers would be absolutely prohibited from imposing any prohibitions or restrictions on such protected communications.


          We propose that to maintain compliance with this Condition of Certification, a health IT developer must not impose or enforce any contractual requirement or legal right that contravenes this Condition of Certification. Furthermore, we propose that if a health IT developer has contracts/agreements in existence that contravene this condition, the developer must notify all affected customers or other persons or entities that the prohibition or restriction will not be enforced by the health IT developer. Going forward, health IT developers would be required to amend their contracts/agreements to remove or make void the provisions that contravene this Condition of Certification within a reasonable period of time, but not later than two years from the effective date of a subsequent final rule for this proposed rule.


          \subsubsection{Application Programming Interfaces (APIs)}

          The Cures Act's API Condition of Certification includes several key phrases (including, for example, “without special effort”) and requirements for health IT developers that indicate the Cures Act's focus on the technical requirements as well as the actions and practices of health IT developers in implementing the certified API. In section VII.B.4 of the preamble, we outline our proposals to implement the Cures Act's API Condition of Certification. These proposals include new standards, new implementation specifications, a new certification criterion, as well as detailed Conditions and Maintenance of Certification requirements.


          \subsubsection{Real World Testing}

          The Cures Act adds a new Condition and Maintenance of Certification requirement that health IT developers successfully test the real world use of the technology for interoperability in the type of setting in which such technology would be marketed. In this proposed rule, we outline what successful “real world testing” means for the purpose of this Condition of Certification, as well as proposed Maintenance requirements—including standards updates for widespread and continued interoperability.


          We propose to limit the applicability of this Condition of Certification to health IT developers with Health IT Modules certified to one or more 2015 Edition certification criteria focused on interoperability and data exchange specified in section VII.B.5. We propose Maintenance of Certification requirements that would require health IT developers to submit publicly available annual real world testing plans as well as annual real world testing results for certified health IT products focused on interoperability. We also propose a Maintenance of Certification flexibility we have named the Standards Version Advancement Process, under which health IT developers with health IT certified to the criteria specified for interoperability and data exchange would have the option to update their health IT to a more advanced version(s) of the standard(s) or implementation specification(s) included in the criteria once such versions are approved by the National Coordinator through the Standards Version Advancement Process for use in health IT certified under the Program. Similarly, we propose that health IT developers presenting new health IT for certification to one of the criteria specified in Section VII.B.5 would have the option to certify to a National Coordinator-approved more advanced version of the adopted standards or implementation specifications included in the criteria. We propose that health IT developers voluntarily opting to avail themselves of the Standards Version Advancement Process must address their planned and actual timelines for implementation and rollout of standards updates in their annual real world testing plans and real world testing results submissions. We also propose that health IT developers of products with existing certifications who plan to avail themselves of the Standards Version Advancement Process flexibility notify both their ONC-ACB and their affected customers of their intention and plans to update their certified health IT and its anticipated impact on their existing certified health IT and customers, specifically including but not limited to whether, and if so for how long, the health IT developer intends to continue to support the certificate for the health IT certified to the prior version of the standard.



          We propose a new PoPC for ONC-ACBs that would require ONC-ACBs to review and confirm that applicable health IT developers submit real world testing plans and real world results in accordance with our proposals. Once  \pagenum{7430}\ifhmode\expandafter\xspace\fi completeness is confirmed, ONC-ACBs would upload the plans and results via hyperlinks to the Certified Health IT Product List (CHPL). We propose to revise the PoPC in \textsection{} 170.523(m) to require ONC-ACBs to collect, no less than quarterly, all updates successfully made to standards in certified health IT pursuant to the developers having voluntarily opted to avail themselves of the Standards Version Advancement Process flexibility under the real world testing Condition of Certification. We propose in \textsection{} 170.523(t), a new PoPC for ONC-ACBs requiring them to ensure that developers seeking to take advantage of the Standards Version Advancement Process flexibility in \textsection{} 170.405(b)(5) comply with the applicable requirements.


          \subsubsection{Attestations}


          The Cures Act requires that a health IT developer, as a Condition and Maintenance of Certification under the Program, provide to the Secretary an attestation to all the Conditions of Certification specified in the Cures Act, except for the “EHR reporting criteria submission” Condition of Certification. We propose to implement the Cures Act “attestations” Condition of Certification in \textsection{} 170.406. Health IT developers would attest twice a year to compliance with the Conditions and Maintenance of Certification requirements (except for the EHR reporting criteria requirement, which would be metrics reporting requirements separately implemented through a future rulemaking). The 6-month attestation period we propose in \textsection{} 170.406(b)(2) would properly balance the need to support appropriate enforcement with the attestation burden placed on health IT developers. In this regard, the proposed rule includes provisions to make the process as simple and efficient for health IT developers as possible (\emph{e.g.,} 14-day grace period, web-based form submissions, and attestation alert reminders).


          We propose that attestations would be submitted to ONC-ACBs on behalf of ONC and the Secretary. We propose a new PoPC in \textsection{} 170.523(q) that an ONC-ACB must review and submit the health IT developers' attestations to ONC. ONC would then make the attestations publicly available through the CHPL.


          \subsubsection{EHR Reporting Criteria Submission}

          The Cures Act specifies that health IT developers be required, as a Condition and Maintenance of Certification under the Program, to submit reporting criteria on certified health IT in accordance with the EHR reporting program established under section 3009A of the PHSA, as added by the Cures Act. We have not yet established an EHR reporting program. Once ONC establishes such program, we will undertake rulemaking to propose and implement the associated Condition and Maintenance of Certification requirement(s) for health IT developers.


          \subsubsection{Enforcement}

          Section 4002 of the Cures Act adds Program requirements aimed at addressing health IT developer actions and business practices through the Conditions and Maintenance of Certification requirements, which expands the current focus of the Program requirements beyond the certified health IT itself. Equally important, section 4002 also provides that the Secretary of HHS may encourage compliance with the Conditions and Maintenance of Certification requirements and take action to discourage noncompliance. We, therefore, propose a general enforcement approach to encourage consistent compliance with the requirements. The proposed rule outlines a corrective action process for ONC to review potential or known instances where a Condition or Maintenance of Certification requirement has not been or is not being met by a health IT developer under the Program. We propose, with minor modifications, to utilize the processes previously established for ONC direct review of certified health IT and codified in \textsection{}\textsection{} 170.580 and 170.581 for the enforcement of the Conditions and Maintenance of Certification requirements. Where noncompliance is identified, our first priority would be to work with the health IT developer to remedy the matter through a corrective action process. However, we propose that, under certain circumstances, ONC may ban a health IT developer from the Program or terminate the certification of one or more of its Health IT Modules.


          \subsubsection{6. Information Blocking}

          Section 4004 of the Cures Act added section 3022 of the PHSA (42 U.S.C. 300jj-52, “the information blocking provision”), which defines conduct by health care providers, and health IT developers of certified health IT, exchanges, and networks that constitutes information blocking. Section 3022(a)(1) of the PHSA defines information blocking in broad terms, while section 3022(a)(3) authorizes and charges the Secretary to identify reasonable and necessary activities that do not constitute information blocking (section 3022(a)(3) of the PHSA).


          We identify several reasonable and necessary activities as exceptions to the information blocking definition, each of which we propose would not constitute information blocking for purposes of section 3022(a)(1) of the PHSA. The exceptions would extend to certain activities that interfere with the access, exchange, or use of EHI but that may be reasonable and necessary if certain conditions are met.



          In developing the proposed exceptions, we were guided by three overarching policy considerations. First, the exceptions would be limited to certain activities that clearly advance the aims of the information blocking provision; promoting public confidence in health IT infrastructure by supporting the privacy and security of EHI, and protecting patient safety; and promoting competition and innovation in health IT and its use to provide health care services to consumers. Second, each exception is intended to address a significant risk that regulated individuals and entities (\emph{i.e.,} health care providers, health IT developers of certified health IT, health information networks, and health information exchanges) will not engage in these reasonable and necessary activities because of potential uncertainty regarding whether they would be considered information blocking. Third, and last, each exception is intended to be tailored, through appropriate conditions, so that it is limited to the reasonable and necessary activities that it is designed to exempt.



          The seven proposed exceptions are set forth in section VIII.D below. The first three exceptions, set forth in VIII.D.1-D.3 address activities that are reasonable and necessary to promote public confidence in the use of health IT and the exchange of EHI. These exceptions are intended to protect patient safety; promote the privacy of EHI; and promote the security of EHI. The next three exceptions, set forth in VIII.D.4-D.6, address activities that are reasonable and necessary to promote competition and consumer welfare. These exceptions would allow for the recovery of costs reasonably incurred; excuse an actor from responding to requests that are infeasible; and permit the licensing of interoperability elements on reasonable and non- discriminatory terms. The last exception, set forth in VIII.D.7, addresses activities that are reasonable and necessary to promote the performance of health IT. This proposed exception recognizes that actors may make health IT temporarily unavailable for maintenance or improvements that  \pagenum{7431}\ifhmode\expandafter\xspace\fi benefit the overall performance and usability of health IT.


          To qualify for any of these exceptions, we propose that an individual or entity would, for each relevant practice and at all relevant times, have to satisfy all of the applicable conditions of the exception. Additionally, we propose (in section VIII.C of this preamble) to define or interpret terms that are present in section 3022 of the PHSA (such as the types of individuals and entities covered by the information blocking provision). We also propose certain new terms and definitions that are necessary to implement the information blocking provisions. We propose to codify the proposed exceptions and other information blocking proposals in a new part of title 45 of the Code of Federal Regulations, part 171.


          \subsection{C. Costs and Benefits}

          Executive Orders 12866 on Regulatory Planning and Review (September 30, 1993) and 13563 on Improving Regulation and Regulatory Review (February 2, 2011) direct agencies to assess all costs and benefits of available regulatory alternatives and, if regulation is necessary, to select regulatory approaches that maximize net benefits (including potential economic, environmental, public health and safety effects, distributive impacts, and equity). A regulatory impact analysis (RIA) must be prepared for major rules with economically significant effects (\$100 million or more in any one year). OMB has determined that this proposed rule is an economically significant rule as the potential costs associated with this proposed rule could be greater than \$100 million per year. Accordingly, we have prepared an RIA that to the best of our ability presents the costs and benefits of this proposed rule.



          We have estimated the potential monetary costs and benefits of this proposed rule for health IT developers, health care providers, patients, ONC-ACBs, ONC-ATLs, and the federal government (\emph{i.e.,} ONC), and have broken those costs and benefits out into the following categories: (1) Deregulatory actions (no associated costs); (2) updates to the updates to the 2015 Edition health IT certification criteria; (3) Conditions and Maintenance of Certification for a health IT developer; (4) oversight for the Conditions and Maintenance of Certification; and (5) information blocking.


          We note that we have rounded all estimates to the nearest dollar and all estimates are expressed in 2016 dollars as it is the most recent data available to address all cost and benefit estimates consistently. We also note that we did not have adequate data to quantify some of the costs and benefits within this RIA. In those situations, we have described the qualitative costs and benefits of our proposals; however, such qualitative costs and benefits have not been accounted for in the monetary cost and benefit totals below.


          We estimate that the total annual cost for this proposed rule for the first year after it is finalized (including one-time costs), based on the cost estimates outlined above and throughout this RIA, would, on average, range from \$365 million to \$919 million with an average annual cost of \$642 million. We estimate that the total perpetual cost for this proposed rule (starting in year two), based on the cost estimates outlined above, would, on average, range from \$228 million to \$452 million with an average annual cost of \$340 million.


          We estimate the total annual benefit for this proposed rule would range from \$3.08 billion to \$9.15 billion with an average annual benefit of \$6.1 billion.


          We estimate the total annual net benefit for this proposed rule for the first year after it is finalized (including one-time costs), based on the cost and benefit estimates outlined above, would range from \$2.7 billion to \$8.2 billion with an average net benefit of \$5.5 billion. We estimate the total perpetual annual net benefit for this proposed rule (starting in year two), based on the cost-benefit estimates outlined above, would range from \$2.9 billion to \$8.7 billion with an average net benefit of \$5.8 billion.


          \section{II. Background}

          \subsection{A. Statutory Basis}

          The Health Information Technology for Economic and Clinical Health (HITECH) Act, Title XIII of Division A and Title IV of Division B of the American Recovery and Reinvestment Act of 2009 (the Recovery Act) (Pub. L. 111-5), was enacted on February 17, 2009. The HITECH Act amended the Public Health Service Act (PHSA) and created “Title XXX—Health Information Technology and Quality” (Title XXX) to improve health care quality, safety, and efficiency through the promotion of health IT and electronic health information (EHI) exchange.


          The Cures Act was enacted on December 13, 2016, to accelerate the discovery, development, and delivery of 21st century cures, and for other purposes. The Cures Act, through Title IV—Delivery, amended the HITECH Act (Title XIII of Division A of Pub. L. 111-5) by modifying or adding certain provisions to the PHSA relating to health IT.


          \subsubsection{1. Standards, Implementation Specifications, and Certification Criteria}

          The HITECH Act established two new federal advisory committees, the HIT Policy Committee (HITPC) and the HIT Standards Committee (HITSC). Each was responsible for advising the National Coordinator for Health Information Technology (National Coordinator) on different aspects of standards, implementation specifications, and certification criteria.



          Section 3002 of the Cures Act amended the PHSA by replacing the HITPC and HITSC with one committee, the Health Information Technology Advisory Committee (HIT Advisory Committee or HITAC). Section 3002(a) establishes that the HITAC shall advise and recommend to the National Coordinator on different aspects of standards, implementation specifications, and certification criteria, relating to the implementation of a health IT infrastructure, nationally and locally, that advances the electronic access, exchange, and use of health information. Further described in section 3002(b)(1)(A) of the PHSA, this includes providing to the National Coordinator recommendations on a policy framework to advance interoperable health IT infrastructure, updating recommendations to the policy framework, and making new recommendations, as appropriate. Section 3002(b)(2)(A) identifies that in general, the HITAC shall recommend to the National Coordinator for purposes of adoption under section 3004, standards, implementation specifications, and certification criteria and an order of priority for the development, harmonization, and recognition of such standards, specifications, and certification criteria. Like the process previously required of the former HITPC and HITSC, the HITAC will develop a schedule for the assessment of policy recommendations for the Secretary to publish in the \textbf{Federal Register}.



          Section 3004 of the PHSA identifies a process for the adoption of health IT standards, implementation specifications, and certification criteria and authorizes the Secretary to adopt such standards, implementation specifications, and certification criteria. As specified in section 3004(a)(1), the Secretary is required, in consultation with representatives of other relevant federal agencies, to jointly review standards, implementation specifications, and certification criteria endorsed by the National Coordinator under section 3001(c) and subsequently  \pagenum{7432}\ifhmode\expandafter\xspace\fi determine whether to propose the adoption of any grouping of such standards, implementation specifications, or certification criteria. The Secretary is required to publish all determinations in the \textbf{Federal Register}.


          Section 3004(b)(3) of the PHSA titled, Subsequent Standards Activity, provides that the Secretary shall adopt additional standards, implementation specifications, and certification criteria as necessary and consistent with the schedule published by the HITAC. We consider this provision in the broader context of the HITECH Act and Cures Act to grant the Secretary the authority and discretion to adopt standards, implementation specifications, and certification criteria that have been recommended by the HITAC and endorsed by the National Coordinator, as well as other appropriate and necessary health IT standards, implementation specifications, and certification criteria.


          \subsubsection{2. Health IT Certification Program(s)}


          Under the HITECH Act, section 3001(c)(5) of the PHSA provides the National Coordinator with the authority to establish a certification program or programs for the voluntary certification of health IT. Specifically, section 3001(c)(5)(A) specifies that the National Coordinator, in consultation with the Director of the National Institute of Standards and Technology (NIST), shall keep or recognize a program or programs for the voluntary certification of health IT that is in compliance with applicable certification criteria adopted under this subtitle (\emph{i.e.,} certification criteria adopted by the Secretary under section 3004 of the PHSA). The certification program(s) must also include, as appropriate, testing of the technology in accordance with section 13201(b) of the HITECH Act. Overall, section 13201(b) of the HITECH Act requires that with respect to the development of standards and implementation specifications, the Director of NIST shall support the establishment of a conformance testing infrastructure, including the development of technical test beds. The HITECH Act also indicates that the development of this conformance testing infrastructure may include a program to accredit independent, non-federal laboratories to perform testing.


          Section 3001(c)(5) of the PHSA was amended by the Cures Act, which instructs the National Coordinator to encourage, keep, or recognize, through existing authorities, the voluntary certification of health IT under the Program for use in medical specialties and sites of service for which no such technology is available or where more technological advancement or integration is needed. Section 3001(c)(5)(C)(iii) identifies that the Secretary, in consultation with relevant stakeholders, shall make recommendations for the voluntary certification of health IT for use by pediatric health providers to support the care of children, as well as adopt certification criteria under section 3004 to support the voluntary certification of health IT for use by pediatric health providers. The Cures Act further amended section 3001(c)(5) of the PHSA by adding section 3001(c)(5)(D), which provides the Secretary with the authority, through notice and comment rulemaking, to require conditions and maintenance of certification requirements for the Program.


          \subsection{B. Regulatory History}

          The Secretary issued an interim final rule with request for comments (75 FR 2014, Jan. 13, 2010), which adopted an initial set of standards, implementation specifications, and certification criteria. On March 10, 2010, ONC published a proposed rule (75 FR 11328) that proposed both a temporary and permanent certification program for the purposes of testing and certifying health IT. A final rule establishing the temporary certification program was published on June 24, 2010 (75 FR 36158) and a final rule establishing the permanent certification program was published on January 7, 2011 (76 FR 1262). ONC issued multiple rulemakings since these initial rulemaking to update standards, implementation specifications, and certification criteria and the certification program, a history of which can be found in the final rule titled, “2015 Edition Health Information (Health IT) Certification Criteria, 2015 Edition Base Electronic Health Record (EHR) Definition, and ONC Health IT Certification Program Modifications” (Oct. 16, 2015, 80 FR 62602) (“2015 Edition final rule”). A correction notice was published for the 2015 Edition final rule on December 11, 2015 (80 FR 76868) to correct preamble and regulatory text errors and clarify requirements of the Common Clinical Data Set (CCDS), the 2015 Edition privacy and security certification framework, and the mandatory disclosures for health IT developers.


          The 2015 Edition final rule established a new edition of certification criteria (“2015 Edition health IT certification criteria” or “2015 Edition”) and a new 2015 Edition Base EHR definition. The 2015 Edition established the capabilities and specified the related standards and implementation specifications that CEHRT would need to include to, at a minimum, support the achievement of “meaningful use” by eligible clinicians, eligible hospitals, and critical access hospitals under the Medicare and Medicaid EHR Incentive Programs (EHR Incentive Programs) (now referred to as the Promoting Interoperability Programs) \textsuperscript{3}

             when the 2015 Edition is required for use under these and other programs referencing the CEHRT definition. The 2015 Edition final rule also made changes to the Program. The final rule adopted a proposal to change the Program's name to the “ONC Health IT Certification Program” from the ONC \emph{HIT} Certification Program, modified the Program to make it more accessible to other types of health IT beyond EHR technology and for health IT that supports care and practice settings beyond the ambulatory and inpatient settings, and adopted new and revised Principles of Proper Conduct (PoPC) for ONC-ACBs.


          \insfootnote{\textsuperscript{3} \emph{\url{https://www.federalregister.gov/d/2018-16766/p-4.}}}


          After issuing a proposed rule on March 2, 2016 (81 FR 11056), ONC published a final rule titled, “ONC Health IT Certification Program: Enhanced Oversight and Accountability” (81 FR 72404) (“EOA final rule”) on October 19, 2016. The final rule finalized modifications and new requirements under the Program, including provisions related to ONC's role in the Program. The final rule created a regulatory framework for ONC's direct review of health IT certified under the Program, including, when necessary, requiring the correction of non-conformities found in health IT certified under the Program and suspending and terminating certifications issued to Complete EHRs and Health IT Modules. The final rule also sets forth processes for ONC to authorize and oversee accredited testing laboratories under the Program. In addition, it includes provisions for expanded public availability of certified health IT surveillance results.


          \section{III. Deregulatory Actions for Previous Rulemakings}

[Omitted]

          \section{IV. Updates to the 2015 Edition Certification Criteria}

          This rule proposes to update the 2015 Edition by revising and adding certification criteria that would establish the capabilities and related standards and implementation specifications for the certification of health IT. The updates to the 2015 Edition would enhance interoperability and improve the accessibility of patient records consistent with section 4006(a) of the Cures Act.


          \subsection{A. Standards and Implementation Specifications}

          \subsubsection{1. National Technology Transfer and Advancement Act}


          The National Technology Transfer and Advancement Act (NTTAA) of 1995 (15 U.S.C. 3701 \emph{et seq.}) and the Office of Management and Budget (OMB) Circular A-119 \textsuperscript{14}
             require the use of, wherever practical, technical standards that are developed or adopted by voluntary consensus standards bodies to carry out policy objectives or activities, with certain exceptions. The NTTAA and OMB Circular A-119 provide exceptions to electing only standards developed or adopted by voluntary consensus standards bodies, namely when doing so would be inconsistent with applicable law or otherwise impractical. Agencies have the discretion to decline the use of existing voluntary consensus standards if determined that such standards are inconsistent with applicable law or otherwise impractical, and instead use a government-unique standard or other standard. In addition to the consideration of voluntary consensus standards, the OMB Circular A-119 recognizes the contributions of standardization activities that take place outside of the voluntary consensus standards process. Therefore, in instances where use of voluntary consensus standards would be inconsistent with applicable law or otherwise impracticable, other standards should be considered that meet the agency's regulatory, procurement or program needs, deliver favorable technical and economic outcomes, and are widely utilized in the marketplace. In this proposed rule, we use voluntary consensus standards except for:


          \insfootnote{\textsuperscript{14} \emph{\url{https://www.whitehouse.gov/sites/whitehouse.gov/files/omb/circulars/A119/revised_circular_a-119_as_of_1_22.pdf.}}}


          
          

            • The standard we propose to adopt in \textsection{} 170.213. We propose to remove the Common Clinical Data Set (CCDS) definition and effectively replace it with a government  \pagenum{7440}\ifhmode\expandafter\xspace\fi unique standard, the United States Core Data for Interoperability (USCDI), Version 1(v1);


            • The standard we propose to adopt in \textsection{} 170.215(a)(2). We propose the government unique API Resource Collection in Health (ARCH) Version 1 implementation specification;



            • The standards we propose to adopt in \textsection{} 170.215(a)(3) through (5) for application programming interfaces (APIs). These market driven consortia standards have been developed through a streamlined process that does not meet the full definition of voluntary consensus standards development but still includes representation from those interested in the use cases supported by the standards (\emph{e.g.,} health IT developers and health care providers). In the absence of available voluntary consensus standards that would meet our needs, these standards deliver favorable technical and economic outcomes, particularly improved interoperability. Further, some of these standards may eventually proceed through a standards development organization for approval; and


            • The standards we propose to adopt in \textsection{} 170.205(h)(3) and (k)(3). We propose to replace the current HL7 QRDA standards with government unique standards that more effectively support the associated certification criterion's use case, which is reporting eCQM data to CMS.


          
          \subsubsection{2. Compliance With Adopted Standards and Implementation Specifications}


          In accordance with Office of the Federal Register regulations related to “incorporation by reference,” 1 CFR part 51, which we follow when we adopt proposed standards and/or implementation specifications in any subsequent final rule, the entire standard or implementation specification document is deemed published in the \textbf{Federal Register} when incorporated by reference therein with the approval of the Director of the Federal Register. Once published, compliance with the standard and implementation specification includes the entire document unless we specify otherwise. For example, if we adopted the Argonaut Data Query Implementation Guide (IG) proposed in this proposed rule (\emph{see} section VII.B.4.b), health IT certified to certification criteria referencing this IG would need to demonstrate compliance with all mandatory elements and requirements of the IG. If an element of the IG is optional or permissive in any way, it would remain that way for testing and certification \emph{unless} we specified otherwise in regulation. In such cases, the regulatory text would preempt the permissiveness of the IG.


          \subsubsection{3. “Reasonably Available” to Interested Parties}


          The Office of the Federal Register has established requirements for materials (\emph{e.g.,} standards and implementation specifications) that agencies propose to incorporate by reference in the Code of Federal Regulations (79 FR 66267; 1 CFR 51.5(a)). To comply with these requirements, in section XI (“Incorporation by Reference”) of this preamble, we provide summaries of, and uniform resource locators (URLs) to, the standards and implementation specifications we propose to adopt and subsequently incorporate by reference in the Code of Federal Regulations. To note, we also provide relevant information about these standards and implementation specifications throughout the relevant sections of the proposed rule.


          \subsection{B. Revised and New 2015 Edition Criteria}

[Omitted.]

          \subsection{C. Unchanged 2015 Edition Criteria—Program Reference Alignment}

[Omitted.]

          \section{V. Modifications to the ONC Health IT Certification Program}

[Omitted.]

          \section{VI. Health IT for the Care Continuum}

[Omitted.]

          \section{VII. Conditions and Maintenance of Certification}

          Section 4002 of the Cures Act requires the Secretary of HHS, through notice and comment rulemaking, to establish Conditions and Maintenance of Certification requirements for the Program. Specifically, health IT developers or entities must adhere to certain Conditions and Maintenance of Certification requirements concerning information blocking; appropriate exchange, access, and use of electronic health information; communications regarding health IT; application programming interfaces (APIs); real world testing for interoperability; attestations regarding certain Conditions and Maintenance of Certification requirements; and submission of reporting criteria under the EHR reporting program.


          \subsection{A. Implementation}

          To implement Section 4002 of the Cures Act, we propose an approach whereby the Conditions and Maintenance of Certification express both initial requirements for health IT developers and their certified Health IT Module(s) as well as ongoing requirements that must be met by both health IT developers and their certified Health IT Module(s) under the Program. If these requirements are not met, then the health IT developer may no longer be able to participate in the Program and/or its certified health IT may have its certification terminated. We propose to implement each Cures Act Condition of Certification with further specificity as it applies to the Program. We also propose to establish the Maintenance of Certification requirements for each Condition of Certification as standalone requirements. This approach would establish clear baseline technical and behavior Conditions of Certification requirements with evidence that the Conditions of Certification are continually being met through the Maintenance of Certification requirements.


          \subsection{B. Provisions}

          \subsubsection{1. Information Blocking}

          The Cures Act requires that a health IT developer, as a Condition and Maintenance of Certification under the Program, not take any action that constitutes “information blocking” as defined in section 3022(a) of the PHSA (see 3001(c)(5)(D)(i) of the PHSA). We propose to establish this information blocking Condition of Certification in \textsection{} 170.401. The Condition of Certification prohibits any health IT developer under the Program from taking any action that constitutes information blocking as defined by section 3022(a) of the PHSA and proposed in \textsection{} 171.103. 


          We clarify that this proposed “information blocking” Condition of Certification and its requirements would be substantive requirements of the Program and would use the definition of “information blocking” established by section 3022(a) of the PHSA and as also proposed in \textsection{} 171.103, as it relates to health IT developers of certified health IT. In addition to ONC's statutory authority for this Condition of Certification, the HHS Office of the Inspector General (OIG) has both investigatory and enforcement authority over information blocking and may issue civil money penalties for information blocking conducted by health IT developers of certified health IT, health information networks and health information exchanges. OIG may also investigate health care providers for information blocking for which health care providers could be subject to disincentives.


          We refer readers to section VII.D of this proposed rule for additional discussion of ONC's enforcement of this and other proposed Conditions and Maintenance of Certification requirements. We also refer readers to section VIII of this proposed rule for our proposals to implement the information blocking provisions of the Cures Act, including proposed \textsection{} 171.103.


          We do not, at this time, propose any associated Maintenance of Certification requirements for this Condition of Certification.


          \subsubsection{2. Assurances}

          The Cures Act requires that a health IT developer, as a Condition and Maintenance of Certification under the Program, provide assurances to the Secretary, unless for legitimate purposes specified by the Secretary, that it will not take any action that constitutes information blocking as defined in section 3022(a) of the PHSA, or any other action that may inhibit the appropriate exchange, access, and use of electronic health information (EHI). We propose to implement this Condition of Certification and accompanying Maintenance of Certification requirements in \textsection{} 170.402. As a Condition of Certification requirement, a health IT developer must comply with the Condition as recited here and in the Cures Act. We refer readers to section VIII of this proposed rule for the proposed reasonable and necessary activities specified by the Secretary, which constitute the exceptions to the information blocking definition.


          We also propose to establish more specific Conditions and Maintenance of Certification requirements for a health IT developer to provide assurances that it does not take any action that may inhibit the appropriate exchange, access, and use of EHI. These proposed requirements serve to provide further clarity under the Program as to how health IT developers can provide such broad assurances with more specific actions.


          \subsubsection{a. Full Compliance and Unrestricted Implementation of Certification Criteria Capabilities}


          We propose, as a Condition of Certification, that a health IT developer must ensure that its health IT certified under the ONC Health IT Certification Program (Program) conforms to the full scope of the certification criteria to which its health IT is certified. This has always been an expectation of ONC and users of certified health IT and, importantly, a requirement of the Program. We believe, however, that by incorporating this expectation and requirement as a Condition of Certification under the Program, there would be assurances, and documentation via the “Attestations” Condition and Maintenance of Certification requirements proposed in \textsection{} 170.406, that all health IT developers fully understand their responsibilities under the Program, including not to take any action with their certified health IT that may inhibit the appropriate exchange, access, and use of EHI. To this point, certification criteria are designed and issued so that certified health IT can support interoperability and the appropriate exchange, access, and use of electronic health information. \pagenum{7466}\ifhmode\expandafter\xspace\fi 
          


          We propose that, as a complementary Condition of Certification, health IT developers of certified health IT must provide an assurance that they have made certified capabilities available in ways that enable them to be implemented and used in production environments for their intended purposes. More specifically, developers would be prohibited from taking any action that could interfere with a user's ability to access or use certified capabilities for any purpose within the scope of the technology's certification. Such actions may inhibit the appropriate access, exchange, or use of EHI and are therefore contrary to this proposed Condition of Certification and the statutory provision that it implements. While such actions are already prohibited under the Program (80 FR 62711), making these existing requirements explicit would ensure that health IT developers are required to attest to them on a regular basis pursuant to the Condition of Certification proposed in \textsection{} 170.406, which will in turn provide additional assurances to the Secretary that developers of certified health IT support and do not inhibit appropriate access, exchange, or use of EHI.



          By way of example, actions that would violate this aspect of the proposed Condition include failing to fully deploy or enable certified capabilities; imposing limitations (including restrictions) on the use of certified capabilities once deployed; or requiring subsequent developer assistance to enable the use of certified capabilities, contrary to the intended uses and outcomes of those capabilities (\emph{see} 80 FR 62711). The Condition would also be violated were a developer to refuse to provide documentation, support, or other assistance reasonably necessary to enable the use of certified capabilities for their intended purposes (\emph{see} 80 FR 62711). More generally, any action that would be likely to substantially impair the ability of one or more users (or prospective users) to implement or use certified capabilities for any purpose within the scope of applicable certification criteria would be prohibited by this Condition (\emph{see} 80 FR 62711). Such actions may include imposing limitations or additional types of costs, especially if these were not disclosed when a customer purchased or licensed the certified health IT (\emph{see} 80 FR 62711).


          \subsubsection{b. Certification to the “Electronic Health Information Export” Criterion}

          We propose, as a Condition of Certification requirement, that a health IT developer that produces and electronically manages EHI must certify health IT to the 2015 Edition “electronic health information export” certification criterion in \textsection{} 170.315(b)(10). We discuss the proposed “electronic health information (EHI) export” criterion in section IV.B.4 of this proposed rule. Further, as a Maintenance of Certification requirement, we propose that a health IT developer that produces and electronically manages EHI must provide all of its customers of certified health IT with health IT certified to the functionality included in \textsection{} 170.315(b)(10) within 24 months of a subsequent final rule's effective date or within 12 months of certification for a health IT developer that never previously certified health IT to the 2015 Edition, whichever is longer. Consistent with these proposals, we also propose to amend \textsection{} 170.550 to require that ONC-ACBs certify health IT to the proposed 2015 Edition “EHI export” when the health IT developer of the health IT presented for certification produces and electronically manages EHI.


          As discussed in section IV.C.1 of this proposed rule, the availability of the capabilities in the proposed 2015 Edition “EHI export” certification criterion to providers and patients would promote access, exchange, and use of EHI to facilitate health care providers in switching practices and health IT systems and patients' electronic access to all their health information stored by a provider. As such, health IT developers with health IT certified to the proposed 2015 Edition “EHI export” certification criterion that is made available to its customers provides assurances that the developer is not taking actions that constitute information blocking or any other action that may inhibit the appropriate exchange, access, and use of EHI.


          \subsubsection{c. Records and Information Retention}

[Omitted.]

          \subsubsection{d. Trusted Exchange Framework and the Common Agreement—Request for Information}


          The Cures Act added section 3001(c)(9) to the PHSA, which requires the National Coordinator to work with stakeholders with the goal of developing or supporting a Trusted Exchange Framework and a Common Agreement (collectively, “TEFCA”) for the purpose of ensuring full network-to-network exchange of health information. Section 3001(c)(9)(B) outlines a process for establishing a TEFCA between health information networks (HINs)—including provisions for the National Coordinator, in collaboration with the NIST, to provide technical assistance on implementation and pilot testing of the TEFCA. In accordance with section 3001(c)(9)(C), the National Coordinator shall publish the TEFCA on its website and in the \textbf{Federal Register}, as well as annually publish on its website a directory of the HINs that have adopted the Common Agreement and are capable of trusted exchange pursuant to the Common Agreement. The process, application, and construction of the  \pagenum{7467}\ifhmode\expandafter\xspace\fi TEFCA are further outlined in section 3001(c)(9)(D), including requiring that the Secretary shall through notice and comment rulemaking, establish a process for HINs that voluntarily adopt the TEFCA to attest to such adoption. We request comment as to whether certain health IT developers should be required to participate in the TEFCA as a means of providing assurances to their customers and ONC that they are not taking actions that constitute information blocking or any other action that may inhibit the appropriate exchange, access, and use of EHI. We would expect that such a requirement, if proposed in a subsequent rulemaking, would apply to health IT developers that have a Health IT Module(s) certified to any of the certification criteria in \textsection{}\textsection{} 170.315(b)(1), (c)(1) and (c)(2), (e)(1), (f), and (g)(9) through (11); and provide services for connection to health information networks (HINs). These services could be routing EHI through a HIN or responding to requests for EHI from a HIN.



          We have identified health IT developers that certify health IT to the criteria above because the capabilities included in the criteria support access and exchange of EHI. Therefore, we believe such health IT developers, as opposed to a health IT developer that only supports clinical decision support (\textsection{} 170.315(a)(9)) with its certified health IT, would be best suited to participate in the Trusted Exchange Framework and adhere to the Common Agreement. Similarly, we believe that many such health IT developers with the identified certified health IT would be in position, and requested by customers, to provide connection services to HINs. When such criteria are met (certified to the identified criteria above and actually providing connection services), participation in the Trusted Exchange Framework and adherence to the Common Agreement are consistent with this Condition and Maintenance of Certification as specified by the Cures Act, the intent of Congress to establish widespread interoperability and exchange of health information without information blocking, and supports ONC's responsibility, as established by the HITECH Act, to develop and support a nationwide health IT infrastructure that allows for the electronic use and exchange of information. More specifically, by participating in the Trusted Exchange Framework and adhering to the Common Agreement, these health IT developers provide assurances that they are not taking actions that constitute information blocking or any other action that may inhibit the appropriate exchange, access, and use of EHI. For more information on the Trusted Exchange Framework and Common Agreement, please visit: \emph{\url{https://www.healthit.gov/topic/interoperability/trusted-exchange-framework-and-common-agreement.}}
          


          In consideration of this request for comment, we welcome comment on the certification criteria we have identified as the basis for health IT developer participation in the Trusted Exchange Framework and adherence to the Common Agreement, other certification criteria that would serve as a basis for health IT developer participation in the Trusted Exchange Framework and adherence to the Common Agreement, and whether the current structure of the Trusted Exchange Framework and Common Agreement are conducive to health IT developer participation and in what manner.


          \subsubsection{3. Communications}

          The Cures Act requires that a health IT developer, as a Condition and Maintenance of Certification under the Program, does not prohibit or restrict communication regarding the following subjects:


          • The usability of the health information technology;


          • The interoperability of the health information technology;


          • The security of the health information technology;


          • Relevant information regarding users' experiences when using the health information technology;


          • The business practices of developers of health information technology related to exchanging electronic health information; and


          • The manner in which a user of the health information technology has used such technology.


          We propose to implement this Condition of Certification and its requirements in \textsection{} 170.403. The Cures Act placed no limitations on the protection of the communications delineated above (referred to hereafter as “protected communications”). As such, we propose to broadly interpret the subject matter of communications that are protected from developer prohibition or restriction as well as the conduct of developers that implicate the protection afforded to communications by this Condition of Certification and discuss this proposed approach in detail below. While we propose to implement a broad general prohibition against developers imposing prohibitions and restrictions on protected communications, we also recognize that there are circumstances where it is both legitimate and reasonable for developers to limit the sharing of information about their products. As such, we propose to allow developers to impose prohibitions or restrictions on protected communications in certain narrowly defined circumstances. In order for a prohibition or restriction on a protected communication to be permitted, we propose that it must pass a two-part test. First, the communication that is being prohibited or restricted must not fall within a class of communication about which no restriction or prohibition would ever be legitimate or reasonable—such as communications required by law, made to a government agency, or made to a defined category of safety organizations—and which we refer to hereafter as “communications with unqualified protection.” Second, to be permitted, a developer's prohibition or restriction must also fall within a prescribed category of circumstances for which we propose it is both legitimate and reasonable for a developer to limit the sharing of information about its products. This would be because of the nature of the relationship between the developer and the communicator or because of the nature of the information that is, or could be, the subject of the communication (referred to hereafter as “permitted prohibitions and restrictions”). A restriction or prohibition that does not satisfy this two-part test will contravene this Condition of Certification. As discussed in more detail below, we propose that this two-part test strikes a reasonable balance between the need to promote open communication about health IT and related business practices, and the need to protect the legitimate interests of health IT developers and other entities.


          \subsubsection{a. Background and Purpose}

          This Condition of Certification addresses industry practices that severely limit the ability and willingness of health IT customers, users, researchers, and other stakeholders who use and work with health IT to openly discuss and share their experiences and other relevant information about the performance of health IT, including the ability of health IT to exchange health information electronically. These practices result in a lack of transparency around health IT that can contribute to and exacerbate patient safety risks, system security vulnerabilities, and product performance issues. As discussed below, these issues have been documented and reported on over a number of years.



          The challenges presented by health IT developer actions that prohibit or  \pagenum{7468}\ifhmode\expandafter\xspace\fi restrict communications have been examined for some time. The problem was identified in a 2012 report by the Institute of Medicine of the National Academies (IOM) entitled “Health IT and Patient Safety: Building Safer Systems for Better Care” \textsuperscript{57}
             (IOM Report). The IOM Report stated that health care providers, researchers, consumer groups other health IT users lack information regarding the functionality of health IT.\textsuperscript{58}
             The IOM Report observed, relatedly, that many developers restrict the information that users can communicate about developers' products through nondisclosure clauses, confidentiality clauses, intellectual property protections, hold-harmless clauses, and other boilerplate contract language.\textsuperscript{59}
             Importantly, the IOM Report found that such clauses discourage users from sharing information about patient safety risks related to health IT, which significantly limits the ability of health IT users to understand how health IT impacts patient safety.\textsuperscript{60}
             The report stressed the need for health IT developers to enable the free exchange of information regarding the experience of using their health IT products, including the sharing of screenshots.\textsuperscript{61}
            
          


          \insfootnote{\textsuperscript{57} IOM (Institute of Medicine), \emph{Health IT and Patient Safety: Building Safer Systems for Better Care} (2012). Available at \emph{\url{http://www.nationalacademies.org/hmd/Reports/2011/Health-IT-and-Patient-Safety-Building-Safer-Systems-for-Better-Care.aspx.}}}


          \insfootnote{\textsuperscript{58}
              \emph{Id,} 195.}


          \insfootnote{\textsuperscript{59}
              \emph{Ibid.}}


          \insfootnote{\textsuperscript{60}
              \emph{Ibid.}}


          \insfootnote{\textsuperscript{61}
              \emph{Ibid.}}


          Other close observers of health IT have similarly noted that broad restrictions on communications can inhibit the communication of information about errors and adverse events.\textsuperscript{62}
             Concerns have also been raised by researchers of health IT products,\textsuperscript{63}
             who emphasize that confidentiality and intellectual property provisions in contracts often place broad and unclear limits on authorized uses of information related to health IT, which in turn seriously impacts the ability of researchers to conduct and publish their research.\textsuperscript{64}
            
          


          \insfootnote{\textsuperscript{62} \emph{See} Kathy Kenyon, \emph{Overcoming Contractual Barriers to EHR Research,} Health Affairs Blog (October 14, 2015). Available at \emph{\url{http://healthaffairs.org/blog/2015/10/14/overcoming-contractual-barriers-to-ehr-research/.}}}


          \insfootnote{\textsuperscript{63} \emph{See} Hardeep Singh, David C. Classen, and Dean F. Sittig, \emph{Creating an Oversight Infrastructure} for \emph{Electronic Health Record-Related Patient Safety Hazards,} 7(4) Journal of Patient Safety 169 (2011). Available at \emph{\url{https://www.ncbi.nlm.nih.gov/pmc/articles/PMC3677059/.}}}


          \insfootnote{\textsuperscript{64} Kathy Kenyon, \emph{Overcoming Contractual Barriers to EHR Research,} Health Affairs Blog (October 14, 2015). Available at \emph{\url{http://healthaffairs.org/blog/2015/10/14/overcoming-contractual-barriers-to-ehr-research/.}}}


          The issue of health IT developers prohibiting or restricting communications about health IT has been the subject of a series of hearings by the Senate Committee on Health, Education, Labor and Pensions (HELP Committee), starting in the spring of 2015. During several hearings, stakeholders emphasized the lack of transparency around the performance of health IT in a live environment, noting that this can undermine a competitive marketplace, hinder innovation, and prevent improvements in the safety and usability of the technology.\textsuperscript{65} \textsuperscript{66}
             Additionally, the HELP Committee indicated serious concerns regarding the reported efforts of health IT developers to restrict, by contract and other means, communications regarding user experience, including information relevant to safety and interoperability.\textsuperscript{67}
             When one Senator asked a panel of experts—which included a health IT developer—if there were any reasons for health IT contracts to have confidentiality clauses restricting users of health information technology from discussing their experience of using the health IT, all panel members agreed that such clauses should be prohibited.\textsuperscript{68}
            
          


          \insfootnote{\textsuperscript{65} HELP 6/10/15 pg 12; Available at \emph{\url{https://www.gpo.gov/fdsys/pkg/CHRG-114shrg25971/pdf/CHRG-114shrg25971.pdf.}}
            


            
              \textsuperscript{66} HELP 3/17/15 pg 47; Available at \emph{\url{https://www.gpo.gov/fdsys/pkg/CHRG-114shrg93864/pdf/CHRG-114shrg93864.pdf.}}}


          \insfootnote{\textsuperscript{67} HELP 7/23/15 pg 13, pg 27; Available at \emph{\url{https://www.help.senate.gov/hearings/achieving-the-promise-of-health-information-technology-information-blocking-and-potential-solutions.}}}


          \insfootnote{\textsuperscript{68} HELP 7/23/15 pg 38; Available at \emph{\url{https://www.help.senate.gov/hearings/achieving-the-promise-of-health-information-technology-information-blocking-and-potential-solutions.}}}


          Prior to the HELP Committee hearings described above, the issue of developers prohibiting and restricting communications about the performance of their health IT was also addressed in House Energy and Commerce Committee hearings when committee members heard testimony and held discussions related to the Cures Act.\textsuperscript{69}
             Commentary by witnesses at the hearings emphasized the need to ensure that health IT products are safe and encouraged the availability of information around health IT products to improve quality and ensure patient safety.


          \insfootnote{\textsuperscript{69} Energy and Commerce 7/17/14 pg 35; Available at \emph{\url{http://docs.house.gov/meetings/IF/IF16/20140717/102509/HHRG-113-IF16-20140717-SD008.pdf.}}}


          Developer actions that prohibit or restrict communications about health IT have also been the subject of investigative reporting.\textsuperscript{70}
             A September 2015 report examined eleven contracts between health systems and major health IT developers and found that, with one exception, all of the contracts protected large amounts of information from being disclosed, including information related to safety and performance issues.\textsuperscript{71}
             The report stated that broad confidentiality and intellectual property protection clauses were the greatest barriers to allowing the communication of information regarding potential safety issues and adverse events.\textsuperscript{72}
            
          


          \insfootnote{\textsuperscript{70} D Tahir, \emph{POLITICO Investigation: EHR gag clauses exist—and, critics say, threaten safety,} Politico, August 27, 2015.}


          \insfootnote{\textsuperscript{71} \emph{Ibid.}}


          \insfootnote{\textsuperscript{72} \emph{Ibid.}}


          Finally, ONC has itself been made aware of health IT developer contract language that purports to prohibit the disclosure of information about health IT, including even a customer's or user's opinions and conclusions about the performance and other aspects of the technology. Our extensive interactions with health care providers, researchers, and other stakeholders consistently indicate that such terms are not uncommon and that some developers may actively enforce them and engage in other practices to discourage communications regarding developers' health IT products and related business practices.


          This proposed Condition of Certification is needed to significantly improve transparency around the functioning of health IT in the field. This will help ensure that the health IT ultimately selected and used by health care providers and others functions as expected, is less likely to have safety issues or implementation difficulties, enables greater interoperability of health information, and more fully allows users to reap the benefits of health IT utilization, including improvements in care and quality, and reductions in costs.


          \subsubsection{b. Condition of Certification Requirements}

          \subsubsection{i. Protected Communications and Communicators}


[Omitted.]

          \subsubsection{ii. Protected Subject Areas}

[Omitted. This broadly covers communications about usability, interoperability,
security, functionality, and uses of Health IT.]


          \subsubsection{iii. Meaning of “Prohibit or Restrict”}

[Omitted. Interestingly, using a TPM under 1201 is considered a prohibition, as
is sending a takedown notice under 512.]


          \subsubsection{iv. Communications With Unqualified Protection}

[Omitted. These generally relate to communications based on legal obligations,
protecting safety, or whistleblowing.]


          \subsubsection{v. Permitted Prohibitions and Restrictions}

          We propose that, \emph{except for communications with unqualified protection discussed above and enumerated in \textsection{} 170.403(a)(2)(i),} health IT developers would be permitted to impose certain narrow kinds of prohibitions and restrictions discussed below and specified in \textsection{} 170.403(a)(2)(ii). We believe this policy strikes a reasonable balance between the need to promote open communication about health IT and related business practices and the need to protect the legitimate interests of health IT developers and other entities. Specifically, with the exception of communications with unqualified protection, developers would be permitted to prohibit or restrict the following communications, subject to certain conditions:


          • Communications of their own employees;


          • Disclosure of non-user-facing aspects of the software;


          • Certain communications that would infringe the developer's or another person's intellectual property rights;


          • Publication of screenshots in very narrow circumstances; and


          • Communications of information that a person or entity knows only because of their participation in developer-led product development and testing.


          As discussed in detail in the sections that follow, the proposed Condition of Certification carefully delineates the circumstances under which these types of prohibitions and restrictions would be permitted, including certain associated conditions that developers would be required to meet. To be clear, any prohibition or restriction not expressly permitted would violate the Condition. Additionally, it would be the developer's burden to demonstrate to the satisfaction of ONC that the developer has met all associated requirements. Further, as an additional safeguard, we propose that where a developer seeks to avail itself of one of the permitted types of prohibitions or restrictions, the developer must ensure that potential communicators are clearly and explicitly notified about the information and material that can be communicated, and that which cannot. We propose this would mean that the language of health IT contracts must be precise and specific. Contractual provisions or public statements that support a permitted prohibition or restriction on communication should be very specific about the rights and obligations of the potential communicator. Contract terms that are vague and cannot be readily understood by a reasonable health IT customer will not benefit from the qualifications to this Condition of Certification outlined below.


          \subsubsection{(A) Developer Employees and Contractors}


          [Omitted.]


          \subsubsection{(B) Non-User-Facing Aspects of Health IT}

          [Omitted.]


          \subsubsection{(C) Intellectual Property}

          Many aspects of health IT—including software and documentation—will contain intellectual property that belongs to the health IT developer (or a third party) and is protected by law. Health IT products may have portions in which copyrighted works exist, or that are subject to patent protection. As in other technology sectors, health IT developers place a high value on their intellectual property and go to significant lengths to protect it, including intellectual property provisions in their health IT contracts.



          This Condition of Certification is not intended to operate as a \emph{de facto} license for health IT users and others to act in any way that might infringe the legitimate intellectual property rights of developers. Indeed, we propose that health IT developers are permitted to prohibit or restrict communications that would infringe their intellectual property rights so long as the communication in question is not a communication with unqualified protection. However, any prohibition and restriction imposed by a developer must be no broader than legally permissible and reasonably necessary to protect the developer's legitimate intellectual property interests. We are aware that some health IT contracts contain broad intellectual property provisions (and related terms, such as nondisclosure provisions) that purport to prevent health IT customers and users from using copyright material in ways that are lawful. On this basis, while we are providing an exception for the protection of intellectual property interests, we want to clarify that under this Condition of Certification health IT developers are not permitted to prohibit or restrict, or purport to prohibit or restrict, communications that would be a “fair use” of any copyright work comprised in the developer's health IT. That is, a developer is not permitted to prohibit or restrict communications under the guise of copyright protection (or under the guise of a confidentiality or non-disclosure obligation) when the communication in question makes a use of the copyright material in a way that would qualify that use as a “fair use.” \textsuperscript{80}
            
          


          \insfootnote{\textsuperscript{80} \emph{See} 17 U.S.C. 107.}



          We welcome comments on whether an appropriate balance has been struck between protecting legitimate intellectual property rights of developers and ensuring that health IT customers, users, researchers, and other stakeholders who use and work with health IT can openly discuss and share their experiences and other relevant  \pagenum{7475}\ifhmode\expandafter\xspace\fi information about the performance of health IT.


          \subsubsection{(D) Faithful Reproductions of Health IT Screenshots}

          We propose that health IT developers generally would not be permitted to prohibit or restrict communications that disclose screenshots of the developer's health IT. We consider screen displays an essential component of health IT performance and usability, and their reproduction may be necessary in order for a health IT user or other health IT stakeholder to properly make communications about the subject matters enumerated in \textsection{} 170.403(a)(1). We acknowledge that some health IT developers have historically and aggressively sought to prohibit the disclosure of such communications. We consider that developers may benefit from screen displays being faithfully reproduced so that health IT users and other stakeholders can form an objective opinion on any question raised about usability in communications protected by this proposed Condition of Certification. Moreover, we consider that the reproduction of screenshots in connection with the making of a communication protected by this Condition of Certification would ordinarily represent a “fair use” of any copyright subsisting in the screen display, and developers should not impose prohibitions or restrictions that would limit that fair use.


          Notwithstanding the above, we propose to permit certain prohibitions and restrictions on the communication of screenshots. Except in connection with communications with unqualified protection, developers would be permitted to impose certain restrictions on the disclosure of screenshots, as described below.


          In order to ensure that disclosures of screenshots are reasonable and represent a faithful reproduction of the developer's screen design and health IT, we propose that developers would be permitted to prevent communicators from altering screenshots, other than to annotate the screenshot or to resize it for the purpose of publication. We consider this a reasonable limitation on the disclosure of screenshots and one that would help developers' health IT avoid being misrepresented by communicators seeking to make a communication protected by this proposed Condition of Certification.


          We also propose that health IT developers could impose restrictions on the disclosure of a screenshot on the basis that it would infringe third-party intellectual property rights (on their behalf or as required by license). However, to take advantage of this exception, the developer would need to first put all potential communicators on sufficient written notice of those parts of the screen display that contain trade secrets or intellectual property rights and cannot be communicated, and would still need to allow communicators to communicate redacted versions of screenshots that do not reproduce those parts.


          Finally, we also recognize that health IT developers may have obligations under HIPAA as a business associate and that it would be reasonable for developers to impose restrictions on the communication of screenshots that contain protected health information, provided that developers permit the communication of screenshots that have been redacted to conceal protected health information, or the relevant individual's consent or authorization had been obtained.


          \subsubsection{(E) Testing and Development}

          [Omitted.]


          \subsubsection{c. Maintenance of Certification Requirements}

[Omitted.]


          \subsubsection{4. Application Programming Interfaces}

          As a Condition of Certification (and Maintenance thereof) under the Program, the Cures Act requires health IT developers to publish APIs that allow “health information from such technology to be accessed, exchanged, and used without special effort through the use of APIs or successor technology or standards, as provided for under applicable law.” The Cures Act's API Condition of Certification also states that a developer must, through an API, “provide access to all data elements of a patient's electronic health record to the extent permissible under applicable privacy laws.”


          The Cures Act's API Condition of Certification includes several key phrases and requirements for health IT developers that go beyond just the technical functionality of the products they present for certification. In this section of the preamble we outline our proposals to implement the Cures Act's API Condition of Certification in order to provide compliance clarity for health IT developers.


          These proposals include new standards, new implementation specifications, and a new certification criterion as well as detailed Conditions and Maintenance of Certification requirements. We also propose to modify the Base EHR definition. We note that health IT developers should also consider these proposals in the context of what could warrant review from an information blocking perspective in so far as action (or inaction) that would be inconsistent with this proposed rule's API Conditions and Maintenance of Certification requirements.


          \subsubsection{a. Statutory Interpretation and API Policy Principles}


          One of the most significant phrases in the Cures Act's API Condition of Certification concerns the deployment and use of APIs “without special effort.” Specifically, the Cures Act requires health IT developers to publish APIs and allow health information from such technology “to be accessed, exchanged, and used without special effort.” In this context, we interpret the “effort” exerted (\emph{i.e.,} by whom) to be focused on the API users, which could include third-party software developers, the health care providers that acquired this API technology, and patients, health care providers, and payers that use apps/services that connect to API technology.



          As we considered the meaning and context associated with the phrase “without special effort” and what  \pagenum{7477}\ifhmode\expandafter\xspace\fi would make APIs included in certified health IT truly “open,” we focused on key attributes that could be used to refine our interpretation and guide our proposals. We interpret “without special effort” to require that APIs, and the health care ecosystem in which they are deployed, have three attributes: \emph{Standardized, transparent,} and \emph{pro-competitive.} Each of these attributes is briefly described in more detail below and all of our subsequent proposals address one or a combination of these attributes.


          • \emph{Standardized}—meaning that all health IT developers seeking certification would have to implement the same technical API capabilities in their products (using modern, computing standards such as RESTful interfaces and XML/JSON). Technical consistency and implementation predictability are fundamental to scale API-enabled interoperability and reduce the level of custom development and costs necessary to access, exchange, and use health information. Further, from a regulatory standpoint, health IT developers would gain certainty in regards to pre-certification testing requirements and post-certification “real world testing” expectations. Equally, from an industry standpoint, a consistent and predictable set of API functions would provide the health IT ecosystem with known technical requirements against which “app” developers and other innovative services can be built.


          • \emph{Transparent}—meaning that all health IT developers seeking certification would need to make the specific business and technical documentation necessary to interact with the APIs in production freely and publicly accessible. Such transparency and openness is commonplace in many other industries and has fueled innovation, growth, and competition.


          • \emph{Pro-competitive}—meaning that all health IT developers seeking certification would need to abide by business practices that promote the efficient access, exchange, and use of EHI to support a competitive marketplace that enhances consumer value and choice. Moreover, health care providers should have the sole authority and autonomy to unilaterally permit third-party software developers to connect to the API technology they have acquired. In other words, health IT developers must not interfere with a health care provider's use of their acquired API technology in any way, especially ways that would impact its equitable access and use based on (for example) another software developer's size, current client base, or business line. It also means that developers (together with health care providers that deploy APIs) are accountable to patients who, as consumers of health care services, have paid for their care and the information generated from such care. Thus, patients should be able to access their EHI via any API-enabled app they choose without special effort, including without incurring additional costs and without encountering access requirements that impede their ability to access their information in a persistent manner.


          \subsubsection{b. Key Terms}

          To clearly convey the stakeholders on which our proposals focus and are meant to support, we propose to use the following terms to reflect these meanings and/or roles:


          • The term “\emph{API technology”} (with a lowercase “t”) generally refers to the capabilities of certified health IT that fulfill the API-focused certification criteria adopted or proposed for adoption at 45 CFR 170.315(g)(7) through (g)(11).


          • “\emph{API Technology Supplier”} refers to a health IT developer that creates the API technology that is presented for testing and certification to any of the certification criteria adopted or proposed for adoption at 45 CFR 170.315(g)(7) through (g)(11). We propose to adopt this term in \textsection{} 170.102.


          • “\emph{API Data Provider”} refers to the organization that deploys the API technology created by the “API Technology Supplier” and provides access via the API technology to data it produces and electronically manages. In some cases, the API Data Provider may contract with the API Technology Supplier to perform the API deployment service on its behalf. However, in such circumstances, the API Data Provider retains control of what and how information is disclosed and so for the purposes of this definition is considered to be the entity that deploys the API technology. We propose to adopt this term in \textsection{} 170.102.


          • “\emph{API User”}—refers to persons and entities that use or create software applications that interact with the APIs developed by the “API Technology Supplier” and deployed by the “API Data Provider.” An API User includes, but is not limited to, third-party software developers, developers of software applications used by API Data Providers, patients, health care providers, and payers that use apps/services that connect to API technology. We propose to adopt this term in \textsection{} 170.102.


          We also use:


          • The term “\emph{(g)(10)-certified API”} for ease of reference throughout the preamble to refer to health IT certified to the certification criterion proposed for adoption in 45 CFR 170.315(g)(10).


          • The term “\emph{app”} for ease of reference to describe any type of software application that would be designed to interact with the (g)(10)-certified APIs. This generic term is meant to include, but not be limited to, a range of applications from mobile and browser-based to comprehensive business-to-business enterprise applications administered by third parties.


          \subsubsection{c. Proposed API Standards, Implementation Specifications, and Certification Criterion}

          APIs can be thought of as a set of commands, functions, protocols, and/or tools published by one software developer (“software developer A”) that enable other software developers (X, Y, and Z) to create programs and applications that interact with A's software without needing to know the “internal” workings of A's software. APIs can facilitate more seamless access to health information and it is important to note for context that ONC adopted three 2015 Edition certification criteria that specified API capabilities for Health IT Modules (criteria adopted in 45 CFR 170.315(g)(7), (g)(8), and (g)(9)). The following sections detail our proposals to adopt standards, implementation specifications, and a new API certification criterion. Together, these proposals account for the technical requirements we propose to associate with the Cures Act's API Condition of Certification and are reinforced through the condition's policy proposals.


          \subsubsection{i. Proposed Adoption of FHIR DSTU2 Standard}

[Omitted. ONC proposes to adopt the FHIR Draft Standard for Trial Use (DSTU) 2.]

          \subsubsection{ii. Proposed Adoption of Associated FHIR Release 2 Implementation Specifications}

[Omitted. Several specific FHIR standards are proposed to be adopted.]

          \subsubsection{iii. Proposed Adoption of Standards and Implementation Specifications To Support Persistent User Authentication and App Authorization}

[Omitted.]


          \subsubsection{iv. Proposed Adoption of a New API Certification Criterion in \textsection{} 170.315(g)(10)}

[Omitted.]

          \subsubsection{d. Condition of Certification Requirements}

          To implement the Cures Act, we have designed this API Condition of Certification in a manner that will complement the technical capabilities described in our other proposals, while addressing the broader technology and business landscape in which these API capabilities will be deployed and used.



          Consistent with the attributes we have identified for the statutory phrase “without special effort,” our overarching vision for this Condition of Certification is to ensure that (g)(10)- certified APIs, among all API  \pagenum{7485}\ifhmode\expandafter\xspace\fi technology, are deployed in a manner that supports an experience that is as seamless and frictionless as possible. To that end, we seek to promote a standards-based ecosystem that is transparent, scalable, and open to robust competition and innovation.


          The specific requirements of this Condition of Certification are discussed in several sections below. These requirements would address certain implementation, maintenance, and business practices for which clear and consistent parameters are needed to ensure that API technology is deployed in a manner that achieves the policy goals we have described. The proposed requirements would also align this Condition of Certification with other requirements and policies of the Cures Act that promote interoperability and deter information blocking, as discussed in more detail in the sections that follow.


          \subsubsection{i. Scope and Compliance}

          To start this Condition of Certification, we propose in \textsection{} 170.404 to apply this Condition of Certification to health IT developers with health IT certified to any of the API-focused certification criteria. These criteria include the proposed “standardized API for patient and population services” (\textsection{} 170.315(g)(10)) and “consent management for APIs” (\textsection{} 170.315(g)(11)) as well as the current “application access—patient selection” (\textsection{} 170.315(g)(7)), “application access—data category request” (\textsection{} 170.315(g)(8)), “application access—all data request” (\textsection{} 170.315(g)(9)). In other words, this entire Condition of Certification would not apply to health IT developers that do not have technology certified to any of these API-focused certification criteria. Similarly, this condition is solely applicable to these API-focused certification criteria. As a result, the proposed policies for this Condition of Certification would not apply to a health IT developer's practices associated with, for example, the immunization reporting certification criterion adopted in \textsection{} 170.315(f)(1) because that criterion is not one of the API-focused criteria. However, health IT developers should remain mindful that other proposals in this proposed rule, especially those related to information blocking, could still apply to its practices associated with non-API-focused certification criteria.


          Given the proposed applicability of this condition to current API-focused criteria and that health IT developers with products certified to \textsection{}\textsection{} 170.315(g)(7)-(9) would need to meet new compliance requirements associated with such criteria, we also propose certain compliance timelines associated with this Condition of Certification that would need to be met.


          \subsubsection{ii. Cures Act Condition and Interpretation of Access to “All Data Elements”}

[Omitted.]

          \subsubsection{iii. Transparency Conditions}

          We propose as part of this Condition of Certification that API Technology Suppliers be required to make specific business and technical documentation freely and publicly accessible. Thus, we propose to adopt several transparency conditions as part of \textsection{} 170.404(a)(2).


          Similar to our policy associated with the API-focused certification criteria, we propose in \textsection{} 170.404(a)(2)(i) that all published documentation be complete and available via a publicly accessible hyperlink that allows any person to directly access the information without any preconditions or additional steps. For example, the API Technology Supplier cannot impose any access requirements, including, without limitation, any form of registration, account creation, “click-through” agreements, or requirement to provide contact details or other information prior to accessing the documentation.


          \subsubsection{Terms and Conditions Transparency}

[Omitted.]

          \subsubsection{iv. Permitted Fees Conditions}

[Omitted. Limitations on fees for API access.]

          \subsubsection{iv. Openness and Pro-Competitive Conditions}


          We propose that API Technology Suppliers would have to comply with certain requirements to promote an open and competitive marketplace. As a general condition, we propose in \textsection{} 170.404(a)(4) that API Technology Suppliers must grant API Data Providers (\emph{i.e.,} health care providers who purchase or license API technology) the sole authority and autonomy to permit API Users to interact with the API technology deployed by the API Data Provider. We reinforce this general condition through more specific proposed conditions proposals discussed below that would require API Technology Suppliers to provide equitable access to API technology, which would include granting the rights and providing the cooperation necessary to enable apps to be deployed that use the API technology to access, exchange, and use EHI in production environments.


          As important context for these proposals, we note that the API technology required by this Condition of Certification falls squarely within the concept of “essential interoperability elements” described in section VIII.C.4.b of this preamble and, as such, are subject to strict protections under the information blocking provision. As a corollary, to the extent that API Technology Suppliers claim an intellectual property right or other proprietary interest in the API technology, they must take care not to impose any fees, require any license terms, or engage in any other practices that could add unnecessary cost, difficulty, or other burden that could impede the effective use of the API technology for the purpose of enabling or facilitating access, exchange, or use of EHI. Moreover, even apart from these information blocking considerations, we believe that, as developers of technology certified under the Program, API Technology Suppliers owe a special responsibility to patients, providers, and other stakeholders to make API technology available in a manner that is truly “open” and minimizes any costs or other burdens that could result in special effort. The proposed conditions set forth below are intended to provide clear rules and expectations for API Technology Suppliers so that they can meet these obligations.


          \subsubsection{Non-Discrimination}


          We propose in \textsection{} 170.404(a)(4)(i) that an API Technology Supplier must adhere to a strictly non-discriminatory policy regarding the provision of API technology. As a starting point, we propose to require in \textsection{} 170.404(a)(4)(i)(A) that API Technology Suppliers comply with all of the requirements discussed in section VIII.C.4.b of this proposed rule regarding the non-discriminatory provision of interoperability elements. Accordingly, and consistent with developers' obligations under the Program and our expectation that API technology be truly “open,” we propose to require that API Technology Suppliers must provide API technology to API Data Providers on terms that are no less favorable than they would provide to themselves and their customers, suppliers, partners, and other persons with whom they have a business relationship. This requirement would apply to both price and non-price terms and thus would apply to any fees that the API Technology Supplier is permitted to charge under the “permitted charges conditions” of this Condition of Certification. We believe this requirement would ensure that API Data Providers (\emph{i.e.,} health care providers) who purchase or license API technology have sole authority and autonomy to permit third-party software developers to connect to and use the API technology they have acquired.


          Next, we propose in \textsection{} 170.404(a)(4)(i)(B) that any terms and conditions associated with API technology would have to be based on objective and verifiable criteria that are uniformly applied for all substantially similar or similarly situated classes of persons and requests. For example, if the API Technology Supplier applied an “app store” entry/listing process unequally and added arbitrary criteria based on the use case(s) an app was focused on, such business practices would not comply with this specific condition and could also be in violation of the information blocking provision.



          Moreover, we propose in \textsection{} 170.404(a)(4)(i)(C) that an API Technology Supplier would be prohibited from offering or varying such terms or conditions on the basis of impermissible criteria, such as whether the API User with whom the API Data Provider has a relationship is a competitor, potential competitor, or will be using EHI obtained via the API technology in a way that facilitates competition with the API Technology Supplier. The API Technology Supplier would also be prohibited from taking into consideration the revenue or other value the API User with whom the API Data Provider has a relationship may derive from access, exchange, or use of EHI obtained by means of the API technology. We believe these proposals  \pagenum{7493}\ifhmode\expandafter\xspace\fi will help promote greater equity and competition in market as well as prevent discriminatory business practices by API Technology Suppliers.


          \subsubsection{Rights To Access and Use API Technology}

          We propose in \textsection{} 170.404(a)(4)(ii)(A) that an API Technology Supplier would have to make API technology available in a manner that enables API Data Providers and API Users to develop and deploy apps to access, exchange, and use EHI in production environments. To this end, we propose that an API Technology Supplier must have and, upon request, must grant to API Data Providers and their API Users all rights that may be reasonably necessary to access and use API technology in a production environment. In other words, this proposal is focused on the provision of rights reasonably necessary to access and use API technology and does not extend to other intellectual property maintained by the API Technology Supplier, especially intellectual property that has no nexus with the access and use of API technology. In situations where such a nexus exists, even partially, the API Technology Supplier would have the duty to determine a method to grant the applicable rights reasonably necessary to access and use the API technology. And if practicable, under these partial cases, we note that it would be possible for the API Technology Supplier to exclude the intellectual property that would have no impact on the access and use of the API technology.


          Accordingly, following our proposal, API Technology Suppliers would need to grant API Data Providers and their API Users with rights that could include but not be limited to the following in order to sufficiently support the use of the API technology:


          • For the purposes of developing products or services that are designed to be interoperable with the API Technology Supplier's health IT or with health IT under the API Technology Supplier's control.


          • Any marketing, offering, and distribution of interoperable products and services to potential customers and users that would be needed for the API technology to be used in a production environment. Note, API Technology Suppliers, pursuant to the “value-added services” permitted fee, would be able to offer and charge for services such as preferential marketing agreements, co-marketing agreements, and other business arrangements so long as such services are beyond what is necessary for the API technology to be put into use in a production environment.


          • Enabling the use of the interoperable products or services in production environments, including accessing and enabling the exchange and use of electronic health information.


          Relatedly, in \textsection{} 170.404(a)(4)(ii)(B) we propose to prohibit an API Technology Supplier from imposing any collateral terms or agreements that could interfere with or lead to special effort in the use of API technology for any of the above purposes. We note that these collateral terms or agreements may also implicate the information blocking provision for the reasons described in section VIII.D.3.c of this preamble. These specific proposed conditions would expressly prohibit an API Technology Supplier from conditioning any of the rights described above on the requirement that the recipient of the rights do, or agree to do, any of the following:


          • Pay a fee to license the rights described above, including but not limited to a license fee, royalty, or revenue-sharing arrangement.


          • Not compete with the API Technology Supplier in any product, service, or market.


          • Deal exclusively with the API Technology Supplier in any product, service, or market.


          • Obtain additional licenses, products, or services that are not related to or can be unbundled from the API technology.


          • License, grant, assign, or transfer any intellectual property to the API Technology Supplier.


          • Meet additional developer or product certification requirements.


          • Provide the API Technology Supplier or its technology with reciprocal access to application data.



          These prohibitions largely mirror those proposed under the exception to the information blocking definition in \textsection{} 171.206 and reflect the same concerns expressed in that context in section VIII.D.3.c of this preamble. However, we note the following important distinction: Whereas proposed \textsection{} 171.206 would permit a developer to charge a reasonable royalty to license interoperability elements, this API Condition of Certification would not permit any such royalty, license fee, or other type of fee of any kind whatsoever pursuant to the general fee prohibition proposed in the “permitted charges condition.” This additional limitation reflects the more exacting standards that apply to API Technology Suppliers with respect to the provision of API technology under this Condition of Certification. While we believe that, for the reasons described in section VIII.D.3.c of this preamble, health IT developers should generally be permitted to charge reasonable royalties for the use of their intellectual property, we consider API technology to be a special case. Certified health IT developers (\emph{i.e.,} API Technology Suppliers) are required to provide these capabilities as part of their statutory duty to facilitate the access, exchange, and use of patient health information from EHRs “without special effort.” We believe the language requiring that these capabilities be “open” precludes an API Technology Supplier from conditioning access to API technology on the payment of a royalty or other fee, however “reasonable” the fee might otherwise be.


          We clarify that the prohibitions explained above against additional developer or Health IT Module certification requirements and, separately, against requirements for reciprocal access to application data, are within the scope of the collateral terms prohibited by proposed \textsection{} 171.206 even though these additional API Technology Supplier requirements are not explicitly referenced by that exception because they are not generally applicable to all types of interoperability elements. Nevertheless, permitting an API Technology Supplier to impose these kinds of additional requirements would be inconsistent with the Cures Act's expectation that API technology be made available openly and in a manner that promotes competition. For the same reason such practices may raise information blocking concerns.


          \subsubsection{API Technology Suppliers—Additional Obligations}


          To support the use of API technology in production environments, we propose in \textsection{} 170.404(a)(4)(iii) that an API Technology Supplier must provide all support and other services that are reasonably necessary to enable the effective development, deployment, and use of API technology by API Data Providers and its API Users in production environments. In general, the precise nature of these obligations will depend on the specifics of the API Technology Supplier's technology and the manner in which it is implemented and made available for specific customers. Therefore, with the following exceptions, we do not delineate the API Technology Supplier's specific support obligations and instead propose a general requirement to this effect in \textsection{} 170.404(a)(4)(iii). \pagenum{7494}\ifhmode\expandafter\xspace\fi 
          


          \subsubsection{Changes and Updates to API Technology and Terms and Conditions}

          We propose to require in \textsection{} 170.404(a)(4)(iii)(A) that API Technology Suppliers must make reasonable efforts to maintain the compatibility of the API technology they develop and assist API Data Providers to deploy in order to avoid disrupting the use of API technology. Similarly, we propose in \textsection{} 170.404(a)(4)(iii)(B) that prior to making changes or updates to its API technology or to the terms or conditions thereof, an API Technology Supplier would need to provide notice and a reasonable opportunity for its API Data Provider customers and registered application developers to update their applications to preserve compatibility with its API technology or to comply with any revised terms or conditions. Without this opportunity, clinical and patient applications could be rendered inoperable or operate in unexpected ways unbeknownst to the users or software developers.


          Further, we note that this proposal aligns with the exception to the information blocking definition proposed in \textsection{} 171.206. As explained in section VIII.D.3.c of this preamble, the information blocking definition would be implicated were an API Technology Supplier to make changes to its API technology that “break” compatibility or otherwise degrade the performance or interoperability of the licensee's products or services that incorporate the licensed API technology. We propose these additional safeguards are important in light of the ease with which an API Technology Supplier could make subtle “tweaks” to its technology or related services, which could disrupt the use of the licensee's compatible technologies or services and result in substantial competitive and consumer injury.


          We clarify that this requirement would in no way prevent an API Technology Supplier from making improvements to its technology or responding to the needs of its own customers or users. However, the API Technology Supplier would need to demonstrate that whatever actions it took were necessary to accomplish these purposes and that it afforded the licensee a reasonable opportunity under the circumstances to update its technology to maintain interoperability. Relatedly, we recognize that an API Technology Supplier may have to suspend access or make other changes immediately and without prior notice in response to legitimate privacy, security, or patient safety-related exigencies. Such practices would be permitted by this Condition of Certification provided they are tailored and do not unnecessarily interfere with the use of API technology. From an information blocking standpoint, if such practices interfered with access, exchange, or use of EHI, the API Technology Supplier could seek coverage under the exceptions to the information blocking provision described in section VIII.D of this preamble. For instance, if the suspended access was in response to a privacy exigency, the API Technology Supplier may be able to seek coverage under the exception for promoting the privacy of EHI at proposed \textsection{} 171.202.


          \subsubsection{e. Maintenance of Certification Requirements}

[Omitted.]

          \subsubsection{f. 2015 Edition Base EHR Definition}

[Omitted.]

          \subsubsection{5. Real World Testing}

[Omitted.]

          \subsubsection{6. Attestations}

[Omitted.]

          \section{VIII. Information Blocking}

          \subsection{A. Statutory Basis}

          Section 4004 of the Cures Act added section 3022 of the PHSA (42 U.S.C. 300jj-52, “the information blocking provision”). Section 3022(a)(1) of the PHSA defines practices that constitute information blocking when engaged in by a health care provider, or a health information technology developer, exchange, or network. Section 3022(a)(3) authorizes the Secretary to identify, through notice and comment rulemaking, reasonable and necessary activities that do not constitute information blocking for purposes of the definition set forth in section 3022(a)(1). We propose to establish seven exceptions to the information blocking definition, each of which would define certain activities that would not constitute information blocking for purposes of section 3022(a)(1) of the PHSA because they are reasonable and necessary to further the ultimate policy goals of the information blocking provision. We also propose to interpret or define certain statutory terms and concepts that are ambiguous, incomplete, or provide the Secretary with discretion, and that we believe are necessary to carry out the Secretary's rulemaking responsibilities under section 3022(a)(3).


          \subsection{B. Legislative Background and Policy Considerations}

          In this section, we outline the purpose of the information blocking provision and related policy and practical considerations that we considered in identifying the reasonable and necessary activities that are proposed as exceptions to the definition of information blocking described subsequently in section VIII.D of this preamble.


          \subsubsection{1. Purpose of the Information Blocking Provision}

          The information blocking provision was enacted in response to concerns that some individuals and entities are engaging in practices that unreasonably limit the availability and use of electronic health information (EHI) for authorized and permitted purposes. These practices undermine public and private sector investments in the nation's health IT infrastructure and frustrate efforts to use modern technologies to improve health care quality and efficiency, accelerate research and innovation, and provide greater value and choice to health care consumers.


          The nature and extent of information blocking has come into sharp focus in recent years. In 2015, at the request of Congress, we submitted a Report on Health Information Blocking \textsuperscript{101}
             (“Information Blocking Congressional Report”), in which we commented on the then current state of technology and of health IT and health care markets. Notably, we observed that prevailing market conditions create incentives for some individuals and entities to exercise their control over EHI in ways that limit its availability and use.


          \insfootnote{\textsuperscript{101} ONC, Report to Congress on Health Information Blocking (Apr. 2015), \emph{\url{https://www.healthit.gov/sites/default/files/reports/info_blocking_040915.pdf}} [hereinafter “Information Blocking Congressional Report”].}


          Since that time, we have continued to receive complaints and reports of information blocking from patients, clinicians, health care executives, payers, app developers and other technology companies, registries and health information exchanges, professional and trade associations, and many other stakeholders. ONC has listened to and reviewed these complaints and reports, consulted with stakeholders, and solicited input from our federal partners in order to inform our proposed information blocking policies. Stakeholders described discriminatory pricing policies that have the obvious purpose and effect of excluding competitors from the use of interoperability elements. Many of the industry stakeholders who shared their perspectives with us in listening sessions, including several health IT developers of certified health IT, condemned these practices and urged us to swiftly address them. Our engagement with stakeholders confirms that, despite significant public and private sector efforts to improve interoperability and data accessibility, adverse incentives remain and continue to undermine progress toward a more connected health system.


          Based on these economic realities and our first-hand experience working with the health IT industry and stakeholders, in the Information Blocking Congressional Report, we concluded that information blocking is a serious problem and recommended that Congress prohibit information blocking and provide penalties and enforcement mechanisms to deter these harmful practices.


          Recent empirical and economic research further underscores the intractability of this problem and its harmful effects. In a national survey of health information organizations, half of respondents reported that EHR developers routinely engage in information blocking, and a quarter of respondents reported that hospitals and health systems routinely do so. The survey reported that perceived motivations for such conduct included, for EHR vendors, maximizing short-term revenue and competing for new clients, and for hospitals and health systems, strengthening their competitive position relative to other hospitals and health systems.\textsuperscript{102}
             Other research suggests that these practices weaken competition among health care providers by limiting patient mobility, encouraging consolidation, and creating barriers to entry for developers of new and innovative applications and technologies that enable more effective uses of clinical data to improve population health and the patient experience.\textsuperscript{103}
            
          


          \insfootnote{\textsuperscript{102} \emph{See, e.g.,} Julia Adler-Milstein and Eric Pfeifer, \emph{Information Blocking: Is It Occurring And What Policy Strategies Can Address It?,} 95 Milbank Quarterly 117, 124-25 (Mar. 2017), \emph{available at \url{http://onlinelibrary.wiley.com/doi/10.1111/1468-0009.12247/full.}}}


          \insfootnote{\textsuperscript{103} \emph{See, e.g.,} Martin Gaynor, Farzad Mostashari, and Paul B. Ginsberg, \emph{Making Health Care Markets Work: Competition Policy for Health Care,} 16-17 (Apr. 2017), \emph{available at \url{http://heinz.cmu.edu/news/news-detail/index.aspx?nid=3930;}} Diego A. Martinez et al., \emph{A Strategic Gaming Model For Health Information Exchange Markets,} Health Care Mgmt. Science (Sept. 2016). (“[S]ome healthcare provider entities may be interfering with HIE across disparate and unaffiliated providers to gain market advantage.”) Niam Yaraghi, \emph{A Sustainable Business Model for Health Information Exchange Platforms: The Solution to Interoperability in Healthcare IT} (2015), \emph{available at \url{http://www.brookings.edu/research/papers/2015/01/30-sustainable-business-model-health-information-exchange-yaraghi;}} Thomas C. Tsai \& Ashish K. Jha, \emph{Hospital Consolidation, Competition, and Quality: Is Bigger Necessarily Better?,} 312 J. AM. MED. ASSOC. 29, 29 (2014).}



          The information blocking provision provides a comprehensive response to these concerns. The information blocking provision defines and creates possible penalties and disincentives for  \pagenum{7509}\ifhmode\expandafter\xspace\fi information blocking in broad terms, while working to deter the entire spectrum of practices that unnecessarily impede the flow of EHI or its use to improve health and the delivery of care. The information blocking provision applies to the conduct of health care providers, and to health IT developers of certified health IT, exchanges, and networks, and seeks to deter it with substantial penalties, including civil money penalties, and disincentives for violations. Additionally, developers of health IT certified under the Program are prohibited from information blocking under 3001(c)(5)(D)(i) of the PHSA. To promote effective enforcement, the information blocking provision empowers the HHS Office of Inspector General (OIG) to investigate claims of information blocking and provides referral processes to facilitate coordination with other relevant agencies, including ONC, the HHS Office for Civil Rights (OCR), and the Federal Trade Commission (FTC). The information blocking provision also provides for a complaint process and corresponding confidentiality protections to encourage and facilitate the reporting of information blocking. Enforcement of the information blocking provision is buttressed by section 3001(c)(5)(D)(i) and (vi) of the PHSA, which prohibits information blocking by developers of certified health IT as a Condition and Maintenance of Certification requirement under the Program and requires them to attest that they have not engaged in such practices.


          \subsubsection{2. Policy Considerations and Approach to the Information Blocking Provision}

          To ensure that individuals and entities that engage in information blocking are held accountable, the information blocking provision encompasses a relatively broad range of potential practices. For example, it is possible that some activities that are innocuous, or even beneficial, could technically implicate the information blocking provision. Given the possibility of these practices, Congress authorized the Secretary to identify reasonable and necessary activities that do not constitute information blocking (see section 3022(a)(3) of the PHSA) (in this proposed rule, we refer to such reasonable and necessary activities identified by the Secretary as “exceptions” to the information blocking provision). The information blocking provision also excludes from the definition of information blocking practices that are required by law (section 3022(a)(1) of the PHSA) and clarifies certain other practices that would not be penalized (sections 3022(a)(6) and (7) of the PHSA).


          In considering potential exceptions to the information blocking provision, we must balance a number of policy and practical considerations. To minimize compliance and other burdens for stakeholders, we seek to promote policies that are clear, predictable, and administrable. In addition, we seek to implement the information blocking provision in a way that is sensitive to legitimate practical challenges that may prevent access, exchange, or use of EHI in certain situations. We must also accommodate practices that, while they may inhibit access, exchange, or use of EHI, are reasonable and necessary to advance other compelling policy interests, such as preventing harm to patients and others, promoting the privacy and security of EHI, and promoting competition and consumer welfare.


          At the same time, while pursuing these objectives, we must adhere to Congress's plainly expressed intent to provide a comprehensive response to the information blocking problem. Information blocking can occur through a variety of business, technical, and organizational practices that can be difficult to detect and that are constantly changing as technology and industry conditions evolve. The statute responds to these challenges by defining information blocking broadly and in a manner that allows for careful consideration of relevant facts and circumstances in individual cases.


          Accordingly, we propose to establish certain defined exceptions to the information blocking provision. These exceptions would be subject to strict conditions that balance the considerations described above. Based on those considerations, in developing the proposed exceptions, we applied three overarching policy criteria. First, each exception would be limited to certain activities that are both reasonable and necessary to advance the aims of the information blocking provision. These reasonable and necessary activities include: Promoting public confidence in the health IT infrastructure by supporting the privacy and security of EHI, and protecting patient safety; and promoting competition and innovation in health IT and its use to provide health care services to consumers. Second, we believe that each exception addresses a significant risk that regulated individuals and entities will not engage in these reasonable and necessary activities because of uncertainty regarding the breadth or applicability of the information blocking provision. Third, and last, each exception is intended to be tailored, through appropriate conditions, so that it is limited to the reasonable and necessary activities that it is designed to protect and does not extend protection to other activities or practices that could raise information blocking concerns.


          We discuss these policy considerations in more detail in the context of each of the exceptions proposed in section VIII.D of this preamble.


          \subsection{C. Relevant Statutory Terms and Provisions}

          In this section of the preamble, we discuss how we propose to interpret certain aspects of the information blocking provision that we believe are ambiguous, incomplete, or that provide the Secretary with discretion. We propose to define or interpret certain terms or concepts that are present in the statute and, in a few instances, to establish new regulatory terms or definitions that we believe are necessary to implement the Secretary's authority under section 3022(a)(3) to identify reasonable and necessary activities that do not constitute information blocking. Our goal in interpreting the statute and defining relevant terms is to provide greater clarity concerning the types of practices that could implicate the information blocking provision and, relatedly, to more effectively communicate the applicability and scope of the proposed exceptions outlined in this proposed rule. We believe that these proposals will provide a more meaningful opportunity for the public to comment on the proposed exceptions and our overall approach to interpreting and administering the information blocking provision. Additionally, we believe additional interpretive clarity will assist regulated actors to comply with the requirements of the information blocking provision.


          \subsubsection{1. “Required by Law”}


          With regard to the statute's exclusion of practices that are “required by law” from the definition of information blocking, we emphasize that “required by law” refers specifically to interferences with access, exchange, or use of EHI that are explicitly required by state or federal law. By carving out practices that are “required by law,” the statute acknowledged that there are state and federal laws that advance important policy interests and objectives by restricting access, exchange, and use of their EHI, and that practices that follow such laws should not be considered information blocking. \pagenum{7510}\ifhmode\expandafter\xspace\fi 
          


          We note that for the purpose of developing an exception for reasonable and necessary privacy-protective practices, we have distinguished between interferences that are “required by law” and those engaged in pursuant to a privacy law, but which are not “required by law.” The former does not fall within the definition of information blocking, but the latter may implicate the information blocking provision and an exception may be necessary. For a detailed discussion of this topic, please see section VIII.D.2 of this preamble.


          \subsubsection{2. Health Care Providers, Health IT Developers, Exchanges, and Networks}

          Section 3022(a)(1) of the PHSA, in defining information blocking, refers to four classes of individuals and entities that may engage in information blocking and which include: Health care providers, health IT developers of certified health IT, networks, and exchanges. We propose to adopt definitions of these terms to provide clarity regarding the types of individuals and entities to whom the information blocking provision applies. We note that, for convenience and to avoid repetition in this preamble, we typically refer to these individuals and entities covered by the information blocking provision as “actors” unless it is relevant or useful to refer to the specific type of individual or entity. That is, when the term “actor” appears in this preamble, it means an individual or entity that is a health care provider, health IT developer, exchange, or network. For the same reasons, we propose to define “actor” in \textsection{} 171.102.


          \subsubsection{a. Health Care Providers}

          The term “health care provider” is defined in section 3000(3) of the PHSA. We propose to adopt this definition for purposes of section 3022 of the PHSA when defining “health care provider” in \textsection{} 171.102. We note that this definition is different from the definition of “health care provider” under the HIPAA Privacy and Security Rules. We are considering adjusting the information blocking definition of “health care provider” to cover all individuals and entities covered by the HIPAA “health care provider” definition. We seek comment on whether this approach would be justified, and commenters are encouraged to specify reasons why doing so might be necessary to ensure that the information blocking provision applies to all health care providers that might engage in information blocking.


          \subsubsection{b. Health IT Developers of Certified Health IT}


          Section 3022(a)(1)(B) of the PHSA defines information blocking, in part, by reference to the conduct of “health information technology developers.” Because title XXX of the PHSA does not define “health information technology developer,” we interpret section 3022(a)(1)(B) in light of the specific authority provided to OIG in section 3022(b)(1)(A) and (b)(2). Section 3022(b)(2) discusses developers, networks, and exchanges in terms of an “individual or entity,” specifically cross-referencing section 3022(b)(1)(A). Sections 3002(b)(1) and (b)(1)(A) state, in relevant part, that the OIG may investigate information blocking claims regarding a health information technology developer of certified health information technology or other entity offering certified health information technology. Together, these sections make clear that the information blocking provisions and OIG's authority extend to individuals \emph{or} entities that develop \emph{or} offer certified health IT. That the individual or entity must develop or offer \emph{certified} health IT is further supported by section 3022(a)(7) of the PHSA—which refers to developers' responsibilities to meet the requirements of certification—and section 4002 of the Cures Act—which identifies information blocking as a Condition of Certification.



          Notwithstanding this, the Cures Act does not prescribe that conduct that may implicate the information blocking provisions be limited to practices related to \emph{only} certified health IT. Rather, the information blocking provisions would be implicated by any practice engaged in by an individual or entity that develops or offers certified health IT that is likely to interfere with the access, exchange, or use of EHI, including practices associated with \emph{any} of the developer or offeror's health IT products that have \emph{not} been certified under the Program. This interpretation is based primarily on section 3022(b)(1) of the PHSA. If Congress had intended that the enforcement of the information blocking provisions were limited to practices connected to certified health IT, we believe the Cures Act would have included language that tied enforcement to the operation or performance of a product certified under the Program. Rather, the description of the practices that OIG can investigate in section 3022(b)(1)(A)(ii) of the PHSA are not tied to the certification status of the health IT at issue, omitting any express reference to a health IT developer's practice needing to be related to “certified health information technology.” That the scope of the information blocking provision should not be limited to practices that involve only certified health IT is further evidenced by no such limitation applying to health care providers, health information exchanges (HIEs), and health information networks (HINs) as listed in sections 3022(b)(1) of the PHSA.


          Additionally, the “practice described” in section 3022(a)(2) of the PHSA refers to “certified health information technologies” when illustrating practices that restrict authorized access, exchange, or use of EHI under applicable state or federal laws (section 3022(a)(2)(A) of the PHSA), but omits any reference to certification when describing “health information technology” in the practices described in sections 3022(a)(2)(B) and (C) of the PHSA. Importantly, sections 3022(a)(2)(B) and (C) of the PHSA address practices that are particularly relevant to health IT developers and offerors, although they could be engaged in by other types of actors. We interpret this drafting as a deliberate decision not to link the information blocking provisions with only the performance or use of certified health IT.


          Finally, we note that the Cures Act does not impose a temporal nexus that would require that information blocking be carried out at a time when an individual or entity had health IT certified under the Program. Ostensibly, then, once an individual or entity has health IT certified, or otherwise maintains the certification of health IT, the individual or entity becomes forever subject to the information blocking provision. We do not believe that, understood in context, the Cures Act supports such a broad interpretation. Noting the above discussion concerning OIG's scope of authority under section 3022(b)(1)(A) and (b)(2) of the PHSA, we believe that to make developers and offerors of certified health IT subject to the information blocking provision in perpetuity would be inconsistent with the voluntary nature of the Program. However, we also believe that the Cures Act does not provide any basis for interpreting the information blocking provision so narrowly that a developer or offeror of certified health IT could escape penalty as a consequence of having its certification terminated or by withdrawing all of its extant certifications.



          We consider that in the circumstances where a health IT developer has its certification terminated, or withdraws its certification, such that it no longer has any health IT certified under the  \pagenum{7511}\ifhmode\expandafter\xspace\fi Program, it should nonetheless be subject to penalties for information blocking engaged in during the time that it did have health IT certified under the Program. Accordingly, we propose to adopt a definition of “heath information technology (IT) developer of certified health IT” for the purposes of interpretation and enforcement of the information blocking provisions, including those regulatory provisions proposed under Title 45, part 171, of the Code of Federal Regulations, that would capture such developers or offerors. We propose, in \textsection{} 171.102, that “health IT developer of certified health IT” means an individual or entity that develops or offers health information technology (as that term is defined in 42 U.S.C. 300jj(5)) and which had, at the time it engaged in a practice that is the subject of an information blocking claim, health IT (one or more) certified under the Program. To note, we propose that the term “information blocking claim” within this definition should be read broadly to encompass any statement of information blocking or potential information blocking. “Claims” of information blocking within this definition would not be limited, in any way, to a specific form, format, or submission approach or process.


          We are also considering additional approaches to help ensure that developers and offerors of certified health IT remain subject to the information blocking provision for an appropriate period of time after leaving the Program. The rationale for this approach would be that a developer or offeror of certified health IT should be subject to penalties if, following the termination or withdrawal of certification, it refused to provide its customers with access to the EHI stored in the decertified health IT, provided that such interference was not required by law and did not qualify for one of the information blocking exceptions. Adopting this broader approach would help avoid the risk that a developer would be able to engage in the practices described in section 3022(a)(2) of the PHSA in respect to EHI that was collected on behalf of a health care provider when that health care provider would reasonably expect that the information blocking provision would protect against unreasonable and unnecessary interferences with that EHI. If the information blocking provision did not extend to capture such conduct, the protection afforded by the information blocking provision could become illusory, and providers would need to consider securing contractual rights to prevent interference, which we are aware they typically have great difficulty doing.\textsuperscript{104}
            
          


          \insfootnote{\textsuperscript{104} \emph{See} ONC, \emph{EHR Contracts Untangled Selecting Wisely, Negotiating Terms, And Understanding The Fine Print, \url{https://www.healthit.gov/sites/default/files/EHR_Contracts_Untangled.pdf}} (September 2016).}


          One way that this could be achieved would be to define “health IT developer of certified health IT” as including developers and offerors of certified health IT that continue to store EHI that was previously stored in health IT certified in the Program. Alternatively, we are considering whether developers and offerors of certified health IT should remain subject to the information blocking provision for an appropriate period of time after leaving the Program. Namely, that the information blocking provision should apply for a specific time period, say one year, after the developer or offeror no longer has any health IT certified in the Program. This second approach has the attraction of providing a more certain basis for understanding which developers are subject to the information blocking provision. However, it also potentially captures developers and offerors who have fully removed themselves from the Program and, for example, no longer exercise control over EHI that was stored in their certified health IT.


          We seek comment on which of these two models best achieves our policy goal of ensuring that health IT developers of certified health IT will face consequences under the information blocking provision if they engage in information blocking in connection with EHI that was stored or controlled by the developer or offeror whilst they were participating in the program. Commenters are also encouraged to identify alternative models and approaches for identifying when a developer or offeror should, and should no longer, be subject to the information blocking provision.



          We note that a developer or offeror of a single health IT product that has had its certification suspended would be considered to \emph{have} certified health IT for the purpose of the definition. We also note that we interpret the requirement that the health IT developer of certified health IT “exercise control” over EHI broadly. A developer would not necessarily need to have access to the EHI in order to exercise control. For example, a developer that implemented a “kill-switch” for a decertified software product that was locally hosted by a health care provider, preventing that provider from accessing its records, would be exercising control over the EHI for the purpose of this definition.


          We clarify that we interpret “individual or entity that develops the certified health IT” as the individual or entity that is legally responsible for the certification status of the health IT, which would be the individual or entity that entered into a binding agreement that resulted in the certification status of the health IT under the Program or, if such rights are transferred, the individual or entity that holds the rights to the certified health IT. We also clarify that an “individual or entity that offers certified health IT” would include an individual or entity that under any arrangement makes certified health IT available for purchase or license. We seek comment on both of our interpretations. More specifically, we seek comment on whether there are particular types of arrangements under which certified health IT is “offered” in which the offeror should not be considered a “health IT developer of certified health IT” for the purposes of the information blocking provisions.


          We also clarify that the proposed definition of “health IT developer of certified health IT” and our interpretation of the use of “health information technology developer” applies to Part 171 only and does not apply to the implementation of any other section of the PHSA or the Cures Act, including section 4005(c)(1) of the Cures Act.


          We clarify that API Technology Suppliers, as described in section VII.4 of this preamble and defined in \textsection{} 170.102, would be considered health IT developers of certified health IT subject to the conditions described above.


          Last, we clarify that a “self-developer” of certified health IT, as the term has been used in the ONC Health IT Certification Program (Program) and described in this rulemaking (section VII.D.7) and previous rulemaking,\textsuperscript{105}
             would be treated as a health care provider for the purposes of information blocking. This is because of our description of a self-developer for Program purposes \textsuperscript{106}

             would essentially mean that such developers would not be supplying or offering their certified health IT to other entities. To be clear, self-developers would still be subject to the proposed Conditions and  \pagenum{7512}\ifhmode\expandafter\xspace\fi Maintenance of Certification requirements because they have health IT certified under the Program (\emph{see also} section VII.D.7). We welcome comments on our determination regarding “self-developers” for information blocking purposes and whether there are other factors we should consider in how we treat “self-developers” of certified health IT for the purposes of information blocking.


          \insfootnote{\textsuperscript{105} The final rule establishing ONC's Permanent Certification Program, “Establishment of the Permanent Certification for Health Information” (76 FR 1261), addresses self-developers.}


          \insfootnote{\textsuperscript{106} The language in the final rule describes the concept of “self-developed” as referring to a complete EHR or EHR Module designed, created, or modified by an entity that assumed the total costs for testing and certification and that will be the primary user of the health IT (76 FR 1300).}


          We also seek comment generally on the definition proposed for “health IT developer of certified health IT.”


          \subsubsection{c. Networks and Exchanges}

          The terms “network” and “exchange” are not defined in the information blocking provision or in any other relevant statutory provisions. We propose to define these terms so that these individuals and entities that are covered by the information blocking provision understand that they must comply with its provisions. In accordance with the meaning and intent of the information blocking provision, we believe it is necessary to define these terms in a way that does not assume the application or use of certain technologies and is flexible enough to apply to the full range and diversity of exchanges and networks that exist today and may arise in the future. We note that in the past few years alone many new types of exchanges and networks that transmit EHI have emerged, and we expect this trend to accelerate with continued advancements in technology and renewed efforts to advance trusted exchange among networks and other entities under the trusted exchange framework and common agreement provided for by section 4003(b) of the Cures Act.


          In considering the most appropriate way to define these terms, we examined how they are used throughout the Cures Act and the HITECH Act. Additionally, we considered dictionary and industry definitions of “network” and “exchange.” While these terms have varied usage and meaning in different industry contexts, certain concepts are common and have been incorporated into the proposed definitions below.


          \subsubsection{i. Health Information Network}

          We propose a functional definition of “health information network” (HIN) that focuses on the role of these actors in the health information ecosystem. We believe the defining attribute of a HIN is that it enables, facilitates, or controls the movement of information between or among different individuals or entities that are unaffiliated. For this purpose, we propose that two parties are affiliated if one has the power to control the other, or if both parties are under the common control or ownership of a common owner. We note that a significant implication of this definition is that a health care provider or other entity that enables, facilitates, or controls the movement of EHI within its own organization, or between or among its affiliated entities, is not a HIN in connection with that movement of information for the purposes of this proposed rule.


          More affirmatively, we propose that an actor could be considered a HIN if it performs any or any combination of the following activities. First, the actor would be a HIN if it were to determine, oversee, administer, control, or substantially influence policies or agreements that define the business, operational, technical, or other conditions or requirements that enable or facilitate the access, exchange, or use of EHI between or among two or more unaffiliated individuals or entities. Second, an actor would be a HIN if it were to provide, manage, control, or substantially influence any technology or service that enables or facilitates the access, exchange, or use of EHI between or among two or more unaffiliated individuals or entities.



          Typically, a HIN will influence the sharing of EHI between many unaffiliated individuals or entities. However, we do not propose to establish any minimum number of parties or “nodes” beyond the requirement that there be some actual or contemplated access, exchange, or use of information between or among at least two unaffiliated individuals or entities that is enabled, facilitated, or controlled by the HIN. We believe such a limitation would be artificial and would not capture the full range of entities that should be considered networks under the information blocking provision. To be clear, any individual or entity that enables, facilitates, or controls the access, exchange, or use of EHI between or among only itself and another unaffiliated individual or entity would not be considered a HIN in connection with the movement of that EHI (although that movement of EHI may still be regulated under the information blocking provision on the basis that the individual or entity is a health care provider or health IT developer of certified health IT). To be a HIN, the individual or entity would need to be enabling, facilitating, or controlling the access, exchange, or use of EHI between or among two or more \emph{other} individuals or entities that were not affiliated with it.


          To illustrate how the proposed definition would operate, we note the following examples. An entity is established within a state for the purpose of improving the movement of EHI between the health care providers operating in that state. The entity identifies standards relating to security and offers terms and conditions to be entered into by health care providers wishing to participate in the network. The entity offering (and then overseeing and administering) the terms and conditions for participation in the network would be considered a HIN for the purpose of the information blocking provision. We note that there is no need for a separate entity to be created in order that an entity be considered a HIN. For instance, a health system that administers business and operational agreements for facilitating the exchange of EHI that are adhered to by unaffiliated family practices and specialist clinicians in order to streamline referrals between those practices and specialists would likely be considered a HIN.



          We note that the proposed definition would also encompass an individual or entity that does not directly enable, facilitate, or control the movement of information, but nonetheless exercises control or substantial influence over the policies, technology, or services of a network. In particular, there may be an individual or entity that relies on another entity—such as an entity specifically created for the purpose of managing a network—for policies and technology, but nevertheless dictates the movement of EHI over that network. For example, a large health care provider may decide to lead an effort to establish a network that facilitates the movement of EHI between a group of smaller health care providers (as well as the large health care provider) and through the technology of health IT developers. To achieve this outcome, the large health care provider, together with some of the participants, creates a new entity that administers the network's policies and technology. In this scenario, the large health care provider would come within the functional definition of a HIN and could be held accountable for the conduct of the network if the large health care provider used its control or substantial influence over the new entity—either in a legal sense, such as via its control over the governance or management of the entity, or in a less formal sense, such as if the large health care provider prescribed a policy to be adopted—to interfere with the access, exchange, or use of EHI. We note that the large health care provider in this example would be treated as a health care provider when utilizing the  \pagenum{7513}\ifhmode\expandafter\xspace\fi network to move EHI via the network's policies, technology, or services, but would be considered a HIN in connection with the practices of the network over which the large health care provider exercises control or substantial influence.


          We seek comment on the proposed definition of a HIN. In particular, we request comment on whether the proposed definition is broad enough (or too broad) to cover the full range of individuals and entities that could be considered health information networks within the meaning of the information blocking provision. Additionally, we specifically request comment on whether the proposed definition would effectuate our policy goal of defining this term in a way that does not assume particular technologies or arrangements and is flexible enough to accommodate changes in these and other conditions.


          \subsubsection{ii. Health Information Exchange}

          We propose to define a “health information exchange” (HIE) as an individual or entity that enables access, exchange, or use of EHI primarily between or among a particular class of individuals or entities or for a limited set of purposes. Our research and experience in working with exchanges drove the proposed definition of this term. HIEs include but are not limited to regional health information organizations (RHIOs), state health information exchanges (state HIEs), and other types of organizations, entities, or arrangements that enable EHI to be accessed, exchanged, or used between or among particular types of parties or for particular purposes. For example, an HIE might facilitate or enable the access, exchange, or use of EHI exclusively within a regional area (such as a RHIO), or for a limited scope of participants and purposes (such as a clinical data registry or an exchange established by a hospital-physician organization to facilitate Admission, Discharge, and Transfer (ADT) alerting). We note that HIEs may be established under federal or state laws or regulations but may also be established for specific health care or business purposes or use cases. Additionally, we note that if an HIE facilitates the access, exchange, or use of EHI for more than a narrowly defined set of purposes, then it may be both an HIE and a HIN.


          We seek comment on this proposed definition of an HIE. Again, we encourage commenters to consider whether this proposed definition is broad enough (or too broad) to cover the full range of individuals and entities that could be considered exchanges within the meaning of the information blocking provision, and whether the proposed definition is sufficiently flexible to accommodate changing technological and other conditions.


          \subsubsection{3. Electronic Health Information}

          The definition of information blocking applies to \emph{electronic} health information (EHI) (section 3022(a)(1) of the PHSA). While section 3000(4) of the PHSA by reference to section 1171(4) of the Social Security Act defines “health information,” EHI is not specifically defined in the Cures Act, HITECH Act, or other relevant statutes. We propose to define EHI to mean:


          (i) Electronic protected health information; and


          (ii) any other information that—


          • is transmitted by or maintained in electronic media, as defined in 45 CFR 160.103;


          • identifies the individual, or with respect to which there is a reasonable basis to believe the information can be used to identify the individual; and


          • relates to the past, present, or future health or condition of an individual; the provision of health care to an individual; or the past, present, or future payment for the provision of health care to an individual.


          This definition of EHI includes, but is not limited to, electronic protected health information and health information that is created or received by a health care provider and those operating on their behalf; health plan; health care clearinghouse; public health authority; employer; life insurer; school; or university. In addition, we clarify that under our proposed definition, EHI includes, but is not limited to, electronic protected health information (ePHI) as defined in 45 CFR 160.103. In particular, unlike ePHI and health information, EHI is not limited to information that is created or received by a health care provider, health plan, health care clearinghouse, public health authority, employer, life insurer, school, or university. EHI may be provided, directly from an individual, or from technology that the individual has elected to use, to an actor covered by the information blocking provisions. We propose that EHI does not include health information that is de-identified consistent with the requirements of 45 CFR 164.514(b). We generally request comment on this proposed definition as well as on whether the exclusion of health information that is de-identified is consistent with the requirements of 45 CFR 164.514(b).


          To be clear, this definition provides for an expansive set of EHI, which could include information on an individual's health insurance eligibility and benefits, billing for health care services, and payment information for services to be provided or already provided, which may include price information.


          \subsubsection{Price Information}

          The fragmented and complex nature of pricing within the health care system has decreased the efficiency of the health care system and has had negative impacts on patients, health care providers, health systems, plans, plan sponsors and other key health care stakeholders. Patients and plan sponsors have trouble anticipating or planning for costs, are not sure how they can lower their costs, are not able to compare costs, and have no practical way to measure the quality of the care or coverage they receive relative to the price they pay. Pricing information continues to grow in importance with the increase of high deductible health plans and surprise billing, which have resulted in an increase in out-of-pocket health care spending. Transparency in the price and cost of health care would help address the concerns outlined above by empowering patients to make informed health care decisions. Further, the availability of price information could help increase competition that is based on the quality and value of the services patients receive. Consistent with its statutory authority, the Department is considering subsequent rulemaking to expand access to price information for the public, prospective patients, plan sponsors, and health care providers.


          Increased consumer demand, aligned incentives, more accessible and digestible information, and the evolution of price transparency tools are critical components to moving to a health care system that pays for value. However, the complex and decentralized nature of how price information is created, structured, formatted, and stored presents many challenges to achieving price transparency. To this point, pricing within health care demands a market-based approach whereby, for example, platforms are created that utilize raw data to provide consumers with digestible price information through their preferred medium.



          ONC has a unique role in setting the stage for such future actions by establishing the framework to prevent the blocking of price information. Given that price information impacts the ability of patients to shop for and make decisions about their care, we seek comment on the parameters and implications of including price information within the scope of EHI for purposes of information blocking. In  \pagenum{7514}\ifhmode\expandafter\xspace\fi addition, the overall Department seeks comment on the technical, operational, legal, cultural, environmental and other challenges to creating price transparency within health care.


          • Should prices that are included in EHI:


          ○ Reflect the amount to be charged to and paid for by the patient's health plan (if the patient is insured) and the amount to be charged to and collected from the patient (as permitted by the provider's agreement with the patient's health plan), including for drugs or medical devices;


          ○ Include various pricing information such as charge master price, negotiated prices, pricing based on CPT codes or DRGs, bundled prices, and price to payer;


          ○ Be reasonably available in advance and at the point of sale;


          ○ Reflect all out-of-pocket costs such as deductibles, copayments and coinsurance (for insured patients); and/or


          ○ Include a reference price as a comparison tool such as the Medicare rate and, if so, what is the most meaningful reference?


          • For the purpose of informing referrals for additional care and prescriptions, should future rulemaking by the Department require health IT developers to include in their platforms a mechanism for patients to see price information, and for health care providers to have access to price information, tailored to an individual patient, integrated into the practice or clinical workflow through APIs?


          • To the extent that patients have a right to price information within a reasonable time in advance of care, how would such reasonableness be defined for:


          ○ Scheduled care, including how far in advance should such pricing be available for patients still shopping for care, in addition to those who have already scheduled care;


          ○ Emergency care, including how and when transparent prices should be disclosed to patients and what sort of exceptions might be appropriate, such as for patients in need of immediate stabilization;


          ○ Ambulance services, including air ambulance services; and


          ○ Unscheduled inpatient care, such as admissions subsequent to an emergency visit?


          • How would price information vary based on the type of health insurance and/or payment structure being utilized, and what, if any, challenges would such variation create to identifying the price information that should be made available for access, exchange, or use?



          • Are there electronic mechanisms/processes available for providing price information to patients who are not registered (\emph{i.e.,} not in the provider system) when they try to get price information?


          • Should price information be made available on public websites so that patients can shop for care without having to contact individual providers, and if so, who should be responsible for posting such information? Additionally, how would the public posting of pricing information through API technology help advance market competition and the ability of patients to shop for care?


          • If price information that includes a provider's negotiated rates for all plans and the rates for the uninsured were to be required to be posted on a public website, is there technology currently available or that could be easily developed to translate that data into a useful format for individuals? Are there existing standards and code sets that would facilitate such transmission and translation? To the extent that some data standards are lacking in this regard, could developers make use of unstandardized data?


          • What technical standards currently exist or may be needed to represent price information electronically for purposes of access, exchange, and use?


          • Are there technical impediments experienced by stakeholders regarding price information flowing electronically?


          • Would updates to the CMS-managed HIPAA transactions standards and code sets be necessary to address the movement of price information in a standardized way?


          • How can price transparency be achieved for care delivered through value based arrangements, including at accountable care organizations, demonstrations and other risk-sharing arrangements?


          • What future requirements should the Department consider regarding the inclusion of price information in a patient's EHI, particularly as it relates to the amount paid to a health care provider by a patient (or on behalf of a patient) as well as payment calculations for the future provision of health care to such patient?


          • If price information is included in EHI, could that information be useful in subsequent rulemaking that the Department may consider in order to reduce or prevent surprise medical billing, such as requirements relating to:


          ○ The provision of a single bill that includes all health care providers involved in a health care service, including their network status;


          ○ The provision of a binding quote reasonably in advance of scheduled care (that is, non-emergent care) or some subset of scheduled care, such as for the most “shoppable” services;


          ○ Ensuring that all health care providers in an in-network facility charge the in-network rate; and


          ○ Notification of billing policies such as timely invoice dates for all providers and facilities, notwithstanding network status, due date for invoice payments by the prospective patient's payers and out-of-pocket obligations, date when unpaid balances are referred for collections, and appeals rights and procedures for patients wishing to contest an invoice?


          \subsubsection{4. Interests Promoted by the Information Blocking Provision}

          \subsubsection{a. Access, Exchange, and Use of EHI}


          The information blocking provision promotes the ability to \emph{access, exchange, and use} EHI, consistent with the requirements of applicable law. We interpret the terms “access,” “exchange,” and “use” broadly, consistent with their generally understood meaning in the health IT industry and their function and context in the information blocking provision.


          The concepts of access, exchange, and use are closely related: EHI cannot be used unless it can be accessed, and this often requires that the EHI be exchanged among different individuals or entities and through various technological means. Moreover, the technological and other means necessary to facilitate appropriate access and exchange of EHI vary significantly depending on the purpose for which the information will be used. For example, the technologies and services that support a payer's access to EHI to assess clinical value will likely differ from those that support a patient's access to EHI via a smartphone app. That is, to deter information blocking in these and many other potential uses of EHI—and, by extension, the many and diverse means of access and exchange that support such uses.



          This is consistent with the way these terms are employed in the information blocking provision and in other relevant statutory provisions. For example, section 3022(a)(2) of the PHSA contemplates a broad range of purposes for which EHI may be accessed, exchanged, and used—from treatment, care delivery, and other permitted purposes, to exporting complete information sets and transitioning between health IT systems, to supporting innovations and advancements in health information access, exchange, and use. Separately,  \pagenum{7515}\ifhmode\expandafter\xspace\fi the Cures Act and the HITECH Act contemplate many different purposes for and means of accessing, exchanging, and using EHI, which include, but are not limited to, quality improvement, guiding medical decisions at the time and place of care, reducing medical errors and health disparities, delivering patient-centered care, and supporting public health and clinical research activities.\textsuperscript{107}
            
          


          \insfootnote{\textsuperscript{107} \emph{See} section 3001(b) of the PHSA; \emph{see also} section 3009(a)(3) of the PHSA (enumerating reporting criteria relating to access, exchange, and use of EHI for a broad and diverse range of purposes).}


          In addition to these statutory provisions, we have considered how the terms access, exchange, and use have been defined or used in existing regulations and other relevant health IT industry contexts. While those definitions have specialized meanings and are not controlling here, they are instructive insofar as they illustrate the breadth with which these terms have been understood in other contexts. For example, the HIPAA Privacy Rule defines an individual's right of access to include the right to have a copy of all or part of their PHI transmitted directly to them or any person or entity he or she designates, in any form and format (including electronically) that the individual requests and that the covered entity holding the information can readily produce (45 CFR 164.524). In a different context, the HIPAA Security Rule defines “access” as the ability or the means necessary to read, write, modify, or communicate data/information or otherwise use any system resource (45 CFR 164.304). The HIPAA Rules also define the term “use,” which includes the sharing, employment, application, utilization, examination, or analysis of individually identifiable health information within an entity that maintains the information (45 CFR 160.103).


          As the examples and discussion above demonstrate, the concepts of access, exchange, and use are used in a variety of contexts to refer to a broad spectrum of activities. We believe that the types of access, exchange, and use described above would be promoted under the information blocking provision, as would other types of access, exchange, or use not specifically contemplated in these or other regulations. Further, we note that the information blocking provision would also extend to innovations and advancements in health information access, exchange, and use that may occur in the future (see section 3022(a)(2) of the PHSA).



          Consistent with the above, and to convey the full breadth of activities that may implicate the information blocking provision, we propose definitions of access, exchange, and use in \textsection{} 171.102. We emphasize the interrelated nature of the definitions. For example, the definition of “use” includes the ability to read, write, modify, manipulate, or apply EHI to accomplish a desired outcome or to achieve a desired purpose, while “access” is defined as the ability or means necessary to make EHI available for \emph{use.} As such, interference with “access” would include, for example, an interference that prevented a health care provider from writing EHI to its health IT or from modifying EHI stored in health IT, whether by the provider itself or by, or via, a third-party app. We encourage comment on these definitions. In particular, commenters may wish to consider whether these definitions are broad enough to cover all of the potential purposes for which EHI may be needed and ways in which it could conceivably be used, now and in the future.


          \subsubsection{b. Interoperability Elements}

          In this proposed rule, we use the term “interoperability element” to refer to any means by which EHI can be accessed, exchanged, or used. We clarify that the means of accessing, exchanging, and using EHI are not limited to functional elements and technical information but also encompass technologies, services, policies, and other conditions \textsuperscript{108}
             necessary to support the many potential uses of EHI as described above. Because of the evolving nature of technology and the diversity of privacy laws and regulations, institutional arrangements, and policies that govern the sharing of EHI, we will not provide an exhaustive list of interoperability elements. However, we believe that it is useful to define this term, both because of its importance for analyzing the likelihood of interference under the information blocking provision, and because some of the proposed exceptions to the provision contain conditions concerning the availability and provision of interoperability elements. Therefore, we propose to define “interoperability element” in \textsection{} 171.102. As noted, our intent is to capture all of the potential means by which EHI may be accessed, exchanged, or used for any relevant purposes; both now and as technology and other conditions evolve. We seek comment on whether the proposed definition realizes that intent and, if not, any changes we should consider.


          \insfootnote{\textsuperscript{108} \emph{See} ONC, \emph{Connecting Health and Care for the Nation: A Shared Nationwide Interoperability Roadmap} at x-xi, \emph{\url{https://www.healthit.gov/topic/interoperability/interoperability-roadmap}} (Oct. 2015) [hereinafter “Interoperability Roadmap”].}


          \subsubsection{5. Practices That May Implicate the Information Blocking Provision}


          To meet the definition of information blocking, a practice must be \emph{likely to interfere with, prevent, or materially discourage} access, exchange, or use of EHI. In this section and elsewhere in this preamble, we discuss various types of hypothetical practices that \emph{could} implicate the provision. We do this to illustrate the scope of the information blocking provision and to explain our interpretation of various statutory concepts. However, we stress that the types of practices discussed in this preamble are illustrative, not exhaustive, and that many other types of practices could also implicate the provision. Nor does the fact that we have not identified or discussed a particular type of practice imply that it is less serious than those that are discussed in this preamble. Indeed, because information blocking may take many forms, it is not possible—and we do not attempt—to anticipate or catalog the many potential types of practices that may raise information blocking concerns.



          We emphasize that any analysis of information blocking necessarily requires a careful consideration of the individual facts and circumstances, including whether the practice was required by law, whether the actor had the requisite statutory knowledge, and whether an exception applies. When we state that a practice would implicate the provision or \emph{could} violate the provision, we are expressing a conclusion that the type of practice is one that would be likely to interfere with, prevent, or materially discourage access, exchange, or use of EHI, and that further analysis of these and other statutory elements would therefore be warranted to determine whether a violation has occurred. We highlight this distinction because to \emph{implicate} the information blocking provision is \emph{not} necessarily to \emph{violate} it, and that each case will turn on its own unique facts. For example, a practice that seemingly meets the statutory definition of information blocking would not be information blocking if it was required by law, if one or more elements of the definition were not met, or if was covered by one of the proposed exceptions.



          We propose in section VIII.D of this preamble to establish seven exceptions to the information blocking provision for certain reasonable and necessary activities. If an actor can establish that an exception applies to each practice for which a claim of information blocking  \pagenum{7516}\ifhmode\expandafter\xspace\fi has been made, including that the actor satisfied all applicable conditions of the exception at all relevant times, then the practice would not constitute information blocking.


          Based on early discussions with stakeholders during the development of this proposed rule, we are aware that the generality with which the information blocking provision describes practices that are likely to interfere with, prevent, or materially discourage access, exchange, or use of EHI may leave some uncertainty as to the scope of the information blocking provision and the types of practices that will implicate enforcement by ONC and/or OIG. To provide additional clarity on this point, we elaborate our understanding of these important statutory concepts below.


          \subsubsection{a. Prevention, Material Discouragement, and Other Interference}

          The information blocking provision and its enforcement subsection do not define the terms “interfere with,” “prevent,” and “materially discourage,” and use these terms collectively and without differentiation. Based on our interpretation of the information blocking provision and the ordinary meanings of these terms in the context of EHI, we do not believe they are mutually exclusive, but that prevention and material discouragement are best understood as types of interference, and that use of these terms in the statute to define information blocking illustrates the desire to reach all practices that an actor knows, or should know, are likely to prevent, materially discourage, or otherwise interfere with the access, exchange, or use of EHI. Consistent with this understanding, in this preamble to the proposed rule, we use the terms “interfere with” and “interference” as inclusive of prevention, material discouragement, and other forms of interference that implicate the information blocking provision.


          We believe that interference could take many forms. In addition to the prevention or material discouragement of access, exchange, or use, we propose that interference could include practices that increase the cost, complexity, or other burden associated with accessing, exchanging, or using EHI. Additionally, interference could include practices that limit the utility, efficacy, or value of EHI that is accessed, exchanged, or used, such as by diminishing the integrity, quality, completeness, or timeliness of the data. We refer readers to section VIII.C.5.c of this preamble below for a discussion of these and other potential practices that could interfere with access, exchange, or use and thereby implicate the information blocking provision.


          Relatedly, to avoid potential ambiguity and clearly communicate the full range of potential practices that could implicate the information blocking provision, we propose to codify a definition of “interfere with” in \textsection{} 171.102, consistent with our interpretation set forth above.


          \subsubsection{b. Likelihood of Interference}


          The information blocking provision is preventative in nature. That is, the information blocking provision proscribes practices that are \emph{likely} to interfere with (including preventing or materially discouraging) access, exchange, or use of EHI—whether or not such harm actually materializes. By including both the likely and the actual effects of a practice, the information blocking provision encourages individuals and entities to avoid engaging in practices that undermine interoperability, and to proactively promote access, exchange, and use of EHI.


          We believe that a practice would satisfy the information blocking provision's “likelihood” requirement if, under the circumstances, there is a reasonably foreseeable risk that the practice will interfere with access, exchange, or use of EHI. For example, where an actor refuses to share EHI or to provide access to certain interoperability elements, it is reasonably foreseeable that such actions will interfere with access, exchange, or use of EHI. As another example, it is reasonably foreseeable that a health care provider may need to access information recorded in a patient's electronic record that could be relevant to the treatment of that patient. For this reason, a policy or practice that limits timely access to such information in an appropriate electronic format creates a reasonably foreseeable likelihood of interfering with the use of the information for these treatment purposes.


          Whether the risk of interference is reasonably foreseeable will depend on the particular facts and circumstances attending the practice or practices at issue. Because of the number and diversity of potential practices, and the fact that different practices will present varying risks of interfering with access, exchange, or use of EHI, we do not attempt to anticipate all of the potential ways in which the information blocking provision could be implicated. Nevertheless, to assist with compliance, we clarify certain circumstances in which, based on our experience, a practice will almost always be likely to interfere with access, exchange, or use of EHI. We caution that these situations are not exhaustive and that other circumstances may also give rise to a very high likelihood of interference under the information blocking provision. In each case, ONC will consider the totality of the circumstances in evaluating whether a practice is likely to implicate the statute and to give rise to a violation.


          \subsubsection{i. Observational Health Information}

          Although the information blocking provision applies to all EHI, we believe that information blocking concerns are especially pronounced when the conduct at issue has the potential to interfere with the access, exchange, or use of EHI that is created or maintained during the practice of medicine or the delivery of health care services to patients. We refer to such information in this section of the preamble collectively as “observational health information.” Such information includes, but is not limited to, health information about a patient that could be captured in a patient record within an EHR and other clinical information management systems; as well as information maintained in administrative and other IT systems when the information is clinically relevant, directly supports patient care, or facilitates the delivery of health care services to consumers. We note that there is a special need for timely, electronic access to this information and that, moreover, the clinical and operational utility of this information is often highly dependent on multiple actors exercising varying forms and degrees of control over the information itself or the technological, contractual, or other means by which it can be accessed, exchanged, and used. Against these indications, practices that adversely impact the access, exchange, or use of observational health information will almost always implicate the information blocking provision.



          We note that observational health information may be technically structured or unstructured (such as “free text”). Therefore, in general, clinicians' notes would constitute observational health information, at least insofar as the notes contain observations or conclusions about a patient or the patient's care. In contrast, we believe certain types of EHI are qualitatively distinct from observational health information, such as EHI that is created through aggregation, algorithms, and other techniques that transform observational health information into fundamentally new data or insights that are not obvious from the observational  \pagenum{7517}\ifhmode\expandafter\xspace\fi information alone. This could include, for example, population-level trends, predictive analytics, risk scores, and EHI used for comparisons and benchmarking activities. Similarly, internally developed quality measures and care protocols are generally distinct from observational health information. In general, we believe that practices that pertain solely to the creation or use of these transformative data and insights would not usually present the very high likelihood of interference described above. However, we emphasize that, depending on the specific facts at issue, practices related to electronic non-observational health information (a type of EHI), such as price information, \emph{could} still be subject to the information blocking provision. We seek comment on this proposed approach and encourage commenters to identify potential practices related to non-observational health information that could raise information blocking concerns.


          Finally, we clarify that merely collecting, organizing, formatting, or processing observational health information maintained in EHRs and other source systems does not change the fundamental nature of that EHI or obligations under the information blocking provisions. Likewise, the mere fact that EHI is stored in a proprietary format or has been combined with confidential or proprietary information does not alter the actor's obligations under the information blocking provisions to facilitate access, exchange, and use of the EHI in response to a request. For example, the information blocking provision would be implicated if an actor were to assert proprietary rights in medical vocabularies or code sets in a way that was likely to interfere with the access, exchange, or use of observational health information stored in such formats. However, as noted in section VIII.D.6 of this preamble, under the exception for licensing of interoperability elements on reasonable and non-discriminatory terms, an actor could charge a royalty for access to proprietary data or data coded in a proprietary manner so long as that royalty were offered on reasonable and non-discriminatory terms pursuant to the conditions outlined in the exception.


          \subsubsection{ii. Purposes for Which Information May be Needed}

          We believe the information blocking provision will almost always be implicated when a practice interferes with access, exchange, or use of EHI for certain purposes, including but not limited to:



          • Providing patients with access to their EHI and the ability to exchange and use it without special effort (\emph{see} section VII.B.4).


          • Ensuring that health care professionals, care givers, and other authorized persons have the EHI they need, when and where they need it, to make treatment decisions and effectively coordinate and manage patient care and can use the EHI they may receive from other sources.


          • Ensuring that payers and other entities that purchase health care services can obtain the information they need to effectively assess clinical value and promote transparency concerning the quality and costs of health care services.


          • Ensuring that health care providers can access, exchange, and use EHI for quality improvement and population health management activities.


          • Supporting access, exchange, and use of EHI for patient safety and public health purposes.


          The need to ensure that EHI is readily available and usable for these purposes is paramount. Therefore, practices that increase the cost, difficulty, or other burden of accessing, exchanging, or using EHI for these purposes would almost always implicate the information blocking provision. Individuals and entities that develop health IT or have a role in making these technologies and services available should consider the impact of their actions and take steps to support interoperability and avoid impeding the availability or use of EHI.


          \subsubsection{iii. Control Over Essential Interoperability Elements; Other Circumstances of Reliance or Dependence}

          An actor may have substantial control over one or more interoperability elements that provide the only reasonable means of accessing, exchanging, or using EHI for a particular purpose. In these circumstances, any practice by the actor that could impede the use of the interoperability elements—or that could unnecessarily increase the cost or other burden of using the elements—would almost always implicate the information blocking provision.



          The situation described above is most likely when customers or users are dependent on an actor's technology or services, which can occur for any number of reasons. For example, technological dependence may arise from legal or commercial relations, such as a health care provider's reliance on its EHR developer to ensure that EHI managed on its behalf is accessible and usable when it is needed. Relatedly, most EHI is currently stored in EHRs and other source systems that use proprietary data models or formats. Knowledge of the data models, formats, or other relevant technical information (\emph{e.g.,} proprietary APIs) is necessary to understand the data and make efficient use of it in other applications and technologies. Because this information is routinely treated as confidential or proprietary, the developer's cooperation is required to enable uses of the EHI that go beyond the capabilities provided by the developer's technology. This includes the capability to export complete information sets and to migrate data in the event that a user decides to switch to a different technology.


          Separate from these contractual and intellectual property issues, users may become “locked in” to a particular technology, HIE, or HIN for financial or business reasons. For example, many health care providers have invested significant resources to adopt EHR technologies—including costs for deployment, customization, data migration, and training—and have tightly integrated these technologies into their information management strategies, clinical workflows, and business operations. As a result, they may be reluctant to switch to other technologies due to the significant cost and disruption this would entail.


          Another important driver of technological dependence is the “network effects” of health IT adoption, which are amplified by a reliance on technologies and approaches that are not standardized and do not enable seamless interoperability. Consequently, health care providers and other health IT users may gravitate towards and become reliant on the proprietary technologies, HIEs, or HINs that have been adopted by other individuals and entities with whom they have the greatest need to exchange EHI. These effects may be especially pronounced within particular product or geographic areas. For example, a HIN that facilitates certain types of exchange or transactions may be so widely adopted that it is a de facto industry standard. A similar phenomenon may occur within a particular geographic area once a critical mass of hospitals, physicians, or other providers adopt a particular EHR technology, HIE, or HIN.



          In these and other analogous circumstances of reliance or dependence, there is a heightened risk that an actor's conduct will interfere with access, exchange, or use of EHI. To assist with compliance, we highlight the following common scenarios, based on our outreach to stakeholders, in which  \pagenum{7518}\ifhmode\expandafter\xspace\fi actors exercise control over key interoperability elements.\textsuperscript{109}
            
          


          \insfootnote{\textsuperscript{109} As an important clarification, we note that control over interoperability elements may exist with or without the actor's ability to manipulate the price of the interoperability elements in the market.}


          • Health IT developers of certified health IT that provide EHR systems or other technologies used to capture EHI at the point of care are in a unique position to control subsequent access to and use of that information.


          • HINs and HIEs may be in a unique position to control the flow of information among particular persons or for particular purposes, especially if the HIN or HIE has achieved significant adoption in a particular geographic area or for a particular type of health information use case.


          • Similar control over EHI may be exercised by other entities, such as health IT developers of certified health IT, that supply or control proprietary technologies, platforms, or services that are widely adopted by a class of users or that are a “de facto standard” for certain types of EHI exchanges or transactions.


          • Health care providers within health systems and other entities that provide health IT platforms, infrastructure, or information sharing policies may have a degree of control over interoperability or the movement of data within a geographic area that is functionally equivalent to the control exercised by a dominant health IT developer, HIN, or HIE.


          To avoid violating the information blocking provision, actors with control over interoperability elements should be careful not to engage in practices that exclude persons from the use of those elements or create artificial costs or other impediments to their use.


          We encourage comment on these and other circumstances that may present an especially high likelihood that a practice will interfere with access, exchange, or use of EHI within the meaning of the information blocking provision.


          \subsubsection{c. Examples of Practices Likely To Interfere With Access, Exchange, or Use of EHI}

          To further clarify the scope of the information blocking provision, below we describe several types of practices that would be likely to interfere with access, exchange, or use of EHI. These examples clarify and expand on those set forth in section 3022(a)(2) of the PHSA.


          Because information blocking can take many forms, we emphasize that the categories of practices described below are illustrative only and do not provide an exhaustive list or comprehensive description of practices that may implicate the information blocking provision and its penalties. We also reiterate that to implicate the provision is not necessarily to violate it, and that each case will turn on its own unique facts. For instance, a practice that seemingly meets the statutory definition of information blocking would not be information blocking if it was required by law, if one or more elements of the definition were not met, or if it was covered by one of the proposed exceptions for certain reasonable and necessary activities detailed in section VIII.D of this preamble. For the purposes of the following discussion, we do not consider the applicability of any exceptions proposed in section VIII.D of this preamble; we therefore strongly encourage readers to review that section in conjunction with the discussion of practices in this section below.


          \subsubsection{i. Restrictions on Access, Exchange, or Use}

          The information blocking provision establishes penalties, including civil monetary penalties, or requires appropriate disincentives, for practices that restrict access, exchange, or use of EHI for permissible purposes. For example, section 3022(a)(2)(A) of the PHSA states that information blocking may include practices that restrict authorized access, exchange, or use for treatment and other permitted purposes under applicable law. Section 3022(a)(2)(C)(i) of the PHSA states that information blocking may include implementing health IT in ways that are likely to restrict the access, exchange, or use of EHI with respect to exporting complete information sets or in transitioning between health IT systems.


          One means by which actors may restrict access, exchange, or use of EHI is through formal restrictions. These may be expressed in contract or license terms, EHI sharing policies, organizational policies or procedures, or other instruments or documents that set forth requirements related to EHI or health IT. Additionally, in the absence of an express contractual restriction, an actor may achieve the same result by exercising intellectual property or other rights in ways that restrict access, exchange, or use. As an illustration, the following non-exhaustive examples illustrate types of formal restrictions that would likely implicate the information blocking provision. As stated above, the examples throughout this section VIII.C.5.c. are presented without consideration to whether a proposed exception applies, and readers are encouraged to familiarize themselves with section VIII.D of this preamble.


          • A health system's internal policies or procedures require staff to obtain an individual's written consent before sharing any of a patient's EHI with unaffiliated providers for treatment purposes even though obtaining an individual's consent is not required by state or federal law.


          • An EHR developer's software license agreement prohibits a customer from disclosing to its IT contractors certain technical interoperability information without which the customer and its IT contractors cannot efficiently export and convert EHI for use in other applications.


          • A HIN's participation agreement prohibits entities that receive EHI through the HIN from transmitting that EHI to entities who are not participants of the HIN.


          • An EHR developer sues to prevent a clinical data registry from providing interfaces to physicians who use the developer's EHR technology and wish to submit EHI to the registry. The EHR developer claims that the registry is infringing the developer's copyright in its database because the interface incorporates data mapping that references the table headings and rows of the EHR database in which the EHI is stored.


          Access, exchange, or use of EHI can also be restricted in less formal ways. The information blocking provision would be implicated, for example, where an actor simply refuses to exchange or to facilitate the access or use of EHI, either as a general practice or in isolated instances. The refusal may be expressly stated, or it may be implied from the actor's conduct, as where the actor ignores requests to share EHI or provide interoperability elements; gives implausible reasons for not doing so; or insists on terms or conditions that are so objectively unreasonable that they amount to a refusal to provide access, exchange, or use of the EHI. Some examples of informal restrictions include, but are not limited to:


          • A health IT developer of certified health IT refuses to license interoperability elements that are reasonably necessary for the developer's customers, their IT contractors, and other health IT developers to develop and deploy software that will work with the certified health IT.



          • A health system incorrectly claims that the HIPAA Rules or other legal requirements preclude it from exchanging EHI with unaffiliated providers. \pagenum{7519}\ifhmode\expandafter\xspace\fi 
          


          • An EHR developer ostensibly allows third-party developers to deploy apps that are interoperable with its EHR system. However, as a condition of doing so, the third-party developers must provide their source code and grant the EHR developer the right to use it for its own purposes—terms that almost no developer would willingly accept.


          • A provider notifies its EHR developer of its intent to switch to another EHR system and requests a complete export of its EHI. The developer will provide only the EHI in a PDF format, even though it already can and does produce the data in a commercially reasonable structured format.


          We emphasize that restrictions on access, exchange, or use that are required by law would not implicate the information blocking provision. Moreover, we recognize that some restrictions, while not required by law, may be reasonable and necessary for the privacy and security of individuals' EHI; such practices may qualify for protection under the exceptions proposed in section VIII.D.2 and 3 of this preamble.


          \subsubsection{ii. Limiting or Restricting the Interoperability of Health IT}


          The information blocking provision includes practices that restrict the access, exchange, or use of EHI in various ways (\emph{see} section 3022(a)(2) of the PHSA). These practices could include, for example, disabling or restricting the use of a capability that enables users to share EHI with users of other systems or to provide access to EHI to certain types of persons or for certain purposes that are legally permissible. In addition, the information blocking provision would be implicated where an actor configures or otherwise implements technology in ways that limit the types of data elements that can be exported or used from the technology. Other practices that would be suspect include configuring capabilities in a way that removes important context, structure, or meaning from the EHI, or that makes the data less accurate, complete, or usable for important purposes for which it may be needed. Likewise, implementing capabilities in ways that create unnecessary delays or response times, or that otherwise limit the timeliness of EHI accessed or exchanged, would interfere with the access, exchange, and use of that information and would therefore implicate the information blocking provision. We note that any conclusions regarding such interference would be based on fact-finding specific to each case and would need to consider the applicability of an exception.



          We propose that the information blocking provision would be implicated if an actor were to deploy technological measures that limit or restrict the ability to reverse engineer the functional aspects of technology in order to develop means for extracting and using EHI maintained in the technology. This may include, for example, employing technological protection measures that, if circumvented, would trigger liability under the Digital Millennium Copyright Act (\emph{see} 17 U.S.C. 1201) or other laws.


          The following hypothetical situations illustrate some (though not all) of the types of practices described above and which would implicate the information blocking provision.



          • A health system implements locally-hosted EHR technology certified to proposed \textsection{} 170.315(g)(10) (the health system acts as an API Data Provider as defined by \textsection{} 170.102). As required by proposed \textsection{} 170.404(b)(2), the technology developer provides the health system with the capability to automatically publish its production endpoints (\emph{i.e.,} the internet servers that an app must “call” and interact with in order to request and exchange patient data). The health system chooses not to enable this capability, however, and provides the production endpoint information only to apps it specifically approves. This prevents other applications—and patients that use them—from accessing data that should be made readily accessible via standardized APIs.



          • A hospital directs its EHR developer to configure its technology so that users cannot easily send electronic patient referrals and associated EHI to unaffiliated providers, even when the user knows the Direct address and/or identity (\emph{i.e.,} National Provider Identifier) of the unaffiliated provider.



          • An EHR developer that prevents (such as by way of imposing exorbitant fees unrelated to the developer's costs, or by some technological means) a third-party clinical decision support (CDS) app from writing EHI to the records maintained by the EHR developer on behalf of a health care provider (despite the provider authorizing the third-party app developer's use of EHI) because the EHR developer: (1) Offers a competing CDS software to the third-party app; and (2) includes functionality (\emph{e.g.,} APIs) in its health IT that would provide the third party with the technical capability to modify those records as desired by the health care provider.



          • Although an EHR developer's patient portal offers the capability for patients to directly transmit or request for direct transmission of their EHI to a third party, the developer's customers (\emph{e.g.,} health care providers) choose not to enable this capability.


          • A health care provider has the capability to provide same-day access to EHI in a form and format requested by a patient or a patient's health care provider, but takes several days to respond.


          \subsubsection{iii. Impeding Innovations and Advancements in Access, Exchange, or Use or Health IT-Enabled Care Delivery}

          The information blocking provision encompasses practices that create impediments to innovations and advancements to the access, exchange, and use of EHI, including care delivery enabled by health IT (section 3022(a)(2)(C)(ii) of the PHSA). Importantly, the information blocking provision would be implicated and penalties may apply if an actor were to engage in exclusionary, discriminatory, or other practices that impede the development, dissemination, or use of interoperable technologies and services that enhance access, exchange, or use of EHI.


          Most acutely, the information blocking provision would be implicated if an actor were to refuse to license or allow the disclosure of interoperability elements to persons who require those elements to develop and provide interoperable technologies or services—including those that might complement or compete with the actor's own technology or services. The same would be true if the actor were to allow access to interoperability elements but were to restrict their use for these purposes. The following examples, which are not exhaustive, illustrate practices that would likely implicate the information blocking provision by interfering with access, exchange, or use of EHI:


          • A health IT developer of certified health IT refuses to license an API's interoperability elements, to grant the rights necessary to commercially distribute applications that use the API's interoperability elements, or to provide the related services necessary to enable the use of such applications in production environments.


          • An EHR developer of certified health IT requires third-party applications to be “vetted” for security before use but does not promptly conduct the vetting or conducts the vetting in a discriminatory or exclusionary manner.



          • A health IT developer of certified health IT refuses to license interoperability elements that other software applications require to efficiently access, exchange, and use  \pagenum{7520}\ifhmode\expandafter\xspace\fi EHI maintained in the developer's technology.


          Rather than restricting interoperability elements, an actor may insist on terms or conditions that are burdensome and discourage their use. These practices would implicate the information blocking provision for the reasons described above. Consider the following non-exhaustive examples:



          • An EHR developer of certified health IT maintains an “app store” through which other developers can have “apps” listed that run natively on the EHR developer's platform. However, if an app “competes” with the EHR developer's apps or apps it plans to develop, the developer \emph{requires} that the app developer grant the developer the right to use the app's source code.


          • A health care provider engages a systems integrator to develop an interface engine. However, the provider's license agreement with its EHR developer prohibits it from disclosing technical documentation that the systems integrator needs to perform the work. The EHR developer states that it will only permit the systems integrator to access the documentation if all of its employees sign a broad non-compete agreement that would effectively bar them from working for any other health IT companies.


          The information blocking provision would be implicated also if an actor were to discourage efforts to develop or use interoperable technologies or services by exercising its influence over customers, users, or other persons, as in the following non-exhaustive examples:


          • An EHR developer of certified health IT maintains an “app store” through which other developers can have “apps” listed that run natively on the EHR developer's platform. The EHR developer charges app developers a substantial fee for this service unless an app developer agrees not to deploy the app in any other EHR developers' app stores.


          • A hospital is working with several health IT developers to develop an application that will enable ambulatory providers who use different EHR systems to access and update patient data in the hospital's EHR system from within their ambulatory EHR workflows. The inpatient EHR developer, being a health IT developer of certified health IT, pressures the hospital to abandon this project, stating that if it does not it will no longer receive the latest updates and features for its inpatient EHR system.


          • A health IT developer of certified health IT discourages customers from procuring data integration capabilities from a third-party developer, claiming that it will be providing such capabilities free of charge in the next release of its product. In reality, the capabilities it is developing are more limited in scope and are still 12-18 months from being production-ready.


          • A health system insists that local physicians adopt its EHR platform, which provides limited connectivity with competing hospitals and facilities. The health system threatens to revoke admitting privileges for physicians that do not comply.


          Similar concerns would arise were an actor to engage in discriminatory practices—such as imposing unnecessary and burdensome administrative, technical, contractual, or other requirements on certain persons or classes of persons—that interfere with access and exchange or EHI by frustrating or discouraging efforts to enable interoperability. The following non-exhaustive examples illustrate some ways this could occur:


          • An HIN charges additional fees, requires more stringent testing or certification requirements, or imposes additional terms for participants that are competitors, are potential competitors, or may use EHI obtained via the HIN in a way that facilitates competition with the HIN.


          • A health care provider imposes one set of fees and terms to establish interfaces or data sharing arrangements with several registries and exchanges, but offers another more costly or significantly onerous set of terms to establish substantially similar interfaces and arrangements with an HIE or HIN that is used primarily by health plans that purchase health care services from the provider at negotiated reduced rates.


          • A health IT developer of certified health IT charges customers fees, throttles speeds, or limits the number of records they can export when exchanging EHI with a regional HIE that supports exchange among users of competing health IT products, but does not impose like fees or limitations when its customers exchange EHI with enterprise HIEs that primarily serve users of the developer's own technology.


          • As a condition of disclosing interoperability elements to third-party developers, an EHR developer requires third-party developers to enter into business associate agreements with all of the EHR developer's covered entity customers, even if the work being done is not for the benefit of the covered entities.


          • A health IT developer of certified health IT takes significantly longer to provide or update interfaces that facilitate the exchange of EHI with users of competing technologies or services.


          We clarify that not all instances of differential treatment would necessarily constitute a discriminatory practice that implicates the information blocking provision. For example, different fee structures or other terms may reflect genuine differences in the cost, quality, or value of the EHI and the effort required to provide access, exchange, or use. We also note that, in certain circumstances, it may be reasonable and necessary for an actor to restrict or impose reasonable and non-discriminatory terms or conditions on the use of interoperability elements, even though such practices could implicate the information blocking provision. For this reason, we propose in section VIII.D.6 of this preamble to establish a narrow exception that would apply to these types of practices.


          \subsubsection{iv. Rent-Seeking and Other Opportunistic Pricing Practices}

          Certain practices that artificially increase the cost and expense associated with accessing, exchanging, and using EHI will implicate the information blocking provision. Such practices are plainly contrary to the information blocking provision and the concerns that motivated its enactment.



          An actor may seek to extract profits or capture revenue streams that would be unobtainable without control of a technology or other interoperability elements that are necessary to enable or facilitate access, exchange, or use of EHI. As discussed in section VIII.C.5.b.iii of this proposed rule, most EHI is currently stored in EHRs and other source systems that use proprietary data models or formats; this puts EHR developers (and other actors that control data models or standards) in a unique position to block access to (including the export and portability of) EHI for use in competing systems or applications, or to charge rents for access to the basic technical information needed to accomplish the access, exchange, or use of EHI for these purposes. These information blocking concerns may be compounded to the extent that EHR developers do not disclose, in advance, the fees they will charge for interfaces, data export, data portability, and other interoperability-related services (\emph{see} 80 FR 62719; 80 FR 16880-81). We note that these concerns are not limited to EHR developers. Other actors who exercise substantial control over EHI or essential interoperability elements may engage in analogous behaviors that would implicate the information blocking provision. \pagenum{7521}\ifhmode\expandafter\xspace\fi 
          


          To illustrate, we provide the following non-exhaustive examples, which reflect some of the more common types of rent-seeking and opportunistic behaviors of which we are aware and that are likely to interfere with access, exchange, or use of EHI:


          • An EHR developer of certified health IT charges customers a fee to provide interfaces, connections, data export, data conversion or migration, or other interoperability services, where the amount of the fee exceeds the actual costs that the developer reasonably incurred to provide the services to the particular customer(s).


          • An EHR developer of certified health IT charges a fee to perform an export using the EHI export capability proposed in \textsection{} 170.315(b)(10) for the purposes of switching health IT systems or to provide patients access to EHI.


          • An EHR developer of certified health IT charges more to export or use EHI in certain situations or for certain purposes, such as when a customer is transitioning to a competing technology or attempting to export data for use with a HIE, third-party application, or other technology or service that competes with the revenue opportunities associated with the EHR developer's own suite of products and services.


          • An EHR developer of certified health IT interposes itself between a customer and a third-party developer, insisting that the developer pay a licensing fee, royalty, or other payment in exchange for permission to access the EHR system or related documentation, where the fee is not reasonably necessary to cover any additional costs the EHR developer incurs from the third-party developer's activities.


          • An analytics company provides services to the customers of an EHR developer of certified health IT, including de-identifying customer EHI and combining it with other data to identify areas for quality improvement. The EHR developer insists on a revenue sharing arrangement whereby it would receive a percentage of the revenue generated from these activities in return for facilitating access to its customers' EHI, which turns out to be disadvantageous to customers. The revenue the EHR developer would receive exceeds its reasonable costs of facilitating the access to EHI.



          The information blocking provision would clearly be implicated by these and other practices by which an actor profits from its unreasonable control over EHI or interoperability elements without adding any efficiency to the health care system or serving any other procompetitive purpose. But the reach of the information blocking provision is not limited to these types of practices. We interpret the definition of information blocking to encompass \emph{any} fee that materially discourages or otherwise imposes a material impediment to access, exchange, or use of EHI. We use the term “fee” in the broadest possible sense to refer to any present or future obligation to pay money or provide any other thing of value and propose to include this definition in \textsection{} 171.102. We believe this scope may be broader than necessary to address genuine information blocking concerns and could unnecessarily diminish investment and innovation in interoperable technologies and services. Therefore, as discussed in section VIII.D.4 of this preamble, we propose to create an exception that, subject to certain conditions, would permit the recovery of costs that are reasonably incurred to provide access, exchange, and use of EHI. We refer readers to that section for additional details regarding this proposal.


          \subsubsection{v. Non-Standard Implementation Practices}

          Section 3022(a)(2)(B) of the PHSA states that information blocking may include implementing health IT in non-standard ways that substantially increase the complexity or burden of accessing, exchanging, or using EHI. In general, this type of interference is likely to occur when, despite the availability of generally accepted technical, policy, or other approaches that are suitable for achieving a particular implementation objective, an actor does not implement the standard, does not implement updates to the standard, or implements the standard in a way that materially deviates from its formal specifications. These practices lead to unnecessary complexity and burden, such as the additional cost and effort required to implement and maintain “point-to-point” connections, custom-built interfaces, and one-off trust agreements.


          While each case will necessarily depend on its individual facts, and while we recognize that the development and adoption of standards across the health IT industry is an ongoing process, we propose that the information blocking provision would be implicated in at least two distinct sets of circumstances. First, information blocking may arise where an actor chooses not to adopt, or to materially deviate from, relevant standards, implementation specifications, and certification criteria adopted by the Secretary under section 3004 of the PHSA. Second, even where no federally adopted or identified standard exists, if a particular implementation approach has been broadly adopted in a relevant industry segment, deviations from that approach would be suspect unless strictly necessary to achieve substantial efficiencies.


          To further illustrate these types of practices that would implicate the information blocking provision, we provide the following non-exhaustive examples of conduct that would be likely to interfere with access, exchange, or use of EHI:


          • An EHR developer of certified health IT implements the C-CDA for receiving transitions of care summaries but only sends transitions of care summaries in a proprietary or outmoded format.


          • A health IT developer of certified health IT adheres to the “required” portions of a widely adopted industry standard but chooses to implement proprietary approaches for “optional” parts of the standard when other interoperable means are readily available.


          Even where no standards exist for a particular purpose, actors should not design or implement health IT in non-standard ways that unnecessarily increase the costs, complexity, and other burden of accessing, exchanging, or using EHI. For example, an EHR developer of certified health IT designs its database tables in a way that is unreasonably difficult to “map” to a non-proprietary format, which is a necessary prerequisite to converting the EHI to a format that can be used in other software applications. When a customer requests the capability to export EHI to a clinical data registry, the EHR developer quotes substantial costs resulting from the need to write custom code to enable this functionality. Based on these facts, the fees do not reflect costs that are reasonably incurred to provide the service and are instead the result of the developer's impractical design choices. We are aware that some actors attribute certain non-standard implementations on legacy systems that the actor did not themselves design but which have to be integrated into the actor's health IT. Such instances will be considered on a case by case basis.



          Again, we reiterate that information blocking can take many forms and that the practices (and categories of practices) described above do not provide an exhaustive list or comprehensive description of practices that may implicate the information blocking provision. \pagenum{7522}\ifhmode\expandafter\xspace\fi 
          


          \subsubsection{6. Applicability of Exceptions}

          \subsubsection{a. Reasonable and Necessary Activities}


          As discussed above, section 3022(a)(3) authorizes the Secretary to identify, through notice and comment rulemaking, reasonable and necessary \emph{activities} that do not constitute information blocking for purposes of the definition set forth in section 3022(a)(1). Separately, the Cures Act identifies at section 3022(a)(1) \emph{practices} that contravene the definition of information blocking. Following this Cures Act terminology, conduct that implicates the information blocking provision and that does not fall within one of the exceptions described in section VIII.D of this preamble, or does not meet all conditions for an exception, would be considered a “practice.” Conduct that falls within an exception and meets all the applicable conditions for that exception would be considered an “activity.” The challenge with this distinction is that when examining conduct that is the subject of an information blocking claim— an actor's actions that likely interfered with access, exchange, or use of EHI—it can be illusory to distinguish, on its face, conduct that is a \emph{practice} and conduct that is an \emph{activity.} Indeed, conduct that implicates the information blocking provision but falls within an exception could nonetheless be considered information blocking in the event that the actor has not satisfied the conditions applicable to that exception.



          While we acknowledge the terminology used in the Cures Act, we propose to use the term “practice” throughout this proposed rule when we describe conduct that is likely to interfere with, prevent, or materially discourage access, exchange, or use of electronic health information, regardless of whether that conduct meets the conditions for an exception to the information blocking provision. Consistent with this approach, when identifying reasonable and necessary activities in \textsection{}\textsection{} 171.200 through 171.206, we describe \emph{practices} that, if all the applicable conditions are met, are reasonable and necessary and not information blocking. We have taken this approach, in part, because we believe that to adopt the terminology of activity to describe conduct that may or may not be information blocking would confuse the reader and obfuscate our intent in certain circumstances. As an illustration, a health care provider may implement an organizational security policy that limits access, exchange, or use of certain information to certain users (\emph{e.g.,} role-based access). Prior to determining whether the implementation of the security policy is reasonable and necessary under the circumstances, such conduct would be considered a “practice” that implicates the information blocking provision. However, it may later be determined that such conduct is reasonable and necessary and would then be considered an “activity.” Due to these types of scenarios, we contend that the better approach is to use one term—practice—throughout the proposed rule and clarify when describing the conduct at issue whether it is a practice that is information blocking, a practice that implicates the information blocking provision, or a practice that is reasonable and necessary and not information blocking.


          \subsubsection{b. Treatment of Different Types of Actors}

          The proposed exceptions would apply to health care providers, health IT developers of certified health IT, HIEs, and HINs who engage in certain practices covered by an exception, provided that all applicable conditions of the exception are satisfied at all relevant times and for each practice for which the exception is sought. The exceptions are generally applicable to all actors. However, in some instances we propose conditions within an exception that apply to a particular type of actor.


          \subsubsection{c. Establishing That Activities and Practices Meet the Conditions of an Exception}

          We propose that, in the event of an investigation of an information blocking complaint, an actor must demonstrate that an exception is applicable and that the actor met all relevant conditions of the exception at all relevant times and for each practice for which the exception is sought. We consider this allocation of proof to be a substantive condition of the proposed exceptions. As a practical matter, we propose that actors are in the best position to demonstrate compliance with the conditions of the proposed exceptions and to produce the detailed evidence necessary to demonstrate that compliance. We request comment about the types of documentation and/or standardized methods that an actor may use to demonstrate compliance with the exception conditions.


          \subsection{D. Proposed Exceptions to the Information Blocking Provision}

          We propose to establish seven exceptions to the information blocking provision. The exceptions would apply to certain activities that may technically meet the definition of information blocking but that are reasonable and necessary to further the underlying public policies of the information blocking provision.


          The seven proposed exceptions are based on three related policy considerations. First, each exception is limited to certain activities that clearly advance the aims of the information blocking provision. These reasonable and necessary activities include providing appropriate protections to prevent harm to patients and others; promoting the privacy and security of EHI; promoting competition and innovation in health IT and its use to provide health care services to consumers, and to develop more efficient means of health care delivery; and allowing system downtime in order to implement upgrades, repairs, and other changes to health IT. Second, each exception addresses a significant risk that regulated actors will not engage in these beneficial activities because of uncertainty concerning the breadth or applicability of the information blocking provision. Finally, each exception is subject to strict conditions to ensure that it is limited to activities that are reasonable and necessary.


          The first three exceptions, set forth in VIII.D.1-D.3, extend to certain activities that are reasonable and necessary to prevent harm to patients and others; promote the privacy of EHI; and promote the security of EHI, subject to strict conditions to prevent the exceptions from being misused. We believe that without these exceptions, actors may be reluctant to engage in the types of reasonable and necessary activities described below, and that this could erode trust in the health IT ecosystem and undermine efforts to provide access and facilitate the exchange and use of EHI for important purposes. Such a result would be contrary to the purpose of the information blocking provision and the broader policies of the Cures Act.



          The next three exceptions, set forth in VIII.D.4-D.6, address activities that are reasonable and necessary to promote competition and consumer welfare. First, we propose to permit the recovery of certain types of reasonable costs incurred to provide technology and services that enable access to EHI and facilitate the exchange and use of that information, provided certain conditions are met. Second, we propose to permit an actor to decline to provide access, exchange, or use of EHI in a manner that is infeasible, subject to a duty to provide a reasonable alternative. And, third, we propose an exception that would permit an actor to license interoperability elements on reasonable  \pagenum{7523}\ifhmode\expandafter\xspace\fi and non-discriminatory terms. These exceptions would be subject to strict conditions to ensure that they do not extend protection to practices that raise information blocking concerns.


          The last exception, set forth in VIII.D.7, recognizes that it may be reasonable and necessary for actors to make health IT temporarily unavailable for the benefit of the overall performance of health IT. This exception would permit an actor to make the operation of health IT unavailable in order to implement upgrades, repairs, and other changes.


          As context for the exceptions proposed below in VIII.D.4-D.6, we note that addressing information blocking is critical for promoting competition and innovation in health IT and for the delivery of health care services to consumers. Indeed, the information blocking provision itself expressly addresses practices that impede innovations and advancements in health information access, exchange, and use, including care delivery enabled by health IT (section 3022(a)(2)(C)(ii) of the PHSA). As discussed in section VIII.C.5.b.iii of this preamble, health IT developers of certified health IT, HIEs, HINs, and, in some instances, health care providers may exploit their control over interoperability elements to create barriers to entry for competing technologies and services that offer greater value for health IT customers and users, provide new or improved capabilities, and enable more robust access, exchange, and use of EHI.\textsuperscript{110}
             More than this, information blocking may harm competition not just in health IT markets, but also in markets for health care services.\textsuperscript{111}
             Dominant providers in these markets may leverage their control over technology to limit patient mobility and choice.\textsuperscript{112}
             They may also pressure independent providers to adopt expensive, hospital-centric technologies that do not suit their workflows, limit their ability to share information with unaffiliated providers, and make it difficult to adopt or use alternative technologies that could offer greater efficiency and other benefits.\textsuperscript{113}
             The technological dependence resulting from these practices can be a barrier to entry by would-be competitors. It can also make independent providers vulnerable to acquisition or induce them into exclusive arrangements that enhance the market power of incumbent providers, while preventing the formation of clinically-integrated products and networks that offer more choice and better value to consumers and purchasers of health care services.


          \insfootnote{\textsuperscript{110} \emph{See also} Martin Gaynor, Farzad Mostashari, and Paul B. Ginsberg, \emph{Making Health Care Markets Work: Competition Policy for Health Care,} 16-17 (Apr. 2017), available at \emph{\url{http://heinz.cmu.edu/news/news-detail/index.aspx?nid=3930.}}}


          \insfootnote{\textsuperscript{111} \emph{See, e.g.,} Keynote Address of FTC Chairwoman Edith Ramirez, Antitrust in Healthcare Conference Arlington, VA (May 12, 2016), \emph{available at \url{https://www.ftc.gov/system/files/documents/public_statements/950143/160519antitrusthealthcarekeynote.pdf.}}}


          \insfootnote{\textsuperscript{112} \emph{See, e.g.,} Martin Gaynor, Farzad Mostashari, and Paul B. Ginsberg, \emph{Making Health Care Markets Work: Competition Policy for Health Care,} 16-17 (Apr. 2017), \emph{available at \url{http://heinz.cmu.edu/news/news-detail/index.aspx?nid=3930.}}}


          \insfootnote{\textsuperscript{113}
              \emph{See, e.g.,}
              \emph{Healthcare Research Firm Toughens Survey Standards as More CIOs Reap the Profits of Reselling Vendor Software,} Black Book, \emph{available at \url{http://www.prweb.com/releases/2015/02/prweb12530856.htm;}} Arthur Allen, \emph{Connecticut Law Bans EHR-linked Information Blocking, Politico.com} (Oct. 29, 2015).}


          Section 3022(a)(5) of the PHSA provides that the Secretary may consult with the Federal Trade Commission (FTC) in defining practices that do not constitute information blocking because they are necessary to promote competition and consumer welfare. We appreciate the expertise and informal technical assistance of FTC staff, which we have taken into consideration in developing the exceptions described in VIII.D.4-D.6 of this preamble. We note that the language in the Cures Act regarding information blocking is substantively and substantially different from the language and goals in the antitrust laws enforced by the FTC. We view the Cures Act as authorizing ONC and OIG to regulate conduct that may be considered permissible under the antitrust laws. On this basis, this proposed rule requires that actors who control interoperability elements cooperate with individuals and entities that require those elements for the purpose of developing, disseminating, and enabling technologies and services that can interoperate with the actor's technology.


          We emphasize that ONC is taking this approach because we view patients as having an overwhelming interest in EHI about themselves, and particularly observational health information (see the discussion in section VIII.C.4.b of this preamble). As such, access to EHI, and the EHI itself, should not be traded or sold by those actors who are custodians of EHI or who control its access, exchange, or use. We emphasize that such actors should not be able to charge fees for providing electronic access, exchange, or use of patients' EHI. We propose that actors should be required to share EHI unless they are prohibited from doing so under an existing law or are covered by one of the exceptions detailed in this preamble. In addition, any remedy sought or action taken by HHS under the information blocking provision would be independent from the antitrust laws and would not prevent FTC or DOJ from taking action with regard to the same actor or conduct.


          We request comment on the following seven proposed exceptions, including whether they will achieve our stated policy goals.


          \subsubsection{1. Preventing Harm}

[Omitted.]

          \subsubsection{2. Promoting the Privacy of EHI}

[Omitted.]

          \subsubsection{3. Promoting the Security of EHI}

[Omitted.]

          \subsubsection{4. Recovering Costs Reasonably Incurred}

[Omitted.]

          \subsubsection{5. Responding To Requests That are Infeasible}

[Omitted.]

          \subsubsection{6. Licensing of Interoperability Elements on Reasonable and Non-Discriminatory Terms }

          We propose to establish an exception to the information blocking provision that would permit actors to license interoperability elements on reasonable and non-discriminatory (RAND) terms, provided that certain conditions are met. The exception and corresponding conditions are set forth in the proposed regulation text in \textsection{} 171.206. As discussed in section VIII.C.5.a of this preamble, the information blocking provision would be implicated if an actor were to refuse to license or allow the disclosure of interoperability elements to persons who require those elements to develop and provide interoperable technologies or services—including those that might complement or compete with the actor's own technology or services. Moreover, the information blocking provision would be implicated if the actor licensed such interoperability elements subject to terms or conditions that have the purpose or effect of excluding or discouraging competitors, rivals, or other persons from engaging in these pro-competitive and interoperability-enhancing activities. Thus, this licensing requirement would apply in both vertical and horizontal relationships. For instance, it would apply when a developer in a vertical relationship to the actor—a network in this example—wants to use interoperability elements in order to access the EHI maintained in the actor's network. The requirement would also apply when a rival network in a horizontal relationship to the actor (network) wants to use interoperability elements so that its network can be compatible with the applications that have already been developed for use with the actor's network.



          We note that some licensees do not require the interoperability elements to develop products or services that can be interoperable with the actor's health IT. For instance, there may be firms that simply want to license the actor's technology for use in developing their own interoperability elements. Their interest would be for access to the technology itself—not for the use of the technology to interoperate with either the actor or its customers. This may be the case, for example, if the relevant intellectual property included patents that were applicable to other information technology applications outside of health IT. In such cases, the actor's licensing of its patents in such a  \pagenum{7545}\ifhmode\expandafter\xspace\fi context would \emph{not} implicate the information blocking provision.


          Below are examples of situations that \emph{would implicate} the information blocking provision (these examples are not exhaustive):


          • An actor refuses to negotiate a license after receiving a request from a developer.


          • An actor offers a license at the request of a developer, but only at a royalty rate that exceeds a RAND rate.


          • An actor offers a license to a competitor at a royalty rate significantly higher than was offered to a party not in direct competition with the actor.


          • An actor files a patent infringement lawsuit against a developer without first offering to negotiate a license on RAND terms.



          There are compelling reasons for this prohibition. In our experience, contractual and intellectual property rights are frequently used to extract rents for access to EHI or to prevent competition from developers of interoperable technologies and services (\emph{see} section VIII.C.5.c.iv. of this preamble). These practices frustrate access, exchange, and use of EHI and stifle competition and innovation in the health IT sector. As a case in point, even following the enactment of the Cures Act, some EHR developers are selectively prohibiting—whether expressly or through commercially unreasonable terms—the disclosure or use of technical interoperability information required for third-party applications to be able to access, exchange, and use EHI maintained in EHR systems. This limits health care providers' use of the EHI maintained on their behalf to the particular capabilities and use cases that their EHR developer happens to support. More than this, by limiting the ability of providers to choose what applications and technologies they can use with their EHR systems, these practices close off the market to innovative applications and services that providers and other stakeholders need to deliver greater value and choice to health care purchasers and consumers.


          Despite these serious concerns, we recognize that the definition of information blocking may be broader than necessary and could have unintended consequences. In contrast to the practices described above, we believe it is generally appropriate for actors to license their intellectual property (IP) on RAND terms that do not block interoperability. Provided certain conditions are met, we believe that these practices would further the goals of the information blocking provision by allowing actors to protect the value of their innovations and earn returns on the investments they have made to develop, maintain, and update those innovations. This in turn will protect future incentives to invest in, develop, and disseminate interoperable technologies and services. Conversely, if actors cannot (or believe they cannot) protect and commercialize their innovations, they may not engage in these productive activities that improve access, exchange, and use of EHI.


          While we believe this exception is necessary to promote competition and consumer welfare, we are highly sensitive to the danger that actors will continue to use their contractual and IP rights to interfere with access, exchange, and use of EHI, undermining the information blocking provision's fundamental objectives. For this reason, the exception would be subject to strict conditions to ensure, among other things, that actors license interoperability elements on RAND terms and that they do not impose collateral terms or engage in other practices that would impede the use of the interoperability elements or otherwise undermine the intent of this exception.


          We acknowledge that preventing intellectual property holders from extracting rents for access to EHI may differ from standard intellectual property policy. Absent specific circumstances, IP holders are generally free to negotiate with prospective licensees to determine the royalty to practice their IP, and this negotiated royalty frequently reflects the value the licensee would obtain from exercising those rights. However, in the context of EHI, we propose that a limitation on rents is essential due to the likelihood that rents will frustrate access, exchange, and use of EHI, particularly because of the power dynamics that exist in the health IT market.


          We remind readers that actors are not required to seek the protection of this (or any other) exception. If an actor does not want to license a particular technology, it may choose to comply with the information blocking provision in another way, such as by developing and providing alternative means of accessing, exchanging, and using EHI that are similarly efficient and efficacious. The purpose of this exception is not to dictate a licensing scheme for all, or even most, health IT, but rather to provide a tailored “safe harbor” that will provide clear expectations for those who desire it.


          \subsubsection{i. Reasonable and Non-Discriminatory (RAND) Terms}

          We propose to require, as a condition of this exception, that any terms upon which an actor licenses interoperability elements must be reasonable and non-discriminatory (RAND). As discussed below, commitments to license technology on RAND terms are frequently required in the context of standards development organizations (SDOs), and we believe that the practical and policy considerations that have led SDOs to adopt these policies are related in many respects to the information blocking concerns presented when an actor exploits control over interoperability elements to extract economic rents or impede the development or use of interoperable technologies and services.


          We recognize that strong legal protections for IP rights can promote competition and innovation.\textsuperscript{125}
             Nevertheless, IP rights can also be misused in ways that undermine these goals.\textsuperscript{126}
             We believe this potential for abuse is heightened when the IP rights pertain to functional aspects of technology that are essential to enabling interoperability. As an important example, a technology developer may encourage the inclusion of its technology in an industry standard created by an SDO while not disclosing that it has IP rights in that technology. After the SDO incorporates the technology into its standard, and industry begins to make investments tied to the standard, the IP-holder may then assert its IP rights and demand royalties or license terms that it could not have achieved before the standard was adopted because companies would incur substantial switching costs to abandon initial designs or adopt different products.\textsuperscript{127}
             To address these  \pagenum{7546}\ifhmode\expandafter\xspace\fi types of concerns, while balancing the legitimate interests and incentives of IP owners, many SDOs now have policies requiring members who contribute technologies to a standard to voluntarily commit to license that technology on RAND terms and will consider whether firms have made voluntary RAND commitments when weighing whether to include their technology in standards.\textsuperscript{128}
             While this commitment to license on RAND terms is voluntary as compared to our proposed requirement to use RAND terms, it serves to illustrate how RAND terms can be used to address such concerns.


          \insfootnote{\textsuperscript{125} \emph{See} FTC and DOJ Antitrust Guidelines for the Licensing of Intellectual Property, at 2 (2017), \emph{\url{https://www.ftc.gov/system/files/documents/public_statements/1049793/ip_guidelines_2017.pdf.}}}


          \insfootnote{\textsuperscript{126} \emph{See Assessment Techs. of WI, LLC} v. \emph{WIREdata, Inc.,} 350 F.3d 640, 644-45 (7th Cir. 2003); \emph{Sega Enterprises Ltd.} v. \emph{Accolade, Inc.,} 977 F.2d 1510, 1520-28 (9th Cir. 1992); \emph{Sony Computer Entertainment, Inc.} v. \emph{Connectix Corp.,} 203 F.3d 596, 602-08 (9th Cir. 2000); \emph{Bateman} v. \emph{Mnemonics, Inc.,} 79 F.3d 1532, 1539-40 n. 18 (11th Cir. 1996); \emph{Atari Games Corp.} v. \emph{Nintendo of America, Inc.,} 975 F.2d 832, 842-44 (Fed. Cir. 1992).}


          \insfootnote{\textsuperscript{127} \emph{See} DOJ and FTC, Antitrust Enforcement and Intellectual Property Rights: Promoting Innovation and Competition, at 37-40 (Apr. 2017), \emph{\url{https://www.ftc.gov/sites/default/files/documents/reports/antitrust-enforcement-and-intellectual-property-rights-promoting-innovation-and-competition-report.s.department-justice-and-federal-trade-commission/p040101promotinginnovationandcompetitionrpt0704.pdf.}}}


          \insfootnote{\textsuperscript{128} \emph{See, e.g., Microsoft Corp.} v. \emph{Motorola, Inc.,} No. C10-1823JLR, 2013 WL 2111217, at *6 (W.D. Wash. Apr. 25, 2013).}


          Similar concerns arise when actors who control proprietary interoperability elements demand royalties or license terms from competitors or other persons who are technologically dependent on the use of those interoperability elements. As discussed in section VIII.C.5 of this preamble, to the extent that the interoperability elements are essential to enable the efficient access, exchange, or use of EHI by particular persons or for particular purposes, any practice by the actor that could impede the use of the interoperability elements for that purpose—or that could unnecessarily increase the cost or other burden of using the elements for that purpose—would give rise to an obvious risk of interference with access, exchange, or use of EHI under the information blocking provision.


          We believe that a RAND requirement would balance the need for robust IP protections with the need to ensure that this proposed exception does not permit actors to exercise their IP or other proprietary rights in inappropriate ways that block the development, adoption, or use of interoperable technologies and services. The exercise of IP rights in these ways is incompatible with the information blocking provision, which protects the investments that taxpayers and the health care industry have made to adopt technologies that will enable the efficient sharing of EHI  to benefit consumers and the health care system. While actors are entitled to protect and exercise their IP rights, to benefit from this exception to the information blocking provision they must do so in a reasonable and non-discriminatory manner that does not undermine these efforts and impede the appropriate flow of EHI.


          Accordingly, we propose that, to qualify for this exception, an actor must license requested interoperability elements on RAND terms. To comply with this condition, any terms or conditions under which the actor discloses or allows the use of interoperability elements must meet several requirements set forth below. These requirements apply to both price terms (such as royalties and license fees) and other terms, such as conditions or limitations on access to interoperability elements or the purposes for which they can be used.


          \subsubsection{Responding To Requests}


          We propose that, upon receiving a request to license or use interoperability elements, an actor would be required to respond to the requestor within 10 business days from receipt of the request. We note that the request could be made to “license” or “use” the interoperability elements because a requestor may not always know that “license” is the legal mechanism for “use” when making the request. This provision is intended to ensure that a requestor is given an opportunity to license \emph{and} use interoperability elements. As such, the requirement for responding to requests should not be limited to requests to “license.”



          In order to meet this requirement, the actor would be required to respond to the requestor within 10 business days from the receipt of the request by: (1) Negotiating with the requestor in a RAND fashion to identify the interoperability elements that are needed; and (2) offering an appropriate license with RAND terms, consistent with its other obligations under this exception. We emphasize that, in order to qualify for this proposed exception, the actor is only required to \emph{negotiate} with the requestor in a RAND fashion and to \emph{offer} a license with RAND terms. The actor is not required to \emph{grant} a license in all instances. For example, the actor would not be required to grant a license if the requestor refuses an actor's offer to license interoperability elements on RAND terms.


          We emphasize that there would be circumstances under which the actor could pursue legal action against parties that infringe its intellectual property whilst complying with this exception. For instance, an actor could bring legal action if a firm appropriates the actor's intellectual property without requesting a license or after refusing to accept a license on RAND terms.


          We do not propose a set timeframe for when the negotiations must be resolved because it is difficult to predict the duration of such negotiations. For instance, there could be situations when the actor and requestor meet once and the actor makes a RAND offer that is immediately accepted by the requestor. However, there could be other situations when the requestor and actor each make counteroffers, which would extend the negotiations.


          We request comment on whether 10 business days is an appropriate amount of time for the actor to respond to the requestor. In proposing this timeframe, we considered the urgency of certain requests to license interoperability elements and our expectation that developers would have standard licenses at their disposal that could be adapted in these situations. We considered proposing response timeframes ranging from 5 business days to 15 business days. We also considered proposing two separate timeframes for: (1) Negotiating with the requestor; and (2) offering the license. If commenters prefer a different response timeframe or approach than proposed, we request that commenters explain their rationale with as much detail as possible.


          In addition, we query whether we should create set limits for: (1) The amount of time the requestor has to accept the actor's initial offer or make a counteroffer; (2) if the requestor makes a counteroffer, the amount of time the actor has to accept the requestor's counteroffer or make its own counteroffer; and (3) an allowable number of counteroffers in negotiations.


          \subsubsection{Scope of Rights}

          To qualify for this proposed exception, we propose that the actor must license the requested interoperability elements with all rights necessary to access and use the interoperability elements for the following purposes, as applicable:


          • All rights necessary to access and use the interoperability elements for the purpose of developing products or services that are interoperable with the actor's health IT or with health IT under the actor's control and/or any third party who currently uses the actor's interoperability elements to interoperate with the actor's health IT or health IT under the actor's control. These rights would include the right to incorporate and use the interoperability elements in the licensee's own technology to the extent necessary to accomplish this purpose.


          • All rights necessary to market, offer, and distribute the interoperable products and services described above to potential customers and users, including the right to copy or disclose the interoperability elements as necessary to accomplish this purpose.



          • All rights necessary to enable the use of the interoperable products or services in production environments,  \pagenum{7547}\ifhmode\expandafter\xspace\fi including using the interoperability elements to access and enable the exchange and use of electronic health information.


          We request comment on whether these rights are sufficiently inclusive to support licensees in developing interoperable technologies, bringing them to market, and deploying them for use in production environments. We also request comment on the breadth of these required rights and if they should be subject to any limitations that would not interfere with the uses we have described above.


          \subsubsection{Reasonable Royalty}

          As a condition of this exception, we propose that if an actor charges a royalty for the use of interoperability elements, the royalty base and rate must be reasonable. Consistent with the requirements for demonstrating that activities and practices meet the conditions of an exception proposed in section VIII.C.6.c, the actor would need to show that the royalty base was reasonable and that the royalty was within a reasonable range for the interoperability elements at issue. Importantly, we note that the reasonableness of any royalties would be assessed solely on basis of the independent value of the actor's technology to the licensee's product,\textsuperscript{129}
            
            \emph{not} on any strategic value stemming from the actor's control over essential means of accessing, exchanging, or using electronic health information. For instance, the reasonableness of royalties could not be assessed based on the strategic value stemming from the adoption of the technology by customers or users, the switching costs associated with the technology, or other circumstances of technological dependence described elsewhere in this preamble (see section VIII.C.5). We note that “strategic value” would stem from the actor's control over essential means of accessing, exchanging, or using electronic health information. Limiting a reasonable royalty to the value of the technology isolated from strategic value is similar in concept to apportionment of reasonable royalties for the infringement of standard essential patents (SEPs).\textsuperscript{130}
             In our context, permitting an actor to charge a royalty on the basis of these considerations would effectively allow the actor to extract rents on access, exchange, and use of EHI, which is contrary to the goals of the information blocking provision.


          \insfootnote{\textsuperscript{129} \emph{See Ericsson, Inc.} v. \emph{D-Link Systems, Inc.,} 773 F.3d 1201, 1226; 1232 (Fed. Cir. 2014).}


          \insfootnote{\textsuperscript{130} \emph{See, e.g., Georgia-Pacific Corp.} v. \emph{United States Plywood Corp.,} 318 F. Supp 1116 (S.D.N.Y. 1970) (utilizing the more common approach).}


          In evaluating the actor's assertions and evidence that the royalty was reasonable, we propose that ONC may consider the following factors:


          • The royalties received by the actor for the licensing of the proprietary elements in other circumstances comparable to RAND-licensing circumstances.


          • The rates paid by the licensee for the use of other comparable proprietary elements.


          • The nature and scope of the license.


          • The effect of the proprietary elements in promoting sales of other products of the licensee and the licensor, taking into account only the contribution of the elements themselves and not of the enhanced interoperability that they enable.


          • The utility and advantages of the actor's interoperability element over the existing technology, if any, that had been used to achieve a similar level of access, exchange, or use of EHI.


          • The contribution of the elements to the technical capabilities of the licensee's products, taking into account only the value of the elements themselves and not the enhanced interoperability that they enable.


          • The portion of the profit or of the selling price that may be customary in the particular business or in comparable businesses to allow for the use of the proprietary elements or analogous elements that are also covered by RAND commitments.


          • The portion of the realizable profit that should be credited to the proprietary elements as distinguished from non-proprietary elements, the manufacturing process, business risks, significant features or improvements added by the licensee, or the strategic value resulting from the network effects, switching costs, or other effects of the adoption of the actor's technology.


          • The opinion testimony of qualified experts.


          • The amount that a licensor and a licensee would have agreed upon (at the time the licensee began using the elements) if both were considering the RAND obligation under this exception and its purposes, and had been reasonably and voluntarily trying to reach an agreement.



          These factors mirror those used by courts that have examined the reasonableness of royalties charged pursuant to a commitment to an SDO to license standard-essential technologies on RAND terms (\emph{see Microsoft Corp.} v. \emph{Motorola, Inc.;} \textsuperscript{131}
             In re \emph{Innovatio IP Ventures, LLC Patent Litig.;} \textsuperscript{132}
             and \emph{Realtek Semiconductor Corp.} v. \emph{LSI Corp} \textsuperscript{133}
            ). However, we have adapted the factors to the information blocking context as follows. In the SDO context, the RAND requirement mitigates the risk that patent-holders will engage in “hold up”—that is, charging excessive royalties that do not reflect the value of their contributions to the standard, but rather reflect the costs associated with switching to alternative technologies after a standard is adopted—and that the cumulative effect of such royalties will make the standard too expensive to implement—a problem called “royalty stacking.” \textsuperscript{134}
             To address the risks of hold-up and royalty stacking in the standards development context, a RAND license should compensate a patentee for their technical contribution to the technology embodied in a standard, but should not compensate them for mere inclusion in the standard.


          \insfootnote{\textsuperscript{131} Case No. 10-cv-1823 JLR, 2013 WL 2111217 (W.D.Wash. Apr. 25, 2013).}


          \insfootnote{\textsuperscript{132} MDL 2303, 2013 WL 5593609 (N.D.Ill. Oct. 3, 2013).}


          \insfootnote{\textsuperscript{133} Case No. 5:12-cv-03451-RMW, 2014 WL 46997 (N.D.Cal. Jan. 6, 2014).}


          \insfootnote{\textsuperscript{134} \emph{Microsoft Corp.} v. \emph{Motorola, Inc.,} 864 F.Supp.2d 1023, 1027 (W.D.Wash. 2012).}



          Similarly, in the context of information blocking, we propose the RAND inquiry focuses on whether the royalty demanded by the actor represents the independent value of the actor's proprietary technology. We propose that if the actor has licensed the interoperability element through a standards development organization in accordance with such organization's policies regarding the licensing of standards-essential technologies on reasonable and non-discriminatory terms, the actor may charge a royalty that is consistent with such policies. Rather than asking whether the royalty inappropriately captures additional value derived from the technology's inclusion in the industry standard, we would ask whether the actor is charging a royalty that is not based on the value of its technology (embodied in the interoperability elements) but rather includes the strategic value stemming from the adoption of that technology by customers or users. Thus, under this proposed approach and the factors set forth above, we would consider the technical contribution of the actor's interoperability elements to the licensee's products—such as any proprietary capabilities or features that the licensee uses in its product—but would screen out any functional aspects of the actor's technology that are used only to establish interoperability and enable EHI to be accessed, exchanged, and used. Additionally, we propose that  \pagenum{7548}\ifhmode\expandafter\xspace\fi to address the potential risk of royalty stacking we would need to consider the aggregate royalties that would apply if owners of other essential interoperability elements made royalty demands of the implementer. Specifically, we propose that, to qualify for this exception, the actor must grant licenses on terms that are objectively commercially reasonable taking into account the overall licensing situation, including the cost to the licensee of obtaining other interoperability elements that are important for the viability of the products for which it is seeking to license interoperability elements from the actor.


          We clarify that, as proposed, this condition would not preclude an actor from licensing its interoperability elements pursuant to an existing RAND commitment to an SDO. We also note that, in addition to complying with the requirements described above, to meet this proposed condition any royalties charged must meet the condition, proposed separately below, that any license terms be non-discriminatory.


          We request comment on these aspects of the proposed exception. Commenters are encouraged to consider, in particular, whether the factors and approach we have described will be administrable and appropriately balance the unreasonable blocking by actors of the use of essential interoperability elements with the need to provide adequate assurance to investors and innovators that they will be able to earn a reasonable return on their investments in interoperable technologies. If our proposed approach does not adequately balance these concerns or would not achieve our stated policy goals, we ask that commenters suggest revisions or alternative approaches. We ask that such comments be as detailed as possible and provide rigorous economic justifications for any suggested revisions or alternative approaches.


          \subsubsection{Non-Discriminatory Terms}

          We propose that for this exception to apply the terms on which an actor licenses and otherwise provides interoperability elements must be non-discriminatory. This requirement would apply to both price and non-price terms, and thus would apply to the royalty terms discussed immediately above as well as other types of terms that may be included in licensing agreements or other agreements related to the provision or use of interoperability elements.


          To comply with this condition, the terms on which the actor licensed the interoperability elements must be based on criteria that the actor applied uniformly for all substantially similar or similarly situated classes of persons and requests. This requirement addresses a root cause of information blocking. In order to be considered non-discriminatory, such criteria would have to be objective and verifiable, not based on the actor's subjective judgment or discretion. We emphasize that this proposal does not mean that the actor must apply the same terms for all persons or classes of persons requesting a license. However, any differences in terms would have to be based on actual differences in the costs that the actor incurred or other reasonable and non-discriminatory criteria. Moreover, we propose that any criteria upon which an actor varies its terms or conditions would have to be both competitively neutral—meaning that the criteria are not based in any part on whether the requestor or other person is a competitor, potential competitor, or will be using EHI obtained via the interoperability elements in a way that facilitates competition with the actor—and neutral as to the revenue or other value that the requestor may be derived from access, exchange, or use of the EHI obtained via the interoperability elements, including any secondary use of such EHI. We believe these limitations are necessary in light of the potential for actors to use their control over interoperability elements to engage in discriminatory practices that create unreasonable barriers or costs for persons seeking to develop, offer, or use interoperable technologies to expand access and enhance the exchange and use of EHI.



          To clarify our expectations for this proposed condition, we provide the following illustration. Consider an EHR developer that establishes an “app store” through which third-party developers can license the EHR developer's proprietary APIs, which we assume are separate from the APIs required by the API Condition of Certification proposed in \textsection{} 170.404. The EHR developer could charge a reasonable royalty and impose other reasonable terms to license these interoperability elements. The terms and conditions could vary based on neutral, objectively verifiable, and uniformly applied criteria. These might include, for example, significantly greater resources consumed by certain types of apps, such as those that export large volumes of data on a continuous basis, or the heightened risks associated with apps designed to “write” data to the EHR database or to run natively within the EHR's user interface. In contrast, the EHR developer could \emph{not} vary its terms and conditions based on subjective criteria, such as whether it thinks an app will be “popular” or is a “good fit” for its ecosystem. Nor could it offer different terms or conditions on the basis of objective criteria that are not competitively neutral, such as whether an app “connects to” other technologies or services, provides capabilities that the EHR developer plans to incorporate in a future release of its technology, or enables an efficient means for customers to export data for use in other databases or technologies that compete directly with the EHR developer. Similarly, the EHR developer could not set different terms or conditions based on how much revenue or other value the app might generate from the information it collects through the APIs, such as by introducing a revenue-sharing requirement for apps that use data for secondary purposes that are very lucrative and for which the EHR developer would like a “piece of the pie.” Such practices would disqualify the actor from this exception and would implicate the information blocking provision.



          The foregoing conditions are not intended to limit an actor's flexibility to set different terms based on legitimate differences in the costs to different classes of persons or in response to different classes of requests, so long as any such classification was in fact based on neutral criteria (in the sense described above) that are objectively verifiable and were applied in a consistent manner for persons and/or requests within each class. As an important example, the proposed condition would not preclude a covered actor from pursuing strategic partnerships, joint ventures, co-marketing agreements, cross-licensing agreements, and other similar types of commercial arrangements under which it provides more favorable terms than for other persons with whom it has a more arms-length relationship. In these instances the actor should have no difficulty identifying substantial and verifiable efficiencies that demonstrate that any variations in its terms and conditions were based on objective and neutral criteria. We do note an important caveat, however, specifically that a health IT developer of certified health IT who is an “API Technology Supplier” under the Condition of Certification proposed in \textsection{} 170.404 would not be permitted to offer different terms in connection with the APIs required by that Condition of Certification. As discussed in section VII.B.4 of this preamble, we propose that API Technology Suppliers are  \pagenum{7549}\ifhmode\expandafter\xspace\fi required to make these APIs available on terms that are no less favorable than provided to their own customers, suppliers, partners, and other persons with whom they have a business relationship. As noted below towards the end of our discussion of this exception to the information blocking provision, the exception incorporates the API Condition of Certification's requirements in full for all health IT developers subject to that condition.


          We welcome comments on the foregoing condition and requirements.


          \subsubsection{Collateral Terms}


          We propose five additional conditions that would reinforce the requirements of this exception discussed above. These additional conditions would provide bright-line prohibitions for certain types of collateral terms or agreements that we believe are inherently likely to interfere with access, exchange, or use of EHI. We propose that any attempt to \emph{require} a licensee or its agents or contractors to do or agree to do any of the following would disqualify the actor from this exception and would be suspect under the information blocking provision.


          First, the actor must not require the licensee or its agents or contractors to not compete with the actor in any product, service, or market, including markets for goods and services, technologies, and research and development. We are aware that such agreements have been used to either directly exclude suppliers of interoperable technologies and services from the market or to create exclusivity that reduces the range of technologies and options available to health care providers and other health IT customers and users.


          Second, and for similar reasons, the actor must not require the licensee or its agents or contractors to deal exclusively with the actor in any product, service, or market, including markets for goods and services, technologies, and research and development.



          Third, the actor must not require the licensee or its agents or contractors to obtain additional licenses, products, or services that are not related to or can be unbundled from the requested interoperability elements. This condition reinforces the condition described earlier requiring that any royalties charged by the actor for the use of interoperability elements be reasonable. Without this condition, we believe that an actor could require a licensee to take a license to additional interoperability elements that the licensee does not need or want, which could enable the actor to extract royalties that are inconsistent with its RAND obligations under this exception. We clarify that this condition would not preclude an actor and a willing licensee from agreeing to such an arrangement, so long as the arrangement was not \emph{required.}
          



          Fourth, the actor must not condition the use of interoperability elements on a requirement or agreement to license, grant, assign, or transfer the licensee's own IP to the actor. We believe it is inconsistent with the actor's RAND licensing obligations under this exception, and would raise information blocking concerns, for an actor to use its control over interoperability elements as leverage to obtain a “grant back” of IP rights or other consideration whose value may exceed that of a reasonable royalty. Consistent with our approach under other conditions of this exception, this condition would not preclude an actor and a willing licensee from agreeing to a cross-licensing, co-marketing, or other agreement if they so choose. However, the actor cannot \emph{require} the licensee to enter into such an agreement. The actor must offer the option of licensing the interoperability elements without a promise to provide consideration beyond a reasonable royalty. We note that in the SDO context, it can sometimes be consistent with RAND terms to require that an SEP licensee also grant a cross-license to any SEPs that it holds, provided that the cross-license is limited to patents essential to the licensed standard. In this way, this condition differs from licensing in the SDO context.


          Finally, the actor must not condition the use of interoperability elements on a requirement or agreement to pay a fee of any kind whatsoever unless the fee meets either the narrowly crafted condition to this exception for a reasonable royalty, or, alternatively, the fee satisfies the separate exception proposed in \textsection{} 171.204, which permits the recovery of certain costs reasonably incurred. As noted in section VIII.D.4, that exception generally does not allow for the recovery of royalties or other fees associated with intangible assets. However, the exception does allow for the reasonable and actual development and acquisition costs of such assets.


          We request comment on the categorical exclusions outlined above. In particular, we encourage commenters to weigh in on our assumption that these practices are inherently likely to interfere with access, exchange, or use of EHI. We also encourage commenters to suggest any conceivable benefits that these practices might offer for interoperability or for competition and consumers that we might have overlooked. Again, we ask that to the extent possible commenters provide detailed economic rationale in support of their comments.


          \subsubsection{Non-Disclosure Agreement}

          We propose that an actor would be permitted under this exception to require a licensee to agree to a confidentiality or non-disclosure agreement (NDA) to protect the actor's trade secrets, provided that the NDA is no broader than necessary to prevent the unauthorized disclosure of the actor's trade secrets. Further, we propose that the actor would have to identify (in the NDA) the specific information that it claims as trade secrets, and that such information would have to meet definition of a trade secret under applicable law. We believe these safeguards are necessary to ensure that the NDA is not used to impose restrictions or burdensome requirements that are not actually necessary to protect the actor's trade secrets and that impede the use of the interoperability elements. The use of an NDA for such purposes would preclude an actor from qualifying for this exception and would implicate the information blocking provision. We note that if the actor is a health IT developer of certified health IT, it may be subject to the Condition of Certification proposed in \textsection{} 170.403, which prohibits certain health IT developer prohibitions and restrictions on communications about a health IT developer's technology and business practices. This exception would not in any way abrogate the developer's obligations to comply with that condition.


          We encourage comment on this condition of the proposed exception.


          \subsubsection{ii. Additional Requirements Relating to the Provision of Interoperability Elements}


          In addition to the conditions described above, we propose that an actor's practice would need to comply with additional conditions that ensure that actors who license interoperability elements on RAND terms do not engage in separate practices that impede the use of those elements or otherwise undermine the intent of this exception. These conditions are analogous to the conditions described in our proposal above concerning collateral terms but address a broader range of practices that may not be effected through the license agreements themselves or that occur separately from the licensing negotiations and other dealings between the actor and the licensee. Specifically, we propose that an actor would not qualify for this exception if it engaged in a practice that had the purpose or  \pagenum{7550}\ifhmode\expandafter\xspace\fi effect of impeding the efficient use of the interoperability elements to access, exchange, or use EHI for any permissible purpose; or the efficient development, distribution, deployment, or use of an interoperable product or service for which there is actual or potential demand. As an illustration, the exception would not apply if the developer licensed its proprietary APIs for use by third-party apps but then prevented or delayed the use of those apps in production environments by, for example, restricting or discouraging customers from enabling the use of the apps, or engaging in “gate keeping” practices, such as requiring apps to go through a vetting process and then applying that process in a discriminatory or unreasonable manner.


          Finally, to ensure the actor's commitments under this exception are durable, we propose one additional safeguard: An actor cannot avail itself of this exception if, having licensed the interoperability elements, the actor makes changes to the elements or its technology that “break” compatibility or otherwise degrade the performance or interoperability of the licensee's products or services. We believe this condition is crucial given the ease with which an actor could make subtle “tweaks” to its technology or related services that could disrupt the use of the licensee's compatible technologies or services and result in substantial competitive and consumer injury.



          We clarify and emphasize that this proposed condition would in \emph{no way} prevent an actor from making improvements to its technology or responding to the needs of its own customers or users. However, to benefit from the exception, the actor's practice would need to be necessary to accomplish these purposes and the actor must have afforded the licensee a reasonable opportunity under the circumstances to update its technology to maintain interoperability. We also recognize that an actor may have to suspend access or make other changes immediately and without prior notice in response to legitimate privacy, security, or patient safety-related exigencies. Such practices would be governed by the exceptions proposed in section VIII.D of this preamble and thus would not need to qualify for this exception.


          \subsubsection{iii. Compliance With Conditions of Certification}

          As a final condition of this proposed exception, we propose that health IT developers of certified health IT who are subject to the Conditions of Certification proposed in \textsection{}\textsection{} 170.402, 170.403, and 170.404 must comply with all requirements of those Conditions of Certification for all practices and at all relevant times. Several of the requirements of these conditions mirror those of this exception. However, in some instances the Conditions of Certification provide additional or more specific requirements that apply to the provision of interoperability elements by developers of certified health IT. For example, developers subject to the API Condition of Certification must make certain public APIs available on terms that are royalty free and no less favorable than provided to themselves and their customers, suppliers, partners, and other persons with whom they have a business relationship. These more prescriptive requirements reflect the specific obligations of health IT developers under the Program, including the duty to facilitate the access, exchange, and use of information from patients' electronic health records without special effort. A health IT developer of certified health IT's failure to comply with these and other certification requirements that specifically support interoperability would, in addition to precluding the developer from invoking this exception, be significant evidence of information blocking.


          \subsubsection{7. Maintaining and Improving Health IT Performance}

[Omitted.]

          \subsection{E. Additional Exceptions—Request for Information}

[Omitted.]

          \subsection{F. Complaint Process}

[Omitted.]

          \subsection{G. Disincentives for Health Care Providers—Request for Information}

          Section 3022(b)(2)(B) of the PHSA provides that any health care provider determined by the OIG to have committed information blocking shall be referred to the appropriate agency to be subject to appropriate disincentives using authorities under applicable federal law, as the Secretary sets forth through notice and comment rulemaking. However, we note that these disincentives may not cover the full range of conduct within the scope of section 3022(a)(1). We request information on disincentives or if modifying disincentives already available under existing HHS programs and regulations would provide for more effective deterrents.


          We also seek information on the implementation of section 3022(d)(4) of the PHSA, which provides that in carrying out section 3022(d) of the PHSA, the Secretary shall, to the extent possible, not duplicate penalty structures that would otherwise apply with respect to information blocking and the type of individual or entity involved as of the day before December 13, 2016—enactment of the Cures Act.


[The remainder is omitted.]
          
      
\end{document}
